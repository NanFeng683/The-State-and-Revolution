\chapter{公社社员这次尝试的英雄主义何在呢?} % 小章节名称
%\blfootnote{} % 底线批注,在所需段落前加入
%\pskip % 跳大行,段落间用
%\noindent % 书信体的开头
%\mbox{} % 插入个盒子,设定好的大小,用于隔开文本
%\leftskip= 文本左边腾出空格,用于文本排版
% \CJKunderdot{} % 文字下加点
% {\kaishu } % 更改字体为楷书

大家知道,在巴黎公社前几个月,即1870年秋,马克思曾经警告巴黎工人说,推翻政府的尝试是一种绝望的愚蠢举动。但是,1871年8月,当工人{\kaishu 被迫}进行决战,起义已经成为事实的时候,尽管当时有种种恶兆,马克思还是以欢欣鼓舞的心情来迎接无产阶级革命。马克思并没有用学究式的言论来非难“不合时宜的”运动,像臭名昭彰的俄国马克思主义叛徒普列汉诺夫那样。普列汉诺夫在1905年11月写了一些鼓舞工人农民进行斗争的文章,而在1905年12月以后,却以自由主义的论调大叫其“本来是不需要拿起武器的”。

然而,马克思不仅是因为公社成员表现了如他所说的“翻天覆地”的英雄主义而感到愉快。虽然这次群众性的革命运动没有达到目的,但是他在这次运动中看到了有极重大意义的历史经验,看到了全世界无产阶级革命的一定进步,看到了比几百种纲领和议论更为重要的实际步骤。分析这个经验,从这个经验中得到策略教训,根据这个经验来重新审查自己的理论,这就是马克思提出的任务。

马克思认为对“共产党宣言”必须做出的唯一“修改”,就是他根据巴黎公社社员的革命经验而做出的。

在“共产党宣言”德文新版上由两位作者签名的最后一篇序言,是1872年6月24日写成的。在这篇序言中,作者马克思和恩格斯说:“共产党宣言”的纲领“现在有一些地方已经过时了”。

\pskip 
\leftskip=10mm
\small

他们又说······“特别是巴黎公社已经证明:‘工人阶级不能简单地掌握现成的国家机器并运用它来达到自己的目的’”······$^{13}$

\leftskip=0mm
\normalsize
\pskip 

这段引文中单引号内的话,是两位作者从马克思的“法兰西内战”一书中借用来的。

总之,马克思和恩格斯认为巴黎公社这个基本的主要的教训具有非常重大的意义。“共产党宣言”再版时,他们就把这一点加进去作为重要的修改。

非常值得注意的正是这个重要的修改被机会主义者歪曲了。“共产党宣言”的读者,大概有十分之九,甚至有百分之九十九是不了解这个修改的意义的。我们在下面专论歪曲的那一章里,还要对这种歪曲加以详细说明。现在只须指出,对于我们前面引证的马克思的那句名言,流行着一种庸俗的“了解”,认为马克思在这里是强调缓慢发展的思想,不主张夺取政权等等。

\blfootnote{[1]库格曼(Ludwig Kugelmann,1828~1902),德国社会主义者,第一国际会员,医生。库格曼在1862~1874年与马克思的通信成为研究马克思主义发展史和《资本论》创作史的珍贵文献。}
\blfootnote{$^{*}$ 见“列宁全集”1959年人民出版社第12卷第96页-105页——编者注}
实际上\CJKunderdot{\kaishu 恰巧相反}。马克思的意思是说工人阶级应当\CJKunderdot{\kaishu 打碎} 和\CJKunderdot{\kaishu 摧毁} “现成的国家机器”,而不是简单地夺取这个机器。
1871年4月12日,正是建立起巴黎公社的时候,马克思在给库格曼$^{[1]}$的信中写道:

\pskip 
\leftskip=10mm
\small

······“如果你读一下我的‘拿破仑第三政变记’的最后一章,你就会看到,我认为法国革命下一次的尝试不应该像以前那样,把官僚军事机器从一些人的手里转到另一些人的手里,而是应该把它\CJKunderdot{\kaishu 摧毁} ”(着重号是马克思加的;原文是zerbrechen),“这正是大陆上任何一次真正的人民革命的先决条件。我们英勇的巴黎同志们的尝试正是这样”(见“新时代”杂志第20卷1901-1902年第1期第709页)$^{14}$。(马克思的“给库格曼的信”至少有两种俄文版本,其中有一种是由我校订和作序的$^{*}$。)

\leftskip=0mm
\normalsize
\pskip 

“摧毁官僚军事国家机器”这几个字,已经把马克思主义关于无产阶级在革命中对国家的任务问题的主要教训,简单地表明了。现在占统治地位的考茨基主义在“解释”马克思主义的时候不仅把这个教训忘了,而且还公然歪曲它!

马克思从“拿破仑第三政变记”中摘录的话,我们在前面已经全部引用了。

在以上引证的马克思的那段论述中,有两个地方是值得特别指出的。第一、他把他的结论只限于大陆。这在1871年是可以理解的,那时英国还是纯粹的资本主义国家的典型,还没有军阀制度,大体上也没有官僚制度。所以马克思把英国除外,当时英国革命,甚至人民革命,很有可能{\kaishu 不必}以破坏“现成的国家机器为先决条件”。

现在到了1917年,在第一次帝国主义大战时期,马克思的这种有限制的说法已经不适用了。英国和美国这两个全世界最大的和最后的盎格鲁撒克逊“自由制”(从没有军阀制度和官僚制度这个意义来说)的代表,已经完全滚到一般欧洲式的、支配一切、压制一切的官僚军事机关的污浊血腥的泥潭中去了。现在,无论在英国或美国,\CJKunderdot{\kaishu 打碎}、\CJKunderdot{\kaishu 摧毁} “现成的”(1914-1917年间,这两个国家就达到了“欧洲式的”一般帝国主义的完备程度)“国家机器”,是“任何一次真正的人民革命的先决条件”。

\blfootnote{[2]彼得·伯恩哈多维奇·司徒卢威(П.Б Стюхловый),俄国经济学家、哲学家,“合法马克思主义”代表人物。其对马克思主义作资产阶级改良主义的歪曲,批判民粹派,美化资本主义制度等。}
第二、马克思说破坏官僚军事国家机器是“任何一次真正人民革命的先决条件”,这个非常深刻的见解是值得特别注意的。“人民”革命这一概念出自马克思的口中似乎是很奇怪的,俄国的普列汉诺夫分子和孟什维克,这些自命为马克思主义者的司徒卢威$^{[2]}$的信徒,也许会说马克思说这句话是“失言”。他们把马克思主义曲解为非常贫乏的自由主义,在他们看来,除了资产阶级革命和无产阶级阶级革命的对立以外,再没有任何东西,而且他们对这种对立的了解也是非常死板的。

\blfootnote{[3]即1910年共和主义者革命}
\blfootnote{[4]即1908~1909年青年土耳其党人发动并领导的以反对阿卜杜勒哈米德二世封建专制统治制度,实行君主立宪制为主要目标的资产阶级革命。}
如果以20世纪的革命为例,那末无论葡萄牙革命$^{[3]}$或土耳其革命$^{[4]}$,当然都应该算是资产阶级革命。但是无论前者或后者,都不是“人民”革命,因为人民群众,大多数人民,在这两次革命中显然都没有积极地、独立地为自己的经济要求和政治要求进行斗争。恰恰相反,1905-1907年的俄国资产阶级革命,虽然没有取得像葡萄牙革命和土耳其革命某些时候得到的那些“辉煌”成绩,但这无疑是一次“真正人民的”革命,因为人民群众,大多数人民,遭受压迫和剥削的社会最“底层”,都站起来了,提出{\kaishu 自己的}要求,要按照{\kaishu 自己的}方式建立新社会来代替正在破坏的旧社会,他们影响了整个革命的进程。

1871年,欧洲大陆上任何一个国家的无产阶级都没有占人民的多说。当时只有把无产阶级和农民都吸引到运动中来的革命,才真正是人民多数的“人民”革命。当时的“人民”就是由这两个阶级构成的。这两个阶级因为都受“官僚军事国家机器”的压迫、摧残和剥削而联合起来{\kaishu 打碎}这个机器,{\kaishu 摧毁}这个机器,——这就是“人民”,人民的多数,即工人和大多数农民的真正利益,这就是贫苦农民同无产者自由联盟的“先决条件”,没有这个联盟,民主制就不能稳固,社会主义改造就不能完成。

大家知道,巴黎公社会为自己开辟过实现这个联盟的道路,但是,由于许多内部和外部的原因,没有达到目的。

所以马克思在谈到“真正人民的革命”时,丝毫也没有忘记小资产阶级的特点(关于这些特点,他说得很多而且常常说),他极严格地估计了1871年欧洲大陆上多数国家中实际的阶级对比关系。另一方面,他又指出,“打碎”国家机器是工人和农民双方的利益所要求的,这个要求使他们联合起来,在他们面前提出了消灭“寄生虫”、用一种新东西来代替它的共同任务。

究竟用什么东西来代替呢?










