%\chapter{} % 小章节名称
%\blfootnote{} % 底线批注,在所需段落前加入
%\pskip % 跳大行,段落间用
%\noindent % 书信体的开头
%\mbox{} % 插入个盒子,设定好的大小,用于隔开文本
%\leftskip= 文本左边腾出空格,用于文本排版
% \CJKunderdot{\textbf{}} % 文字下加点
% {\kaishu } % 更改字体为楷书

\mbox{}

\mbox{}

国家对社会革命的态度和社会革命对国家的态度问题,像一般革命问题一样,是第二国际(1889-1914年)最著名的理论家和政论家们很少注意的,但是,在机会主义逐渐滋长而使第二国际在1914年破产的过程中,最突出的一点就是:甚至当他们接触到这个问题的时候,他们还是{\kaishu 竭力回避}或者不加理会。

\pskip

总的看来,我们可以说,由于在无产阶级革命对国家的态度问题上采取了有利于机会主义、助长机会主义的{\kaishu 回避态度},结果就产生了{\kaishu 曲解}马克思主义、把马克思主义完全庸俗化的现象。

\pskip

为了简要地说明这个可悲的过程,我们就拿最著名的马克思主义理论家普列汉诺夫和考茨基来说。
