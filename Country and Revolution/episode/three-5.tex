\chapter{消灭寄生虫式的国家} % 小章节名称
%\blfootnote{} % 底线批注,在所需段落前加入
%\pskip % 跳大行,段落间用
%\noindent % 书信体的开头
%\mbox{} % 插入个盒子,设定好的大小,用于隔开文本
%\leftskip= 文本左边腾出空格,用于文本排版
% \CJKunderdot{\textbf{}} % 文字加粗下加点
% {\kaishu } % 更改字体为楷书


我们已经引用了马克思有关这个问题的言论,现在再引几段。

\pskip
\leftskip=10mm
\small

马克思写道:······“人们往往把新的历史创举当作是抄袭旧的、甚至已经过时的社会生活形式,只要新的机关稍微与这些形式有点相似。这就是新的历史创举的通常命运。于是摧毁(bricht——打碎)现代国家政权的新公社,也就被看作中世纪公社的复活······看作许多小国家的联盟(孟德斯鸠,吉伦特派)······看作从前反对过分集中的斗争的扩大形式”······

······“公社制度将迄今靠社会供养而又阻碍社会自由发展的寄生赘疣——‘国家’所吞食的一切力量归还给社会机体。仅仅这一点就会把法国的复兴向前推进了”······

······“公社制度会使农村生产者在精神上受各省主要城市的领导,保证他们能够得到城市工人做自身利益的天然代表者。公社的存在自然而然会带来地方自治,但这种地方自治已经不是用来对抗现在已经成为废物的国家政权的东西了”······

\normalsize
\leftskip=0mm
\pskip

消灭“国家政权”这个“寄生赘疣”,“铲除”它,“破坏”它;“国家政权现在已经成为废物”,——马克思在评价和分析公社的经验时,关于国家就是这样谈的。

所有这些都是在大约半世纪以前写的,现在必须把这些话发掘出来,使广大群众能够认识马克思主义的本来面目。马克思观察了他经历的最后一次大革命之后做出的结论,恰巧在新的无产阶级大革命时代到来的时候被人遗忘了。

\pskip
\leftskip=10mm
\small

······“人们对公社的各种估计以及公社所代表的利益证明,公社是一个高度灵活的政治形式,而一切旧有的政权形似在本质上都是压迫者的政权。公社的真正秘密就在于它实质上是{\kaishu 工人阶级的政府},是生产者阶级同占有者阶级斗争的结果,是终于发现的、可以使劳动者在经济上获得解放的政治形式”······

“如果没有最后这个条件,公社制度就没有实现的可能,而是一个骗局”······

\normalsize
\leftskip=0mm
\pskip

空想主义者从事于“发现”可以使社会进行社会主义改造的各种政治形式。无政府主义者避而不谈一般政治形式的问题。现代社会民主党内的机会主义者则认为,资产阶级议会制民主国家的政治形式是不可逾越的极限,他们对着这个“典范”磕头磕得头破血流,他们宣布\CJKunderdot{\textbf{摧毁}}这种政治形式的任何意图都是无政府主义。

马克思从社会主义和政治斗争的全部历史中得出结论:国家一定会消逝;国家消逝的过渡形式(从国家到非国家的过渡),将是“组织成为统治阶级的无产阶级”。但是,马克思并没有去{\kaishu 发现}这个未来的政治{\kaishu 形式}。他只是确切地考察和分析了法国历史,正确地得出了1851年事变的结论:问题在于{\kaishu 破坏}资产阶级的国家机器。

当无产阶级群众革命运动爆发的时候,尽管这个运动遭到挫折,尽管这个运动为期很短而且有显著的弱点,马克思还是来研究这个运动究竟{\kaishu 发现了}怎样的政治形式。

公社是无产阶级革命{\kaishu 打碎}资产阶级国家机器的第一次尝试,是“终于发现的”政治形式,这个政治形式可以而且应该用来{\kaishu 代替}已被打碎的国家机器。

我们往下就会看到,俄国1905年革命和1917年革命在另一个环境和另一种条件下继续着公社的事业,证实着马克思这种天才的历史分析。







