\clearpage
\pagestyle{empty}
\setcounter{page}{1}
\label{afterword}\pdfbookmark{重排后记}{afterword}
\begin{center}
	\zihao{3}\xbsong 重\hspace{0.333em}排\hspace{0.333em}后\hspace{0.333em}记
\end{center}
\vspace{1em}

\ykai

\id

\id 《国家与革命》是俄国思想家列宁创作的马克思主义哲学著作。它是1917年俄国革命第一阶段(二月革命)与第二阶段(十月革命)中间诞生的决定性著作。这本书从同时代的其他著作中脱颖而出,被认为是理解列宁对俄国革命未来的形势及使命之思考的一把钥匙。该书大量引用马克思主义经典作品的原文并针对它们给出了列宁自己的注解,以阐明列宁在国家问题上的立场。

\id 本电子书依1949年8月人民出版社版为原本,封面依据环境选择更替为较为清晰的1970年法文版封面。碍于本人\LaTeX 水平过低,全书仅进行了重置排版,没有对其进行如标签跳转等便捷处理,同时格式进行了一定程度的修改,但并不妨碍书籍的阅读体验。

\id 重排版全部源代码文件在网上公开,项目地址如下:
\begin{center}
	\href{https://github.com/NanFeng683/The-State-and-Revolution}{https://github.com/NanFeng683/The-State-and-Revolution}
\end{center}

\id 因个人精力、水平有限,亦难免遗漏或者改错之处。读者如有发现,可
前往上述网址讨论或提交相关信息。如有兴趣者,亦可在上述项目网址下载源码进行修订。

\id 感谢诸多网友对本人在重置过程中遇到问题时的悉心帮助。

\id 同时欢迎有同志意愿参与重置书籍的工作,让更多书籍在信息时代光芒不减。


\null{}

\hspace{7cm}\zihao{-4}\kaishu{南风易安}

\mbox{}

\hspace{6.5cm}\normalfont{} \zihao{-4}2022年7月\normalsize

\newpage
\label{tips}\pdfbookmark{特别提示}{tips}
\heiti \zihao{4}
\begin{center}
	~\\
	特\hspace{0.333em}别\hspace{0.333em}提\hspace{0.333em}示
\end{center}
\normalsize
\vspace{1em}

该书的版权、著作权由原作者、出版机构及该书权利人所有。如需商用,请与原作者、出版机构及该书权利人联系。

~

本重排版仅作个人学习之用。若有侵犯原书相关权利人权利,请联系本人删除网络发布。

\clearpage