\chapter{用什么东西来代替被打碎的国家机器呢?} % 小章节名称
%\blfootnote{} % 底线批注,在所需段落前加入
%\pskip % 跳大行,段落间用
%\noindent % 书信体的开头
%\mbox{} % 插入个盒子,设定好的大小,用于隔开文本
%\leftskip= 文本左边腾出空格,用于文本排版
% \CJKunderdot{} % 文字下加点
% {\kaishu } % 更改字体为楷书

1847年马克思在"共产党宣言"中对这个问题的回答还十分抽象,更正确些说,只是指出这个任务,而没有指出解决的方法。"无产阶级组织成为统治阶级","争得民主",这就是"共产党宣言"的回答。

无产阶级组织成为统治阶级究竟会采取怎样的具体形式,怎样才能组织得同最完备最彻底的"民主"相适应,关于这个问题,马克思并没有陷于空想,而是期待群众运动的{\kaishu 经验}来解答。

马克思在"法兰西内战"一书中会对公社的经验(尽管经验很少)作过极仔细的分析。现在我们把该书中最重要的几段摘录下来:

\pskip
\leftskip=10mm
\small

中世纪产生的"集中的国家政权及其遍布各地的机关——常备军、警察、官僚、僧侣、法官等级",在19世纪已经发展起来了。随着劳资阶级对抗的发展,"国家政权就愈来愈变成压迫劳动的社会权利,变成阶级统治的机器。在每次标志着阶级斗争的一定进步的革命后,国家政权的纯粹压迫性质,也就日益公开地显露出来"。在1848-1849年革命以后,国家政权就成为"全国资本对劳动作战的武器"。第二帝国使这种情况巩固起来。

"公社是同帝国绝对相反的东西"。"它是共和国的一种形式,这种共和国不仅应该消灭阶级统治的君主制形式,而且应该消灭阶级统治本身"······

\normalsize
\leftskip=0mm
\pskip

无产阶级社会主义共和国的"这种"形式是怎样的呢?公社开始建立的国家是怎样的呢?

\pskip
\leftskip=10mm
\small

······"公社的第一个法令就是废除常备军而用武装的人民来代替它"······

\normalsize
\leftskip=0mm
\pskip

现在一切自命为社会主义的政党的纲领中都有这个要求。但是它们的纲领究竟有什么价值,这从我国社会革命党人和孟什维克的行径中看得最清楚,实际上它们恰巧在2月27日革命以后就已经拒绝实现这个要求!

\pskip
\leftskip=10mm
\small

······"公社是由巴黎各区普选出的城市代表组成的。这些代表对选民负责,随时可以撤换。其中大多数自然是工人,或者是公认的工人阶级的代表"······

······"一向作为国家政府的工具的警察,立刻失去了一切政治职能,并变为公社中随时可以撤换的负责机关······\quad 其他各管理部门的官吏也是一样······\quad 从公社委员起,自上而下一切公职人员,都只是领取相当于{\kaishu 工人工资}的薪金。国家高级官吏所享有的一切特权及支付给他们的办公费,都随着这些官吏的消失而消失······\quad 公社在废除了常备军和警察等旧政府物质权力的武器以后,立刻着手摧毁精神压迫的工具僧侣······\quad 法官已经失去其表面的独立性······他们今后应该公开选出,对选民负责,并且随时可以撤换"······$^{15}$

\normalsize
\leftskip=0mm
\pskip

由此可见,公社用来代替被打碎的国家机器的,似乎"仅仅"是更完备的民主制:取消常备军,对公职人员实行全面的选举制和撤换制。但是这个"仅仅",事实上意味着完全用一些原则上不同的机关来代替另一些机关。在这里恰巧看到了一个由"量变到质变"的实际例子:民主制达到了一般想象的最充分最彻底的程度以后,就会由资产阶级的民主制变为无产阶级的民主制,就会由国家(=镇压一定阶级的特别力量)变为一种已经不是原来的国家的东西。

镇压资产阶级及其反抗,仍然是必要的。这对公社尤其必要,公社失败的原因之一就是在这方面做得不够坚决。但在这里实行镇压的机关是居民中的大多数,而不是居民中的少数,同过去奴隶制、农奴制、雇佣奴隶制时代的情形完全相反。既然人民大多数{\kaishu 亲自}镇压压迫自己的人,实行镇压的"特别力量"也就\CJKunderdot{\kaishu 不需要了}!国家就在这个意义上{\kaishu 开始消亡}。人民大多数可以代替那些享有特权的少数人(享有特权的官吏,常备军军官)的特殊机关来直接进行这些工作,而行使国家政权的职能愈是全民化,这个国家政权就愈不需要了。

在这方面特别值得注意的是马克思着重指出的公社所采取的措施:取消支付给官吏的一切办公费和一切金钱特权,把国家所有公职人员的薪金减到"工人工资"的水平。这里恰巧最明显地表现出一种{\kaishu 转变}:从资产阶级的民主制变为无产阶级的民主制,从压迫者的民主制转变为被压迫阶级的民主制,从国家这个镇压一定阶级的"特别力量"转变为由人民大多数——工人和农民用{\kaishu 共同的力量}来镇压压迫者。正是在这特别明显的一点上,也许是国家问题的最重要的一点上,人们把马克思的教训忘得干干净净!通俗的解释(这种解释多不胜数)是不提这一点的。把这一点看作已经过时的"幼稚行为","照例"不讲他,正如基督教徒在基督教成为国教以后,便"忘记了"原始基督教带有革命民族精神的"幼稚行为"一样。

\blfootnote{[5]爱德华·伯恩斯坦(1850年1月6日-1932年12月18日),德国社会民主主义理论家及政治家,德国社会民主党成员,进化社会主义(改良主义)的建立者之一。}
降低国家高级官吏的薪金,看来"不过"是幼稚的原始的民主主义的要求。最新的机会主义的"创始人"之一,以前的社会民主党人伯恩施坦$^{[5]}$会不止一次地玩弄过资产阶级嘲笑"原始的"民主主义的庸俗把戏。他同一切机会主义者及现在的考茨基主义者一样,完全不懂得:第一、如果不在某种程度上"回复"到"原始的"民主主义,从资本主义过渡到社会主义{\kaishu 是不可能的}(不这样做,怎么能够过渡到由大多数人民和全体人民行使国家职能呢?);第二、以资本主义和资本主义文化为基础的"原始民主主义"同原始时代或资本主义以前的原始民主主义是不一样的。资本主义文化{\kaishu 创立}了大生产、工厂、铁路、邮政、电话等等,在这个基础上,旧的"国家政权"的绝大多数职能就变得很简单,只要简单地办理一些登记、填表、检查等手续就行了,因此,每一个识字的人都完全能够行使这些职能,行使这些职能只须付给普通"工人的工资",并且可以(也应当)把这些职能中任何享受特权的"长官制"的残余铲除干净。

对一切公职人员毫无例外地实行全面选举制并可以{\kaishu 随时}撤换,把他们的薪金减低到普通"工人工资"的水平,所有这些简单的和"不言而喻"的民主措施必然会把工人和大多数农民的利益结合起来,同时也必然会成为从资本主义过渡到社会主义的桥梁,这些措施关系到国家的纯政治的社会改造,但是这些措施只有同正在实行或准备实行的"剥夺剥夺者"的措施联系起来,也就是同变生产资料资本主义私有制为社会公有制的措施联系起来,才会显示出全部意义。

\pskip
\leftskip=10mm
\small

马克思写道:"公社实现了所有资产阶级革命都提出的'廉价政府'的口号,因为它取消了两项最大的开支,即军队和官吏。"

\normalsize
\leftskip=0mm
\pskip

农民同小资产阶级其他阶层一样,他们当中只有极少数人能够"上升",能够"出头"(从资产阶级的意义来说),即变成富人,变成资产者,变成生活上有保障和享有特权的官吏。在任何一个有农民的资本主义国家(这样的资本主义国家占大多数),绝大多数农民是受政府压迫的,是渴望推翻这个政府和渴望"廉价"政府的。能够实现这一要求的{\kaishu 只有}无产阶级,而无产阶级实现了这一要求,也就是向国家的社会主义改造迈进了一步。


