\chapter[初版序言]{初~版~序~言} % 小章节名称
%\blfootnote{} % 底线批注
%\pskip % 跳大行,段落间用
%\noindent % 书信体的开头
%\mbox{} % 插入个盒子,设定好的大小,用于隔开文本
%\leftskip= 文本左边腾出空格,用于文本排版


国家问题,现在无论在理论方面或在政治实践方面,都具有特别重大的意义。帝国主义战争大大加速和加剧了垄断资本主义变为国家垄断资本主义的过程。国家同拥有无限权力的资本家集团日益密切地溶和在一起,它对劳动群众的残酷压迫,愈来愈骇人听闻了。各先进国家(我们在这里是指这些国家的“后方”而言)已经变成了囚禁工人的军事苦工监狱。

连绵不断的战争造成的空前惨剧和灾难,使群众生活困苦不堪,使他们更加义愤填膺。国际无产阶级革命正在显著地发展,这个革命对国家所抱的态度,已经成为具有实际意义的问题了。

在几十年较为和平的发展中积聚起来的机会主义成分,使得社会沙文主义流派在世界各个正式社会主义政党内取得了统治地位。这个流派(在俄国有普列汉诺夫、波特列索夫、布列什柯夫斯卡娅、鲁巴诺维奇以及不太露骨的策列铁里先生、切尔诺夫先生之流;在德国有谢德曼、列金、大卫等;在法国和比利时有列诺得尔、盖德、王德威尔得;在英国有海德门和费边社分子$^{2}$等等)口头上是社会主义,实际上是沙文主义,其特点就在于这些“社会主义领袖”不仅对于“自己”民族资产阶级的利益,而且正是对于“自己”国家的利益,采取卑躬屈膝的迎合态度。因为大多数所谓大国早就在剥削和奴役很多弱小民族,帝国主义战争也正是为了瓜分和重分这些赃物而进行的战争。如果不在“国家”问题上反对机会主义偏见,就不能展开斗争,不能使劳动群众摆脱资产阶级的影响,特别是摆脱帝国主义资产阶级的影响。

首先,我们来考察一下马克思和恩格斯的国家学说,特别详细地谈谈这个学说被人遗忘或者遭到机会主义者歪曲地各个方面。其次,我们要专门分析一下歪曲这个学说地主要代表人物,即在这次战争中可耻地遭到彻底破产的第二国际(1889-1914年)最著名的领袖考茨基。最后,我们要给俄国1905年革命、特别是1917年革命的经验,做一个基本的总结。后面这次革命的第一个阶段现在(1917年8月初)大概正在结束,但整个这次革命只能认为使帝国主义战争引起的社会主义无产阶级革命的链条中的一个环节。无产阶级社会主义革命对国家的态度问题向群众说明,为了使自己从资本的枷锁下解放出来,他们在最近的将来应该做些什么。因此这个问题不仅具有实际的政治意义,而且具有最迫切的意义。

\mbox{} 

\leftskip=80mm 作者

\leftskip=73mm 1917年8月

\leftskip=0mm