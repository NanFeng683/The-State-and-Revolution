\chapter{1891年为马克思的“法兰西内战”所写的序言} % 小章节名称
%\blfootnote{} % 底线批注,在所需段落前加入
%\pskip % 跳大行,段落间用
%\noindent % 书信体的开头
%\mbox{} % 插入个盒子,设定好的大小,用于隔开文本
%\leftskip= 文本左边腾出空格,用于文本排版
% \CJKunderdot{\textbf{}} % 文字下加点
% {\kaishu } % 更改字体为楷书

恩格斯在为“法兰西内战”第3版写的序言中(写于1891年3月18日,最初刊载在“新时代”杂志上),除了顺便就有关对国家的态度问题提出了许多值得注意的意见以外,还对巴黎公社的教训作了一个精辟的总结$^{20}$。这个总结把公社以来二十年的全部经验也包括进去了,而且是专门用来反对流行于德国的“国家迷信”观念的,因而可称为马克思注意在国家问题上的{\kaishu 最高成就}。

\pskip
\small
\leftskip=10mm

恩格斯指出,法国每次革命以后工人总是武装起来了;“因此,掌握国家大权的资产者的第一信条就是解除工人的武装。于是,在每次工人进行革命以后就产生新的斗争,其结果总是工人失败”······

\leftskip=0mm
\normalsize
\pskip

对各次资产阶级革命的经验作出的这个总结,真是又简短,又明了。这里正好抓住了问题的实质,也是国家问题的实质({\CJKunderdot{\kaishu 被压迫阶级有没有武装}?)。这一点正是那些受资产阶级思想影响的教授以及小资产阶级民主派尝尝避而不谈的。在1917年俄国革命的时候,这个“孟什维克”“也是马克思主义者”的策烈铁里很荣幸(卡芬雅克式的荣幸)有机会来泄露资产粘结剂革命的这个秘密。他在6月11日的“具有历史意义的”演说中,脱口说出了资产阶级要解除彼得堡工人武装的决定,当然,他把这个决定说成是他自己的决定,并且说这一般是“国家”所需要的。
	
策烈铁里在6月11日发表的具有历史意义的演说,当然会变成每一个论述1917年革命的历史学家啊都要援引的一个最明显的例子,它证明策烈铁里先生所率领的社会革命党人同孟什维克的联盟如何转到资产阶级方面来{\kaishu 反对}革命的无产阶级。

恩格斯顺便提出的另外一个有关国家问题的意见是关于宗教的。大家知道,德国社会民主党因为腐化而愈益机会主义化,因而对“宣布宗教为私人的事情”这个有名的公式愈来愈加以庸俗的曲解,他们认为宗教问题{\kaishu 对于}革命无产阶级{\kaishu 政党}也是私人的事情!!这种完全背叛无产阶级革命纲领的立场正是恩格斯当时出来反对的,但恩格斯在1891年还只看到自己党内机会主义的{\kaishu 最小的}萌芽,因此他说的很谨慎:

\pskip
\small
\leftskip=10mm

“参加公社的差不多都是工人或公认的工人代表,所以它的决议显然是纯粹的无产阶级性质的。有的决议把共和派资产阶级由于卑鄙的懦怯而拒绝的、然而是工人阶级自由活动所必须的那些改革法定下来,例如实行宗教\CJKunderdot{\kaishu 对国家来说}仅仅是私人的事情的原则。有些决议则直接与工人阶级的利益有关,并且在一定程度上深深刺入了旧社会制度的内脏”······

\leftskip=0mm
\normalsize
\pskip

恩格斯故意把“对国家来说”这几个字加上着重号,目的是要击中德国机会主义的要害,因为德国机会主义宣布宗教对{\kaishu 对党来说}是私人的事情,这样也就把革命无产阶级政党降低到最庸俗的“自由思想的”市侩水平,市侩承认可以不信宗教,但是拒绝执行党对麻醉人民的宗教鸦片进行斗争的任务。

将来研究德国社会民主党党史的历史学家在该党1914年遭到可耻的破产的根源时,会找到关于这个问题的许多有趣的材料:从该党思想领袖考茨基的论文中由机会主义打开大门的暧昧言论起,直到党对1913年的《Los-von-Kirche-Bewegung》(与教会分离的运动)的态度止。

现在我们来看一看恩格斯在公社以后二十年是怎样为斗争的无产阶级总结公社教训的。

下面就是恩格斯认为最重要的教训:

\pskip
\small
\leftskip=10mm

······“正是军队、政治警察、官僚等旧的集权政府的压迫权力,这种由拿破仑在1798年建立,以后一直被每届新政府当作有用的工具接受并利用来反对自己的敌人的权力,应该在全国各地复没,正如它在巴黎复没一样。

公社一开始就应当承认,获得统治权的工人阶级不能继续利用旧的国家机器来进行管理;工人阶级为了不致失去刚刚争得的统治权,它一方面应当铲除全部旧的、一直被利用来反对它的压迫机器,另一方面应当保证自己有反对自己的代表和官吏的权利,宣布他们每个人都毫无例外地可以随时撤换”······

\leftskip=0mm
\normalsize
\pskip

恩格斯一再着重指出,不仅在君主国,而且在{\kaishu 民主共和国},国家依然是国家,也就是说仍然保留着它的基本特征:把公职人员,“社会公仆”,社会机关,变为社会的{\kaishu 主人}。

\pskip
\small
\leftskip=10mm

······“为了不让国家和国家机关由社会公仆变为社会的主人(这种现象在至今所有的国家中都是不可避免的),公社采取了两个正确的方法:第一、把行政、司法和国民教育方面的一切职位交给由普选选出的人担任,同时用法律规定选举者随时可以撤换被选举者。第二、不分职位高低,所有公职人员的工资同其他工人相等。公社一般给的最高薪金为6000法郎$^{*}$。这样就可靠地防止了人们去追求升官发财,何况公社还规定各代表机关的代表必须绝对服从选民的委托”······

\leftskip=0mm
\normalsize
\pskip

\blfootnote{* 名义上约等于2400卢布,但按现在的汇率计算,约等于6000卢布。有些布尔什维克提议例如在市杜马内给9000卢布的现金,而不提议全国以6000卢布(这个数目是足够的)为最高薪金,这种做法是完全不可饶恕的。}
恩格斯在这里谈到了一个有趣的界线,一达到这个界线,彻底的民主制就变为社会主义,同时它也就\CJKunderdot{\textbf{要求}}实行社会主义。因为要消灭国家,就必须把国家服务机关的职能变为非常简单的监督和统计手续,使绝大多数居民以至全体居民都能够办理,都能够胜任。要完全消除升官发财的思想,就必须使国家服务机关中那些收入不多但是“光荣的”位置 \CJKunderdot{\kaishu 不}能成为在银行和股份公司内找到肥缺的桥梁,像在一切最自由的资本主义国家内{\kaishu 经常}看到的那样。

但是,恩格斯没有犯有些马克思主义者在某些问题上,例如在民族自决权问题上所犯的那种错误:他们说民族自决权在资本主义下是不可能实现的,而在社会主义下则是多余的,这种似乎很明智但实际上并不正确的论调,对于{\kaishu 任何一种}民主设施,连给官吏发微薄的薪金的办法也包括在内,都可以这样说,因为在资本主义下彻底的民主制是不可能实现的,而在社会主义下则任何民主制都是会{\kaishu 消亡}的。

这是一种诡辩,正像古时候有句笑话:一个人掉了一根头发,他是否就成了秃头呢?

{\kaishu 彻底}发展民主制,找出发展的{\kaishu 形式},用{\kaishu 实践}来检验这些形式等等,都是为社会革命进行斗争的任务之一。任何单独存在的民主制都不会产生社会主义,但在实际生活中民主制永远不会是“单独存在”,而总是“相互依存”的,它影响经济,推动{\kaishu 经济的}改造,受经济发展的影响等等。这是活生生的历史的辩证法。

恩格斯继续写道:

\pskip
\small
\leftskip=10mm

······“这种炸毁(Sprengung)旧的国家政权并用新的真正民主的国家政权来代替的情形,已经在‘法兰西内战’第三章中作了详细的描述。但是在这里还有必要再来简单地谈一谈这种代替地几个特点,因为正是在德国,对国家地迷信已经从哲学方面转到资产阶级甚至很多工人地一般意识中去了。按照哲学家地学说,国家是‘观念的实现’,或译成哲学语言,就是上帝在人间的统治,也就是永恒真理和正义所由实现或应当实现的场所。由此就产生了崇拜国家以及一切有关国家的事物的心理,这种心理之所以容易生根,是因为人们从小就一直认为全社会的公共事业和公共利益只能用旧的方法来处理和保护,即通过国家及其收入极多的官吏来处理和保护。人们以为,不再迷信世袭君主制而主张民主共和制,那就已经是非常勇敢地前进了一步。实际上,国家无非是一个阶级镇压另一个阶级的机器;这是民主共和制下也丝毫不比在君主制下差。国家最多也不过是无产阶级在争取阶级统治的斗争胜利以后所承受下来的一个祸害;胜利了的无产阶级也将同巴黎公社一样,不得不立即除去这个祸害坏的一面,直到在新的自由的社会条件下成长起来的一代能够把国家制度的这一堆垃圾完全抛掉为止。”

\leftskip=0mm
\normalsize
\pskip

恩格斯曾经警告过德国人,叫他们在共和制代替君主制的时候不要忘记一般国家问题的社会主义原则。他的警告现在看起来好像是直接对策烈铁里和切尔诺夫之流先生们的教训。因为他们在“联合政府”的实践中正好表现出对国家迷信和崇拜。

还应当指出两点:(1)恩格斯说,在民主共和制下,国家之为“一个阶级压迫另一个阶级的机器”,“丝毫不”比在君主制下“差”,但这决不等于说,压迫的{\kaishu 形式}对于无产阶级是无所谓的,像某些无政府主义者所“教导”的那样。更广泛、更自由、更公开的阶级斗争{\kaishu 形式}和阶级压迫{\kaishu 形式},能够大大地促进无产阶级为消灭一切阶级而进行的斗争。

(2)为什么只有新的一代才能够完全抛掉国家制度这一堆垃圾呢?这个问题是同民主制的消除问题联系着的,现在我们就来谈谈这个问题。


