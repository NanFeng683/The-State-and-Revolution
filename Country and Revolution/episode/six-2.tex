\chapter{考茨基与机会主义者的论战} % 小章节名称
%\blfootnote{} % 底线批注,在所需段落前加入
%\pskip % 跳大行,段落间用
%\noindent % 书信体的开头
%\mbox{} % 插入个盒子,设定好的大小,用于隔开文本
%\leftskip= 文本左边腾出空格,用于文本排版
% \CJKunderdot{\textbf{}} % 文字下加点
% {\kaishu } % 更改为楷书字体

在俄国文坛上,考茨基的著作的译本无疑比其他国家多得无可计量。难怪有些德国社会民主党人开玩笑说,在俄国读考茨基著作的人比在德国还多(附带说一说,在这个玩笑里含有比开这个玩笑的人所料想的更深刻得多的历史内容:俄国工人在1905年对世界最优秀的社会民主主义文献中最优秀的著作有空前强烈的要求,他们得到的这些作品的译本也比其他各国多得无可比拟,这样就把比较先进的邻国的丰富经验加速地移植到我国无产阶级运动的所谓年轻的基地上来了)。

\blfootnote{[1]亚历山大·米勒兰(Alexandre Millerand,1859年2月10日-1943年4月7日),法国律师和政治家。1920年出任法国总理,后来以温和派联盟领袖的资格当选共和国总统(1920-1924),并试图通过修改宪法加强总统的权利,后迫于左翼联盟的压力而辞职。}
\blfootnote{[2]让·饶勒斯(Jean Jaurès,1859—1914年),法国和国际社会主义运动的著名活动家,法国社会党的领导人之一,历史学家和哲学家。}
考茨基之所以在俄国特别出名,除了他对马克思主义作了通俗的解释以外,就是他同机会主义者及其首领伯恩施坦进行了论战。但是有一个事实几乎是没有人知道的。如果有人想考察一下考茨基在1914-1915年危机尖锐时期怎样可耻地表现出张皇失措,堕落到替社会沙文主义辩护的地步,那就不能放过他。这个事实就是:考茨基在起来反对法国最著名的社会主义代表(米勒兰$^{[1]}$和饶勒斯$^{[2]}$)和德国最著名的机会主义代表(伯恩施坦)之前,就已经表现出很大的动摇。1901-1902年在斯图加特出版的、旨在捍卫无产阶级观点的、马克思主义的“曙光”杂志$^{24}$,曾不得不同考茨基进行{\kaishu 论战},并把他在1900年巴黎国际社会主义者代表大会$^{25}$上提出的决议叫做“橡皮”决议,因为这个决议对机会主义者的态度是暧昧的,调和的,躲躲闪闪的。在德国的书刊中还刊载过考茨基的信件,这些信件也表明他在攻击伯恩施坦之前有过很大的动摇。

但是另一件事情的意义更重大得多,这就是:当我们研究考茨基最近背叛马克思主义的{\kaishu 经过}的时候,从他同机会主义者的论战中,从他提问题和解释问题的方法上,我们可以看到,他在国家问题上恰恰是一贯倾向于机会主义的。

我们拿考茨基反对机会主义的第一部大作“伯恩施坦与社会民主党的纲领”来说。考茨基详细地驳斥了伯恩施坦。但是下面的情况值得注意。

伯恩施坦在他的遗臭万年的“社会主义的先决条件”一书中,指责马克思主义为“{\kaishu 布朗基主义}”(此后,俄国机会主义者和自由派资产者千百次重复这种指责来攻击革命马克思主义的代表布尔什维克)。而且伯恩施坦还特别谈到马克思的“法兰西内战”,企图(我们看到,这种企图已经彻底失败)把马克思对公社的教训的观点与蒲鲁东的观点混为一谈。伯恩施坦特别注意马克思与1872年在“共产党宣言”的序言中着重指出的结论,这个结论说:“工人阶级不能简单地掌握现成的国家机器并运用它来达到自己的目的。”

伯恩施坦非常“喜爱”这句格言,所以他在自己那本书里至少重复了三遍,并且把它完全曲解成机会主义的见解。

我们已经讲过,马克思是想说工人阶级应当{\kaishu 打碎、摧毁、炸毁}(Sprengung——炸毁,是恩格斯所用的字眼)全部国家机器。但在伯恩施坦看来,似乎马克思说这句话是警告工人阶级{\kaishu 不要}在夺取政权时采用过激的革命手段。

不能想象还有比这种曲解马克思思想的行为更粗暴更丑恶的了。

而考茨基是怎样最详尽地驳斥伯恩施坦主义地呢?

他避免分析机会主义在这一点上曲解马克思主义的根源。他引证了我们在前面引证过的恩格斯为马克思的“法兰西内战”所写的序言中的一段话,说什么根据马克思的意见,工人阶级不能{\kaishu 简单地}掌握{\kaishu 现成的}国家机器,但一般来说它是{\kaishu 能够}掌握这个机器的,如此而已。至于伯恩施坦把\CJKunderdot{\kaishu 完全}同马克思的真正思想\CJKunderdot{\kaishu 相反的}话妄加在马克思的身上,以及马克思从1852年就提出无产阶级革命负有“打碎”国家机器的任务,考茨基却一字不提。

结果是:马克思主义同机会主义在无产阶级革命的任务问题上的最重要差别被考茨基抹杀了!

\pskip
\leftskip=10mm
\small

考茨基在“{\kaishu 反对}”伯恩施坦时写道:“关于无产阶级专政问题,我们可以心平气和地留待将来去解决。”(德文版172页)

\normalsize
\leftskip=0mm
\pskip

这不是{\kaishu 反对}伯恩施坦,同他进行论战,实际上是向他{\kaishu 让步},是把阵地让给机会主义,因为机会主义者现在所需要地,恰恰是把关于无产阶级革命任务的一切根本问题都“心平气和地留待将来去解决”。

马克思和恩格斯在1852年到1891年这四十年当中,教导无产阶级应当打碎国家机器。而考茨基在1899年,即当机会主义者在这一点上完全背叛马克思主义的时候,却用打碎国家机器的具体形式问题来{\kaishu 代替}要不要打碎这个机器的问题,把我们无法预先知道具体形式这种“无可争辩的”(也是争不出结果的)庸俗道理当做护身符!!

在马克思和考茨基之间,在他们对无产阶级政党发动工人阶级进行革命的任务所持的态度上,存在着一条不可逾越的鸿沟。

我们且拿考茨基的另一部更成熟的、多半也是为了驳斥机会主义的错误而写的著作来说。那就是他论“社会革命”的小册子。作者在这里把“无产阶级革命”问题和“无产阶级制度”问题作为专题来研究。作者发表了许多极宝贵的见解,但是恰恰没有{\kaishu 谈到}国家问题。在这本小册子里,到处谈的只是夺取国家政权,也就是说,考茨基的说法都是向机会主义者让步的,他认为{\kaishu 不}破坏国家机器也{\kaishu 能}夺取政权。马克思在187年认为“共产党宣言”的纲领上已经“陈旧的”东西,考茨基却在1902年把它{\kaishu 恢复}了。

在这本小册子里,有一节专门谈“社会革命的形式与武器”问题。其中谈到群众性的政治罢工和国内战争,也谈到“现代大国的强力武器即官僚和军队”,但是一个字也没有提到公社给了工人一些什么教训。可见,恩格斯警告工人特别是德国社会主义者不要“崇拜”国家不是没有原因的。

考茨基把问题说成这样:胜利了的无产阶级“将实现民主纲领”,接着就叙述了这个纲领的条文。至于1871年在以无产阶级民主制代替资产阶级民主制的问题上所提出的一些新东西,他却一个字也没有提到。考茨基用下面这种听起来好像“冠冕堂皇”的老一套的话来搪塞:

\pskip
\leftskip=10mm
\small

“不言而喻,在现行制度下我们是不能取得统治地位的。革命本身预定要有一个长期的深刻的斗争,在这个斗争一定会改变我们目前的政治结构和社会结构。”

\normalsize
\leftskip=0mm
\pskip

毫无疑义,这是“不言而喻”的,正如马吃燕麦,伏尔加河流入里海的真理一样。所可惜的是他拿“深刻的”斗争一类空洞而浮夸的话来{\kaishu 回避}革命无产阶级的迫切问题:{\kaishu 无产阶级}革命对国家、对民主制的态度与以往非无产阶级革命不同的“深刻的地方”{\kaishu 究竟在哪里}。

考茨基回避这个问题,{\kaishu 实际上}就是在这个最重要的问题上向机会主义让步,但他在{\kaishu 口头上}却气势汹汹地向它宣战,强调“革命思想”的意义(如果怕向工人宣传革命的具体教训,那么试问这种“思想”还有多大价值呢?),或者说“革命的唯心主义高于一切”,或者宣称英国工人现在“几乎与小资产者不相上下”。

\pskip
\leftskip=10mm
\small

考茨基写道:“在社会主义社会里同时并存的可以有······各种形式上极不同的企业:官僚的(??)、工联的、合作社的、个人经营的”······“例如,有些企业并非有官僚(??)组织不可,铁路就是这样。在这里,民主组织可以采取这样的形式:工人选出代表来组成某种类似议会的东西,由这个议会制定工作条例并监督官僚机关的行政。有些企业可以交给工会管理,另外一些企业则可以按合作原则组织起来。”(见1903年日内瓦版俄译文第148页和第115页)

\normalsize
\leftskip=0mm
\pskip

这种论调是错误的,它比马克思和恩格斯在70年代用公社的教训做例子来说明的倒退了一步。

从所谓必须有“官僚”组织这一点看来,铁路同大机器工业的一切企业,同任何工厂、大商店和大资本主义农场根本没有区别。在所有这些企业中,技术条件要求每个人绝对严格地遵守纪律,要求每个人十分准确地执行他所担负地一部分工作,不然就会有整个企业陷于停顿或机器和产品损坏的危险。在所有这些企业中,工人当然要“选出代表来组成{\kaishu 某种类似议会的东西}”。

但是整个关键就在于这个“某种类似议会的东西”\CJKunderdot{\kaishu 不}会是资产阶级议会机关式的议会。整个关键就在于,这个“某种类似议会的东西”\CJKunderdot{\kaishu 不}会仅仅“制定条例和监督官僚机关的行政”,像思想没有超出资产阶级范围的考茨基所想象的那样。在社会主义社会里。由工人代表组成的“某种类似议会的东西”当然会“制定条例和监督”“机关的”“行政”,\CJKunderdot{\kaishu 可是}这个机关却\CJKunderdot{\kaishu 不}会是“官僚的”机关。工人在夺得政权之后,就会把旧的官僚机关打碎,把它彻底摧毁,完全粉碎,而用仍然由这些工人和职员组成的新机关来代替它;为了\CJKunderdot{\kaishu 防止}这些人变成官僚,就会立即采取马克思和恩格斯详细分析过的方法:(1)不但实行选举制度,而且随时可以撤换,(2)薪金不得高于工人的工资,(3)立刻转到使{\kaishu 所有的人}都来执行监督和监察的职能,使{\kaishu 所有的人}暂时都变成“官僚”,因而使\CJKunderdot{\kaishu 任何人}都不能成为“官僚”。

考茨基完全没有考虑马克思的话:“公社不是议会式的,而是同时兼管立法和行政的工作机关。”

考茨基完全不了解资产阶级议会制与无产阶级民主制的区别,资产阶级议会制是把民主制(\CJKunderdot{\kaishu 不是供人民享受的})同官僚制(\CJKunderdot{\kaishu 反人民的})连在一起,而无产阶级民主制则立即采取方法来根除官僚制,并且能够把这种方法实行到底,直到官僚制完全消灭,供人民享受的民主制完全实现。

考茨基在这里又暴露了他对国家的“崇拜”和对官僚主义的“迷信”。

现在我们来研究考茨基最后的也是最好的一部反对机会主义者的著作,即他的“取得政权之路”的小册子(好像没有俄文版本,因为它是在1909年俄国反动势力猖獗的时候出版的)。这本小册子使一个很大的进步,因为它不像1899年批评伯恩施坦的小册子那样只谈一般的革命纲领,也不像1902年写的小册子“社会革命”那样不管社会革命到来的时期而泛论社会主义革命的任务,它谈的是使我们不得不承认“革命时代”{\kaishu 已经到来}的具体条件。

作者肯定地指出一般阶级矛盾的尖锐化和这方面起特别巨大作用的帝国主义。在西欧“1789-1871年的革命时期”之后,东方从1905年起也开始了同样的时期。世界大战以惊人的速度日益逼近。“无产阶级已经不能再说革命为时过早了”。“我们已经进入革命时期”。“革命时代已经开始。”

这些话说得非常清楚的。考茨基的这本小册子应当成为衡量德国社会民主党(包括考茨基本人)的言行的一个尺度:他们在帝国主义大战前夜{\kaishu 答应要做}的是什么,而在战争爆发的时候却堕落到怎样下流的地步。考茨基在这本小册子里写道:“目前的形势会引起这样一种危险,使人们很容易把我们(德国社会民主党)看得比实际上温和。”事实上,德国社会民主党原来就比人们想象的要温和得多,更机会主义得多!

更值得注意的是,考茨基虽然如此肯定地说革命时代已经到来,但是他在那本自称为专门分析“{\kaishu 政治}革命”问题地小册子里,却又完全避开了国家问题。

所有这些回避、保持缄默、躲躲闪闪的做法结合起来,就必然使他滚到机会主义那边去,关于这一点我们现在来谈一谈。

德国社会民主党好像以考茨基为代笔声明说:我仍然坚持革命观点(1899年);我特别承认无产阶级的社会革命是不可避免的(1902年);我承认革命的新时代已经到来(1909年);但是,既然问题是无产阶级革命对于国家的任务,那么我还是要反对马克思在1852年说过的话而倒退(1912年)。

在考茨基与潘涅库克的论战中,正是这样直截了当地提出问题的。











