\chapter{普列汉诺夫与无政府主义者的论战} % 小章节名称
%\blfootnote{} % 底线批注,在所需段落前加入
%\pskip % 跳大行,段落间用
%\noindent % 书信体的开头
%\mbox{} % 插入个盒子,设定好的大小,用于隔开文本
%\leftskip= 文本左边腾出空格,用于文本排版
% \CJKunderdot{\textbf{}} % 文字下加点
% {\kaishu } % 楷书文体

普列汉诺夫写了一本专门论述无政府主义与社会主义的关系问题的小册子,这本小册子的名字是“无政府主义与社会主义”,1894年用德文出版。

普列汉诺夫很巧妙地论述这个题目,对反对无政府主义的斗争中最现实、最迫切、政治上最重要的问题,即革命对国家的态度和一般关于国家的问题完全避而不谈!他的这本小册子可以分为两部分:一部分是历史文献,其中有关于施蒂纳和蒲鲁东等人思想演变的宝贵材料;另一部分是庸俗的,其中有关于无政府主义者与强盗没有区别这样拙劣的议论。

这两部分的结合,不但十分可笑,而且足以说明普列汉诺夫在俄国革命前夜以及革命时期的全部活动:在1905-1917年,普列汉诺夫正是这样表明自己是在政治上充当着资产阶级尾巴的半学究,半庸人。

我们看到,马克思和恩格斯在同无政府主义者论战时,极其详尽地说明了自己是怎样看待革命对国家的态度问题的。恩格斯在1891年出版马克思的“哥达纲领批判”时写道:“我们(即恩格斯和马克思)那时正在同以巴枯宁为首的无政府主义者作最猛烈的斗争,——那时离第一国际的海牙会议$^{23}$闭幕才两年。”

无政府主义者正是企图把巴黎公社宣布为他们“自己的”,认为它能证实他们的学说,然而他们根本不懂得公社的教训和马克思对这些教训的分析。对于是否需要{\kaishu 打碎}旧的国家机器、{\kaishu 用什么东西}来代替它等具体政治问题,无政府主义者连一个比较接近真理的答案都没有提出过。

但是在谈“无政府主义和社会主义”时回避整个国家问题,{\kaishu 不理会}马克思主义在公社以前和以后的全部发展,那就必然会滚到机会主义那边去。因为机会主义者求之不得的;正是完全{\kaishu 不}提我们刚才所指出的两个问题。光是这一点,{\kaishu 已经}是机会主义的胜利了。
