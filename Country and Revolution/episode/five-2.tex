\chapter{从资本主义向共产主义的过度} % 小章节名称
%\blfootnote{} % 底线批注,在所需段落前加入
%\pskip % 跳大行,段落间用
%\noindent % 书信体的开头
%\mbox{} % 插入个盒子,设定好的大小,用于隔开文本
%\leftskip= 文本左边腾出空格,用于文本排版
% \CJKunderdot{\textbf{}} % 文字下加点
% {\kaishu } % 更改为楷书字体

\pskip
\leftskip=10mm
\small

马克思继续写道,······“在资本主义社会和共产主义社会之间有一个前者变为后者的革命转变时期。与这个时期相适应的是一个政治上的过渡时期,这个时期的国家只能是{\kaishu 无产阶级的革命专政}”······

\normalsize
\leftskip=0mm
\pskip

这就是马克思根据他对无产阶级在现在资本主义社会中的作用的分析,根据这个社会的发展情况以及无产阶级与资产阶级利益对立并且不可调和的实际材料所得出的结论。

从前,问题的提法是这样的:无产阶级为了求得自身的解放,应当推翻资产阶级,夺取政权,建立自己的革命专政。

现在,问题的提法已有些不同了:从向着共产主义发展的资本主义社会过渡到共产主义社会,非经过一个“政治上的过渡时期”不可,而这个时期的国家只能是无产阶级的革命专政。

这个专政和民主的关系又是怎样的呢?

我们知道,“共产党宣言”就是把“无产阶级变为统治阶级”和“争得民主”这两个概念放在一起的。根据上述一切,可以更准确地断定,民主制在从资本主义到共产主义的过渡时期是怎样变更的。

在资本主义社会里,在它最顺利的发展条件下,比较完备的民主制就是民主共和制。但是这种民主制始终只限于资本主义剥削的狭小范围,因此它实质上始终只是供少数人、供有产阶级、供富人享受的民主制。资本主义社会的自由始终与古希腊共和国只供奴隶主享受的自由大致相同。由于资本主义剥削的条件,现代的雇佣奴隶被贫困压得“无暇过问民主”,“无暇过问政治”,以致大多数居民在通常的和平局面下被排斥在社会政治生活之外。

德国可以说是证实这一论断的最明显的例子,因为在这个国家里,有一段很长的时间,几乎有半世纪之久(1871-1914年),宪法一直承认结社是合法的,在这个时期,德国社会民主党在“利用合法机会”方面做的工作要比其他各国的社会民主党多得多,特别是使工人参加党的比例达到了举世未有的高度。

那么这种有政治觉悟的积极的雇佣奴隶在资本主义社会里所占的最大百分比究竟是多少呢?1500万雇佣工人中只有100万是社会民主党党员!1500万雇佣工人中只有300万是工会会员!

供极少数人享受的民主,供富人享受的民主,——这就是资本主义社会的民主制。如果仔细地考察一下资本主义民主制的结构,那么无论在选举权的“细微的”(似乎是细微的)条纹上(居住年限、妇女被排斥等等),或是在代理机关的办事手续上,在行使集会权的实际障碍上(公共的集会场所不准“旁人”使用!),在纯粹按资本主义原则办报等等事实上,到处都可以看到民主制的重重限制。对穷人的这种种限制、禁止、排斥、阻碍看起来似乎是很细微的,特别是在那些从来没有亲身体验过贫穷困苦、从来没有接近过被压迫阶级群众的生活的人(这种人在资产阶级的政论家和政治家中,如果不占百分之九十九,也得占十分之九)看来是很细微的,但是这些限制加在一起,却把穷人推出和排斥在政治生活之外,使他们不能积极参加民主生活。

马克思出色地暴露了资本主义民主制地这一\CJKunderdot{\kaishu  实质},他在分析公社的经验时说:这就是容许被压迫者数年一次来决定压迫阶级的哪个代表应当在议会里代表和压迫他们!

但是这种必然是狭隘的、暗中排斥穷人的、因而也是完全虚伪和骗人的资本主义民主制,决不像一般自由主义的教授和小资产阶级的机会主义者所想象的那样是简单地、直接地、平稳地朝向“日益彻底的民主制”发展的。不是的。向前发展,即向共产主义发展,必须经过无产阶级专政,决不能走别的道路,因为再没有其他人也没有其他道路能够{\kaishu 粉碎}剥削者资本家的{\kaishu 反抗}。

而无产阶级专政,即被压迫者先锋队组织成为统治阶级来镇压压迫者,不能仅仅限于扩大民主制。{\kaishu 除了}大规模地扩大这种\CJKunderdot{\kaishu 第一次}供贫民享受、供人民享受而不是供富人享受的民主制{\kaishu 之外},无产阶级专政还要对压迫者、剥削者、资本家采取一系列剥夺自由的措施。为了使人类从雇佣奴隶制下面解放出来,我们必须镇压这些人,必须用强力粉碎他们的反抗,——显然,凡是实行镇压和使用强力的地方,也就没有自由,没有民主。

读者还记得,恩格斯在给倍倍尔的信中已经很好地阐明了这一点,他说:“无产阶级需要国家不是为了自由,而是为了镇压自己的敌人,到了有可能来谈自由的时候,国家就不存在了。”

绝大多数人享受民主,对那些剥削和压迫人民的分子实行强力镇压,即不允许他们参加民主生活,——这就是从资本主义到共产主义的{\kaishu 过渡}时期的民主制。

只有在共产主义社会中,只有当资本家的反抗已经彻底粉碎,资本家已经消灭,阶级已经不存在的时候,即社会各个成员在对社会生产资料的关系上已经没有什么差别的时候,——{\kaishu 只有}在那个时候,“{\heiti 国家才会消亡,才谈得上自由}”。只有在那个时候,真正完备的、真正没有任何例外的民主制才有可能实现。也只有在那个时候,民主制才开始{\kaishu 消亡};道理很简单,因为人们既然摆脱了资本主义的奴役制,摆脱了资本主义剥削制所造成的无数残暴、野蛮、荒谬和卑鄙的现象,也就会逐渐\CJKunderdot{\kaishu 习惯于}遵守数百年来人们就知道的、数千年来在一切处世格言上反复谈到的、起码的公共生活规则,自动地遵守这些规则,而不需要强力,不需要强制和服从,\CJKunderdot{\kaishu 不需要}所谓国家的这种\CJKunderdot{\kaishu 特别的}强制\CJKunderdot{\kaishu 机关}。

“国家消亡”这句话说得非常恰当,它即表明了过程的渐进性,又表明了过程的自发性。只有习惯才能够而且一定会发生这样的作用,因为我们随时随地都可以看到,如果没有剥削,如果没有某种会引起抗议、起义并使{\kaishu 镇压}成为必要的令人气愤的现象,那么人们是多么容易惯于遵守他们所必需的公共生活规则。

总之,资本主义社会里的民主制是一种残缺不全的、贫乏的和虚伪的民主制,是只供富人、只供少数人享受的民主制。无产阶级专政,即向共产主义过渡的时期,第一次提供了人民享受的、大多数人享受的民主制,同时对少数人即剥削者实行必要的镇压。只有共产主义才能提供真正完备的民主制,而民主制愈完备,它就愈迅速地成为不需要的东西,愈迅速地自行消亡。

换句话说,在资本主义下存在的是原来意义上的国家,即一个阶级镇压另一个阶级、少数人镇压多数人的特别机器。很明显,为了达到剥削者少数始终压迫被剥削者多数的目的,就必然要采取极凶恶极残酷的镇压手段,就必然会造成无数流血惨案,而这样的流血事件是人类在奴隶制、农奴制和雇佣劳动制下都经历过的。

其次,在资本主义向共产主义{\kaishu 过渡}的时候镇压{\kaishu 还是}}必要的。但这已经是被剥削者多数对剥削者少数的镇压。特别的镇压机关,特别的镇压机器即“国家”,还是必要的,但是已经是过渡性质的国家,已经不是原来意义上的国家,因为由{\kaishu 昨天}还是雇佣奴隶的多数人去镇压剥削者少数人是一件比较容易、比较简单和比较自然的事情,所流的血也会比以前镇压奴隶、农奴和雇佣工人起义时流的少得多,人类为此而付出的代价也要小得多。而且这种镇压同绝大多数居民的广泛的民主是不违背的,因而对{\kaishu 特别的}镇压{\kaishu 机器}的需要就开始消失。自然,如果没有极复杂的镇压机器,剥削者就不能镇压人民,但是{\kaishu 人民}镇压剥削者,却只要有很简单的“机器”,甚至可以不要“机器”,不要特别的机关,而只要有{\kaishu 武装群众的组织}(如工兵代表苏维埃,——我们现在这里提一下)。

最后,只有共产主义才能够为完全不需要国家创造条件,因为那时已经{\kaishu 没有人}须要加以镇压,——这里所谓“没有人”是指阶级而言,是指对某一部分居民进行有系统的斗争而言。我们不是空想主义者,我们丝毫也不否认{\kaishu 个别人}捣乱的可能性和必然性,同样也不否认有镇压{\kaishu 这种}捣乱的必要性。但是,第一、做这件事情用不着什么特别的镇压机器,特别的镇压机关,武装的人民自己会来做这项工作,而且做起来非常简单容易,正像现代社会中任何一群文明人都很容易去劝解打架的人或制止虐待妇女一样。第二、我们知道,产生违反公共生活规则的捣乱行为的社会根源是群众受剥削和群众贫困。这个主要原因一消除,捣乱行为就必然开始“消亡”。虽然我们不知道消亡的速度和进度怎样,但是,我们知道这种行为一定会消亡。国家也会随着这种行为的消亡而{\kaishu 消亡}。

马克思并没有凭空幻想这个未来的远景,他只是更详细地确定{\kaishu 现在}所能确定的东西,即共产主义社会低级阶段和高级阶段之间的差别。




