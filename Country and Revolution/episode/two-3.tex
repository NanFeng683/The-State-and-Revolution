\chapter{1852年马克思对问题的提法$^{*}$} % 小章节名称
%\blfootnote{} % 底线批注,在所需段落前加入
%\pskip % 跳大行,段落间用
%\noindent % 书信体的开头
%\mbox{} % 插入个盒子,设定好的大小,用于隔开文本
%\leftskip= 文本左边腾出空格,用于文本排版
% \CJKunderdot{} % 文字下加点
% {\kaishu } % 更改字体为楷书

\blfootnote{$^{*}$ 第二版增加的一节}
\blfootnote{[1]魏德迈(Joseph·Weydemeyer,1818年~1866年),德国和美国早期工人运动活动家,第一国际美国支部的组织者。}
1907年梅林把1852年3月5日马克思致魏德迈$^{[1]}$的信摘要登载在“新时代”杂志$^{11}$ (第25卷第2期第164页)上。在这封信里有这样一段出色的论述:

\pskip 
\leftskip=10mm
\small

“至于讲到我,无论是发现现代社会中有阶级存在或发现各阶级间的斗争,都不是我的功劳。在我以前很久,资产积极的历史学家就已叙述过阶级斗争的历史发展,资产阶级的经济学家也已对各个阶级做过经济上的分析。我的新贡献就是证明了下列几点:(1)阶级的存在仅仅同生产发展的一定历史阶段(historische Emtwicklungsphasen der Produktion)相联系;(2)阶级斗争必然要导致无产阶级专政;(3)这个专政不过是达到消灭一切阶级和进入无阶级社会的过度”······$^{12}$

\leftskip=0mm
\normalsize
\pskip 

在这一段话里,马克思非常清楚地指出了两点:第一、他的学说同资产阶级最渊博的先进思想家的学说之间的主要和根本的区别;第二、他的国家学说的实质。

马克思学说中的主要内容是阶级斗争。人们时常这样说,这样写。但这是不正确的。根据这个不正确的观点,往往会对马克思主义进行机会主义的歪曲,把马克思主义偷偷地改为资产阶级可以接受的东西。因为阶级斗争学说{\kaishu 不是}由马克思,而是由资产阶级在马克思{\kaishu 以前}创立的,而是一般来说,是资产阶级{\kaishu 可以接受的}。谁要是{\kaishu 仅仅}承认阶级斗争,那他还不是马克思主义者,他可能还没有走出资产阶级思想和资产阶级政策的圈子。用阶级斗争学说来限制马克思主义,就是割裂和歪曲马克思主义,把马克思主义变为资产阶级可以接受的东西。只有承认阶级斗争、{\kaishu 同时}也承认{\kaishu 无产阶级专政}的人,才是马克思主义者。马克思主义者同庸俗小资产者(以及大资产者)之间的最大区别就是这里。必须用这块试金石来验证是否{\kaishu 真正}了解和承认马克思主义。无怪乎当欧洲的历史在{\kaishu 实际上}向工人阶级提出这个问题时,不仅一切机会主义者和改良主义者,而且所有“考茨基主义者”(动摇于改良主义和马克思主义之间的人),都成了否认无产阶级专政的可怜的庸人和小资产阶级民主派。考茨基写的“无产阶级专政”一书,是在1918年8月,即在本书第一版刊行以后很久才出版的,这本书是用市侩的观点歪曲马克思主义、{\kaishu 口头}上假意承认马克思主义而{\kaishu 实际上}卑鄙地背叛马克思主义的典型(见我的“无产阶级革命和叛徒考茨基”1918年彼得格勒和莫斯科版)。

以过去的马克思主义者考茨基为主要代表的现代机会主义,完全陷入了马克思所评述的{\kaishu 资产阶级}立场,因为这个机会主义把承认阶级斗争的范围局限于资产阶级关系的领域以内。(在这个领域、这个范围内,任何一个有知识的自由主义者都不会拒绝在“原则上”承认阶级斗争!)机会主义恰巧在最主要的问题上{\kaishu 不承认}有阶级斗争,即不承认在资本主义向共产主义{\kaishu 过渡}的时期、在{\kaishu 推翻}资产阶级并彻底\CJKunderdot{\textbf{消灭}} 资产阶级的时期有阶级斗争。实际上,这个时期必然是阶级斗争空前残酷、阶级斗争形式空前尖锐的时期,因而这个时期的国家就必须是{\kaishu 新型}的民主国家(对无产者和一般穷人是民主的)和{\kaishu 新型}的专政国家(对资产阶级是专政的)。

其次,一个阶级专政,不仅一般阶级社会需要,不仅推翻资产阶级的{\kaishu 无产阶级}需要,而且,从资本主义过度到“无阶级社会”、过渡到共产主义的整个历史时期都需要,只有了解这一点的人,才算领会了马克思国家学说的实质。资产阶级国家虽然形式非常复杂,但本质上是一个,不管怎样,所有这些国家总是{\kaishu 资产阶级专政}。从资本主义过渡到共产主义,当然不能不产生多种多样的政治形势,但本质必然是一个,就是{\kaishu 无产阶级专政}。












