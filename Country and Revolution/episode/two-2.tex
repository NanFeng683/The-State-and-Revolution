\chapter{革命的总结} % 小章节名称
%\blfootnote{} % 底线批注,在所需段落前加入
%\pskip % 跳大行,段落间用
%\noindent % 书信体的开头
%\mbox{} % 插入个盒子,设定好的大小,用于隔开文本
%\leftskip= 文本左边腾出空格,用于文本排版
% \CJKunderdot{} % 文字下加点
% \part 单独插入标题作为一大页
% {\kaishu } % 更改字体为楷书

关于这个使我们感到兴趣的国家问题,马克思在“拿破仑第三政变记”中总结1848-1851年的革命时写道:

\pskip
\leftskip=10mm
\small

······“然而革命是彻底的。他还在前进中不断锻炼自己。它还在前进中不断锻炼自己。它在有条不紊地执行自己地任务。在1851年12月2日”(拿破仑的第三政变的日子)“以前,它完成了准备工作的前一半,现在正在完成后一半。它首先会使议会权力臻于完备,以便有可能推翻这个权力。现在,它已经做到了这一点,它就使\CJKunderdot{\kaishu 行使权力} 臻于完备,使行政权力达到最纯粹的表现形式,孤立起来,成为自己面前唯一的靶子,\CJKunderdot{\kaishu 以便集中一切破坏力量来反对这个行政权力}”(着重号使我们加的)。“当革命完成后一半准备工作时,欧洲就会跳起来欢呼说:老田鼠挖的好呀!”

这个行政杖力拥有庞大的官僚组织和军事组织,拥有极复杂的人为的国家机器,拥有50万官吏队伍和50万军队——这个像密网一样缠住法国社会的躯体并堵塞其一切毛孔的可怕的寄生机体,是在君主专制时代即封建制度崩溃的时候产生的,这个寄生机体又加速了封建制度的崩溃。”第一次法国革命发展了集中制,“但同时又扩大了政府的职权范围,增加了行政人员。拿破仑完成了这个国家机器”。正统王朝和七月王朝“除了实行更大的分工以外,没有增加任何新东西”······
······“最后,议会制共和国在其反对革命的斗争中,除采用高压手段以外,不得不加强政府权力的工具,不得不使国家权力更加集中。\CJKunderdot{\kaishu 迄今一切政变都是使这个机器更加完备而不是把它摧毁}”(着重号是我们加的),“那些争夺统治权的相互更替的政党,把夺取这个庞大的国家建筑看成是自己胜利时的主要战利品。”(“拿破仑第三政变记”1907年汉堡德文第四版第98一99页$^{10}$

\leftskip=0mm
\normalsize
\pskip

马克思主义在这一段出色的论述里比在“共产党宣言”中向前迈进了一大步。在该书中,国家问题还提得非常抽象,还只有最一般得概念和表述。在这里,问题已经提得具体了,还做出了非常确切、肯定、实际而具体的结论:{\heiti 过去一切革命使国家机器更加完备,但是这个机器是必须打碎,必须摧毁的}。

这个结论是马克思主义国家学说中最主要的基本的东西。正是这个基本的东西,不仅被现在占统治地位的正是社会民主党完全{\kaishu 遗忘},并且被第二国际最著名的理论家考茨基公然{\kaishu 歪曲}了(这点我们在下面还要讲到)。

在“共产党宣言”中做了一般的历史总结,使人们认识到国家是阶级统治的机关,还使人们得出这样一个必然的结论:{\heiti 无产阶级如果不先夺取政权,不取得政治统治,不把国家变成“组织成为统治阶级的无产阶级”,就不能推翻资产阶级};这个无产阶级国家在取得胜利以后就会立刻开始消亡,因为在没有阶级矛盾的社会里,国家是不需要的,是不可能存在的。在这里还没有提出究竟应当(从历史发展的观点来看)怎样以无产阶级国家来代替资产阶级国家的问题。

马克思在1852年提出来加以解决的正是这个问题。马克思是忠于自己的辩证唯物主义哲学的,他拿1848——1851年革命伟大年代的历史经验作为基础。马克思的学说在这里也像其他任何时候一样,是由深刻的哲学世界观和丰富的历史知识阐明的{\kaishu 经验总结}。

国家问题现在提得很具体:资产阶级得国家,资产阶级统治所需要得国家机器在历史上是怎样产生得?在历次资产阶级革命进程中,面临着各种被压迫阶级独立行动的时候,国家机关如何改变,如何演进?无产阶级对这个国家机器的任务怎样?

资本主义社会所特有的集中的国家政权,产生于专制制度崩溃的时代。最能表现这个国家机器特征的有两种机关,即官吏和常备军。马克思和恩格斯的著作中屡次谈到的这两个机器,恰巧同资产阶级有千丝万缕的联系。每个工人的经验都非常清楚非常有力地说明了这种联系。工人阶级由于亲身的体验,现在已经领会到这种联系意味着什么,正因为这样,工人阶级很容易懂得、很深刻地领会到这种联系不可避免的道理,小资产阶级民主派则不是愚蠢地、轻率地否认这一点,便是更轻率地加以“一般地”承认,但是忘了做出相应地实际结论。

官吏和常备军是资产阶级社会躯体上的“寄生虫”,是腐蚀着这个社会的内部矛盾所滋生的寄生虫,而且正是“堵塞”生命的毛孔的寄生虫。目前在正式社会民主党内占统治地位的考茨基机会主义,认为把国家看作一种{\kaishu 寄生机体}的观点是无政府主义独具的特性。这样来歪曲马克思主义,对于那些空前地玷污社会主义,把“保障祖国”的概念应用于帝国主义战争来替这个战争辩护粉饰的市侩,当然是有很大好处的,然而这毕竟是一种绝对的歪曲。

经过从封建制度崩溃以来欧洲所发生的多次资产阶级革命,这个官吏机关和军事机关就逐渐发展、完备和巩固起来。必须指出,小资产阶级被吸引到大资产阶级方面并受它支配多半是通过这个机关的,因为这个机关给农民、小手工业者、商人等等的上层分子以比较舒适、安静和显要的职位,使他们驾于人民之上。我们看一看俄国在1917年2月27日以后这半年中发生的一些事情吧:以前黑帮分子把持的官吏位置,现在成为立宪民主党人、孟什维克和社会革命党的贼物了。实际上他们不想进行任何认真的改革,力图把这些改革拖延到“立宪会议召开的时候”,而召开立宪会议的日期,他们又慢吞吞地拖延到战争结束以后!至于瓜分赃物,摄取部长、次长、总督等肥缺,却没有延期,没有等待任何立宪会议!分配政府要职地把戏,其实不过是全国上下一切中央和地方管理机关中瓜分和重分“赃物”的一种表现。现在各种改革都延期了,官吏职位已经分配完了,分配方面的“错误”也由几次重新分配纠正了,——这无疑是1917年2月27日到8月27日这半年的总结,客观的总结。

但是在各资产阶级政党和小资产阶级政党之间(拿俄国来讲,就是在立宪民主党、社会革命党和孟什维克之间)“重分”官吏机关的事实愈多,各被压迫阶级,首先是领导它们的无产阶级,就会愈清楚地认识到自己同{\kaishu 整个}资产阶级社会不可调和地敌对性。因此,一切资产阶级政党,甚至最民主的和“革命民主的”政党,也必须加强高压手段来反对革命的无产阶级,巩固高压机关,也就是巩固国家机器。这样的事变进程迫使革命“集中一切破坏力量”去反对国家政权,迫使革命提出{\kaishu 破坏}和{\kaishu 消灭}国家机器的任务,而不是去改善国家机器。

这个任务的提出,不是由于逻辑的推论,而是由于实际的事变发展,由于1848—1851年的生动经验。马克思在1852年还没有具体提出\CJKunderdot{\kaishu 用什么东西}去代替这个必须消灭的国家机器的问题,从这里我们可以看出,马克思是怎样严格地依据实际的历史经验的。那时在这个问题上,经验还没有提供材料,后来在1871年,历史才把这个问题提到日程上来。1852年,根据对自然历史过程的精确观察,只能判定无产阶级革命已\CJKunderdot{\kaishu 接近于} 提出“集中一切破坏力量”来反对国家政权的任务,即“摧毁”国家机器的任务。

这里可能会发生这样的问题:吧马克思的经验、观察和结论加以推广,用到比1848-1851这三年法国历史更为广泛的范围上去是否正确呢?为了分析这个问题,我们先回忆一下恩格斯的这一段话,然后再来研究实际材料。

\pskip
\leftskip=10mm
\small

恩格斯在“拿破仑第三政变记”第三版序言里写道:······“法国是这样一个国家,它历史上每一次阶级斗争的结局,都比其他各国更加彻底。它的经常更换的政治形势是最鲜明地表现出阶级斗争地进展及其结果。法国在中世纪时代是封建度的中心,在文艺复兴以后是一个纯粹等级制的典型帝国,它在大革命时代摧毁了封建制度而建立了纯粹的资产阶级统治,它所具有的这种典型的鲜明性,是欧洲其他国家所没有的,抬起头来的无产阶级反对统治的资产阶级的斗争在这里所表现的尖锐形式,也是其他各国从来没有的。”(1907年版第4页)

\leftskip=0mm
\normalsize
\pskip

最后一句话已经过时了,从1871年起,法国无产阶级的革命斗争就停顿了,可是无论停顿多久,法国也还有可能在即将到来的无产阶级革命中成为坚决把阶级斗争进行到底的典型国家。

现在我们来概括地看一看19世纪末20世纪初各先进国家的历史。我们可以看到,这样的过程在更加广阔的场所更缓慢更多样地进行着:一方面,在共和制的国家(法国、美国、瑞士)和君主制的国家(英国、一定程度上的德国、意大利及斯堪的纳维亚半岛各国)里正在形成“议会权力”,另一方面,瓜分和重分官吏职位“赃物”的各资产阶级政党和小资产阶级政党,在不改变资产阶级制度的基础上为争夺政权进行着斗争,最后,“行政权力”及其官吏机关和军事机关日益完备和巩固起来。

毫无疑问,这是现代资本主义国家整个进化过程中的共同特征。在1848-1851这三年内,法国迅速地、尖锐地、集中地表明了整个资本主义世界所固有的那种发展进程。

特别是帝国主义时代,银行资本时代,大资本主义垄断时代,垄断资本主义转变为国家垄断资本主义的时代表明,无论在君主制的国家或最自由的共和制国家,由于要对无产阶级加强镇压,“国家机器”就大大加强起来,它的官吏机关和军事机关也就空前地扩大了。

现在,整个世界历史无疑将在比1852年更加广泛的范围内把无产阶级革命的“一切力量集中起来”去“摧毁”国家机器。

至于无产阶级将会用什么来代替这个国家机器,巴黎公社对这一点提供了极有用的材料。








