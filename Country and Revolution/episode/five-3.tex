\chapter{共产主义社会的第一阶段} % 小章节名称
%\blfootnote{} % 底线批注,在所需段落前加入
%\pskip % 跳大行,段落间用
%\noindent % 书信体的开头
%\mbox{} % 插入个盒子,设定好的大小,用于隔开文本
%\leftskip= 文本左边腾出空格,用于文本排版
% \CJKunderdot{\textbf{}} % 文字下加点
% {\kaishu } % 更改为楷书字体

马克思在“哥达纲领批判”中,详细地驳斥了拉萨尔关于工人在社会主义下将领取“不折不扣的”或“全体劳动产品”的思想。马克思指出,在整个社会的全部社会劳动中,必须拿出一部分作后备基金、作扩大生产的基金和补偿“磨损了的”机器的费用等等,然后在消费品中还要拿出一部分作为管理费以及学校、医院、养老院等等的基金。

马克思不像拉萨尔那样说些含糊不清的笼统的话(“全部劳动产品归工人”),而是对社会主义社会必须怎样管理的问题作了冷静的考察。马克思在{\kaishu 具体}分析这种没有资本主义存在的社会的生活条件时说到:

\pskip
\small
\leftskip=10mm

“我们这里所说的”(在分析工人党的党纲时)“不是在自身基础上{\kaishu 发展起来}的共产主义社会,而是从资本主义社会里刚刚{\kaishu 产生出来}的共产主义社会,因此它在各方面,在经济、道德和思想方面,都还带有它脱胎出来的那个旧社会的痕迹。”

\pskip
\normalsize
\leftskip=0mm

这个刚刚从资本主义脱胎出来的在各方面都还带有旧社会痕迹的共产主义社会,马克思称之为共产主义社会的“第一”阶段或低级阶段。

生产资料已经不是个人的私有财产,它已归整个社会所有。社会的每个成员都担负某一部分社会所必需的工作,并从社会方面领得一张证书,证明他完成了多少工作量。根据这张证书,他从消费品的社会储藏中领取相当数量的产品。这样,除去作为社会基金的一部分劳动之外,每个工人就从社会方面领取相当于他所贡献的一份报酬。

这样,似乎“平等”就实现了。

但是,拉萨尔把这样的社会制度(通常叫做社会主义,而马克思称之为共产主义的第一阶段)说成是“公平的分配”,说成是“每人有获得同等劳动产品的平等权利”,这是错误的,马克思就对他的错误进行了分析。

马克思说:这里确实有“平等权利”,但这\CJKunderdot{\kaishu 还是}“资产阶级的法权”,它同任何权利一样,是\CJKunderdot{\kaishu 以不平等为前提}的。任何权利都是把\CJKunderdot{\kaishu 同一}标准应用在事实上\CJKunderdot{\kaishu 各不相同}、个不相等的人身上,因而“平等权利”就是不平等,就是不公平。的确,每个人付出同别人相等的一份社会劳动,就能领取一份相等的社会产品 (除了上述扣除的以外)。然而每个人是不同的:有的强些,有的弱些;有的结了婚,有的没有结婚;有的子女多些,有的子女少些,以及其他等等。

\pskip
\leftskip=10mm
\small

马克思总结说,······“因此,在同样的劳动下,在平等地享受社会消费品的条件下,某一个人在实际上比另一个人领得多一些,这个人就会比另一个人富裕一些等等。为了避免这一切,权利就不应当是平等的,而应当是不平等的”······

\pskip
\leftskip=0mm
\normalsize

所以,在共产主义第一阶段还不能做到公平和平等,富裕的程度还会不同,不同就是不公平。但是人{\kaishu 剥削}人已经不可能了,因为那时已经不能把工厂、机器、土地等{\kaishu 生产资料}据为己有了。马克思驳倒了拉萨尔关于一般“平等”和“公平”的含糊不清的小资产阶级说法,指出了共产主义社会的{\kaishu 发展进程},说明这个社会最初{\kaishu 只能}消灭私人占有生产资料这一“不公平”现象,却{\kaishu 不能}立即消灭“按工作”(不是按需要)分配消费品这一仍然存在的不公平现象。

\blfootnote{[1]杜冈-巴拉诺夫斯基(М.И Туган-Барановский,1865—1919),俄国经济学家。彼得堡大学教授。最初是“合法马克思主义”者,十月革命后任乌克兰反革命政府财政部长。曾论述俄国资本主义工业的发展,并批判民粹主义观点。后企图以边际效用价值论反对劳动价值论,歪曲马克思经济学说,公开为资本主义制度辩护。}
庸俗的经济学家(包括资产阶级的教授和“我们的”杜刚$^{[1]}$在内)经常谴责社会主义者,说他们忘记了人与人的不平等,“梦想”消灭这种不平等。我们看到,这种谴责只能证明资产阶级思想家先生们的极端无知。

马克思不仅极其准确地估计到人们不可避免的不平等,而且还估计到,仅仅把生产资料转归全社会公有(通常说的“社会主义”)还\CJKunderdot{\kaishu 不能消除}分配方面的缺点和{\kaishu 仍然占着统治地位}的“资产阶级的法权”的不平等,因为产品是“按工作”分配的。

\pskip
\leftskip=10mm
\small

马克思继续说到,······“但是这些缺点在共产主义社会第一阶段,在经过长久的阵痛以后刚刚从资本主义社会脱胎出来的形态中,是不可避免的。权利永远不能超过社会的经济结构以及由经济结构决定的社会的文化发展”······

\pskip
\leftskip=0mm
\normalsize

因此,在共产主义社会的第一阶段(通常成为社会主义),“资产阶级的法权”\CJKunderdot{\kaishu 没有}完全取消,而只是部分地取消,只是在已经发生地经济变革范围内,也就是在对生产资料的关系上取消。“资产阶级的法权”承认生产资料是个人的私有财产。而社会主义则把生产资料变为{\kaishu 公有}财产。{\kaishu 只有在这个范围内},也只能在这个范围内,“资产阶级的法权”才不存在了。

但是它在另一方面却依然存在,依然是社会各个成员间分配产品和分配劳动的调节者(决定者)。“不劳动者不得食”这个社会主义原则{\kaishu 已经}实现了;“按等量劳动领取等量产品”这个社会主义原则也{\kaishu 已经}实现了;“按等量劳动领取等量产品”这个社会主义原则也{\kaishu 已经}实现了。但是,这还不是共产主义,还没有消除不同人按不等量的(事实上是不等量的)劳动领取等量产品的“资产阶级法权”。

马克思说,这是一个“缺点”,但在共产主义第一阶段是不可避免的,如果不愿陷入空想主义,那就不能认为,在推翻资本主义之后,人们立即就会{\kaishu 不需要任何法规}而为社会劳动,况且资本主义的废除{\kaishu 不能立即为这种}变更{\kaishu 创造}经济前提。

可是,除了“资产阶级的法权”以外,没有其他法规。所以在这个范围内,还需要有国家来保卫生产资料公有制,来保卫劳动的平等和分配的平等。

那时国家就会消亡,因为资本家已经没有了,阶级已经没有了,因而也就没有什么{\kaishu 阶级}可以{\kaishu 镇压}了。

但是,国家还没有完全消亡,因为还要保卫容许在事实上存在不平等的“资产阶级的法权”。要使国家完全消亡,就必须有完全的共产主义。






