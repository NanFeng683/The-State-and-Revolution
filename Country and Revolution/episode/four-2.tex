\chapter{同无政府主义者的论战} % 小章节名称
%\blfootnote{} % 底线批注,在所需段落前加入
%\pskip % 跳大行,段落间用
%\noindent % 书信体的开头
%\mbox{} % 插入个盒子,设定好的大小,用于隔开文本
%\leftskip= 文本左边腾出空格,用于文本排版
% \CJKunderdot{\textbf{}} % 文字下加点
% {\kaishu } % 更改字体为楷书

这次论战发生在1873年。马克思和恩格斯曾经把驳斥蒲鲁东主义者、“自治派”或“反权威主义者”的论文寄到意大利,收在一本社会主义文集里,这些论文在1913年才译成德文发表在“新时代”杂志上$^{17}$。

\pskip
\leftskip=10mm
\small

马克思讥笑无政府主义者否认政治时写道,······“如果工人阶级的政治斗争采取革命的形式,如果工人建立起自己的革命专政来代替资产阶级专政,那他们就犯了侮辱原则的莫大罪行,因为工人为了满足自己低微的起码的日常需要,为了打破资产阶级的反抗,竟不放下武器,不废除国家,而赋予国家以革命的暂时的形式”······(“新时代”杂志1913—1914年第32卷第1期第40页)

\normalsize
\leftskip=0mm
\pskip

马克思在驳斥无政府主义者时,只是反对这样的“废除”国家!马克思完全不是反对国家将随阶级的消逝而消逝,或国家将随阶级的废除而废除,而是反对要工人拒绝运用武器,拒绝运用有组织的暴力,{\kaishu 即拒绝}以“打破资产阶级反抗”为目的的{\kaishu 国家}。

马克思故意着重指出无产阶级所需要的国家具有“革命的{\kaishu 暂时的}形式”,以免人们歪曲他同无政府主义者斗争的真实意义。无产阶级需要国家只是暂时的。我们和无政府主义者斗争的真实意义。无产阶级需要国家只是暂时的。我们和无政府主义者都认为废除国家是{\kaishu 目的},在这个问题上完全没有分歧。但我们肯定地说,为了达到这个目的就必须暂时运用国家政权的武器、工具、手段去{\kaishu 反对}剥削者,正如为了消灭阶级,就必须实行被压迫阶级的暂时专政一样。马克思在驳斥无政府主义者时,把问题提得非常尖锐,非常明确,工人在推翻了资本家的压迫以后,应当“放下武器”呢,还是应当利用它来反对资本家,粉碎他们的反抗?一个阶级有步骤地运用武器反对另一个阶级,如果不采用国家那种“暂时的形式”,又采用什么呢?

每一个社会民主党人都应该问问自己:他在同无政府主义者论战时是{\kaishu 这样}提出国家问题的吗?第二国际绝大多数正是社会主义政党{\kaishu 这样}提出国家问题的吗?

恩格斯更加详尽更加通俗地阐明了这个思想。他首先讥笑蒲鲁东主义者地糊涂观念,他们自命为“反权威主义者”,否认任何权威、任何服从、任何权利。恩格斯说,试拿工厂、铁路、航海的轮船来说,这是一些使用机器的、很多人有计划地共同工作的复杂技术企业,如果没有某种从属关系,没有某种权威或权力,那就没有一样能开动起来,这难道还不明显吗?

\pskip
\leftskip=10mm
\small

恩格斯写道:······“如果我拿这种论据来反对最顽固的反权威主义者,那他们只能这样回答:‘是啊!这是对的,但这里说的并不是我们赋予我们的代表的那种权威,而是说的某种委托。’这些人以为只要改变某一事物的名称,就可以改变这一事物本身”······

\normalsize
\leftskip=0mm
\pskip

恩格斯由指明了权威和自治都是相对的概念,运用他们的范围随着社会发展的不同阶段而改变,把它们看作绝对的东西是荒谬的,并且补充说,使用机器和大规模生产的范围在日益扩大,然后恩格斯从权威问题的一般论述谈到国家问题。

\pskip
\leftskip=10mm
\small

恩格斯写道······“如果自治派仅仅是想说将来的社会组织只会在生产条件所必然要求的限度内允许权威存在,那也许还可以同他们说得通。但是,他们闭眼不看一切使权威成为必要的事实,只是拼命反对名词。

为什么反权威主义者不只是限于高喊反对政治权威和反对国家呢?所有的社会主义者都一致认为,国家以及政治权威将由于未来的社会革命而消逝,也就是说,社会职能将失去其政治性质,而变成关心社会利益的简单管理职能。但是,反权威主义者却要求在那些产生政治国家的社会关系废除以前,一举把政治国家废除。他们要求社会革命的第一个行动就是废除权威。

这些先生见过一次革命没有?革命无疑是天下最有权威的东西。革命就是一部分人用枪炮、刺刀,即用非常权威的手段迫使另一部分人服从自己的意志。而获得胜利的政党往往不得不用自己的武器使反动派感到畏惧,来维持自己的统治。要是巴黎公社没有依靠武装人民的权威来反对资产阶级,它能支持一天以上吗?反过来说,难道我们没有理由责备公社把这个权威用得太少吗?总之,二者必居其一。或者是反权威主义者自己不知所云,如果是这样,那他们只是在散步糊涂观念;或者是他们知道这一点,如果是这样,那他们就是背叛无产阶级的事业。在这两种情况下,他们都只是为反动派效劳。”(第39页)$^{[4]}$

\normalsize
\leftskip=0mm
\pskip
\blfootnote{[4]“马克思恩格斯文选(两卷集)1954年莫斯科中文版第1卷第613-614页。”}

在这段论述中涉及了在考察国家消亡时期政治与经济的相互关系(下一章要专门论述这个问题)时应该考察的问题。那就是关于社会职能由政治职能变为简单管理职能的问题和关于“政治国家”的问题。后面这个名词特别容易引起误会,它是指处在消亡过程中的国家,因为正在消亡的国家在它消亡的一定阶段,才可以叫做非政治国家。

恩格斯这段论述中最精彩的地方,又是他用来反驳无政府主义者的问题提法。愿意做恩格斯的学生的社会民主党人,从1873年以来同无政府主义者争论过几百万次,但他们在争论时所采取的态度,恰巧 \CJKunderdot{\kaishu 不是}马克思主义者可以而且应该采取的。无政府主义者对废除国家的观念是糊涂的,而且是{\kaishu 不革命的},恩格斯就是这样提出问题的。无政府主义者不愿看见的,正是革命的产生和发展,以及革命对暴力、权威、政权、国家的特殊任务。

现在社会民主党人通常对无政府主义者的批评,可以归结为纯粹市侩式的极其庸俗的一句话:“我们承认国家,而无政府主义者不承认!”这样庸俗的论调自然不能不使那些稍有思想的革命工人感到厌恶。恩格斯就不是这样谈问题的。他着重指出,所有的社会主义者都承认国家的消逝是社会主义革命的结果。然后他具体提出了关于革命的问题,这个问题机会主义的社会民主党人通常是避而不谈的,他们把它留给无政府主义者来专门“研究”。恩格斯一提出这个问题就抓住了问题的关键:公社难道不应该{\kaishu 更多地}运用{\kaishu 国家的革命}政权,即运用武装起来并组织成为统治阶级的无产阶级的{\kaishu 革命}政权吗?

现在占统治地位的正式社会民主派,对于无产阶级在革命中的具体任务问题,通常是简单地用庸俗的讥笑口吻来敷衍,至多也不过是用诡辩来搪塞,说什么“将来再看吧”。因此无政府主义者有权攻击社会民主派,责备他们背弃了对工人进行革命教育的任务。恩格斯运用最近这次无产阶级革命的经验,正是为了十分具体地研究一下无产阶级对银行和国家究竟应该怎么办。