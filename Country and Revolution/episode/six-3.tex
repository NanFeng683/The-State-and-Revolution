\chapter{考茨基与潘涅库克的论战} % 小章节名称
%\blfootnote{} % 底线批注,在所需段落前加入
%\pskip % 跳大行,段落间用
%\noindent % 书信体的开头
%\mbox{} % 插入个盒子,设定好的大小,用于隔开文本
%\leftskip= 文本左边腾出空格,用于文本排版
% \CJKunderdot{} % 文字下加点
% {\kaishu } % 更换为楷体

\blfootnote{[3]卡尔·伯恩哈多维奇·拉狄克(Карл Бернга́рдович Ра́дек,1885年10月31日-1939年5月19日) 苏联政治活动家,共产主义宣传家,第一次世界大战前在波兰、德国活动,共产国际的早期领导人。为托派重要成员,30年代斯大林大清洗的受害者}
潘涅库克以“左翼激进”派的一个代表的资格出来发对考茨基,在这个派别内有卢森堡、拉狄克$^{[3]}$等人,这个派别坚持革命策略,一致确信考茨基已经转到“中派”立场,无原则地摇摆于马克思主义和机会主义之间。这个看法已经由战争充分证明是正确的,在战时,“中派”(有人称它为马克思主义的派别是错误的)或“考茨基派”充分暴露了自己的全部贫乏可憎。

潘涅库克在一篇论 国家问题的文章“群众行动与革命”(“新时代”杂志1912年第30卷第2册)里,说考茨基的立场是“消极的激进主义”立场,是“毫无作为的等待论”。“考茨基不愿看到革命的过程”(第616页)。潘涅库克这样提出问题之后,就来谈我们所关心的关于无产阶级革命对国家的任务问题。

\pskip
\leftskip=10mm
\small

他写道:“无产阶级的斗争不单纯是{\kaishu 为了夺取}国家政权而反对资产阶级的斗争,而且是{\kaishu 反对}国家政权的斗争······\quad 无产阶级革命的内容,就是用无产阶级的强力武器去消灭和铲除(Aufl\"{o}sung——直译是解散)国家的强力武器······\quad 只有当斗争的最后结果是国家组织的完全破坏时,斗争才告终止。多数人的组织消灭占统治地位的少数人的组织,从这里就证明多数人的组织的优越性。”(第548页)

\pskip 
\leftskip=0mm
\normalsize

潘涅库克表达自己思想的时候在措词上有很大的缺点,但是意思还是很清楚的。现在来看一看考茨基{\kaishu 怎样}反驳这种思想,倒是很有意思的。

\pskip
\leftskip=10mm
\small

考茨基写道:“到现在为止,社会民主党人与无政府主义者之间的对立,就在于前者想夺取国家政权,后者却想破坏国家政权。潘涅库克则即想这样又想那样。”(第742页)

\pskip 
\leftskip=0mm
\normalsize

潘涅库克的说法犯了不明确和不具体的毛病(他的文章的其他缺点因与本题无关,这里暂且不谈),考茨基恰恰是抓住了潘涅库克指出的{\kaishu 具有原则意义}的实质,但是在这个{\kaishu 根本的原则性}的问题上完全离开了马克思主义的立场,完全滚到机会主义那边去了。他对社会民主党人与无政府主义者的区别的论断是完全不对的,马克思主义被他歪曲和庸俗化了。

马克思主义者与无政府主义者之间的区别就在于:(1)马克思主义者的目的是完全消灭国家,但他们认为,只有在社会主义革命把阶级消灭之后,在导向国家消亡的社会主义建成之后,这个目的才能实现;无政府主义者则希望在一天之内完全消灭国家,他们不懂得实现这个目的的条件。(2)马克思主义者认为无产阶级在夺取政权之后,必须彻底破坏旧的国家机器,用新的由武装工人组织组成的公社式的国家机器来代替它;无政府主义者主张破坏国家机器,但是,他们完全没有弄清楚无产阶级应当
用什么去代替它以及{\kaishu 怎样}运用革命政权;无政府主义者甚至否认革命无产阶级运用国家政权,否认无产阶级的革命专政。 (3)马克思主义者主张利用现代国家准备无产阶级进行革命;无政府主义者则否认这一点。

在这个争论中,反对考茨基而代表马克思主义的恰恰是潘涅库克,因为马克思正是教导我们说,无产阶级不能单纯地夺取国家政权,也就是说,不能只是把旧的国家机关
转到新的人手中,而应当打碎、摧毁这个机关,用新的机关来代替它。

考茨基离开了马克思主义而滚到机会主义者那边去了,因为他正好完全抹杀了机会主义者所绝对不能接受的破坏国家机器的思想,把“夺取”只是解释成获得多数,这就
替机会主义者留下了空隙。

考茨基为了掩饰自己对马克思主义的歪曲,就采用了书呆子的方法:“引证”马克思自己的话。马克思在1850年会说必须把“强力完全集中在国家政权中”。考茨基就得意洋洋地问道:潘涅库克不是想破坏“集中制”吗?

这不过是一套把戏,正像伯恩施坦说马克思主义和蒲鲁东主义都主张用联邦制代替集中制一样。

考茨基的“引证”是牛头不对马嘴。无论用旧的国家机器或用新的国家机器,集中制都是可能实现的。工人们自愿地把自己的武装力量集合起来,这就是集中制,但这要以“完全破坏”国家的集中机关即常备军、警察和官僚为基础。考茨基采取了十足的欺骗手段,对大家都知道的马克思和恩格斯关于公社的言论避而不谈,却搬出一些文不对题的引证来。

\pskip
\leftskip=10mm
\small

考茨基继续写道,······“也许是潘涅库克想要消灭官僚的国家职能吧?但是,我们无论在党组织或在公社组织内部都非有官吏不可,更不必说在国家行政机关内了。我们的纲领不是要求消灭国家官吏,而是要求由人民选举官吏”······“现在我们谈的并不是‘未来的国家’的行政机关要采取怎样的形式,而是\CJKunderdot{\kaishu 在我们取得国家政权以前}(着重号是考茨基加的)我们的政治斗争要不要消灭(aufl\"{o}st——直译是解散)国家政权。哪一个部和它的官吏可以消灭呢?”他列举了教育部、司法部、财政部、陆军部。“不,现在内阁中没有一个部是我们反政府的政治斗争可以取消的······\quad 为了避免误会,我再重复一遍:现在谈的不是获得胜利的社会民主党将赋予‘未来的国家’怎样的形式,而是我们作为反对党应该怎样去改变现在的国家。”(第725页)

\pskip 
\leftskip=0mm
\normalsize

这显然是强词夺理。潘涅库克提出的正是{\kaishu 革命}问题。这无论在他的那篇论文的标题上或在上面所引的那段话中都可以明显地看出。考茨基跳到“反对党”问题上去,这正说明他是以机会主义的观点来代替革命的观点。他得出了这样的结论:现在我们是反对党,到取得政权{\kaishu 以后}我们再来专门谈吧。{\kaishu 革命不见了}!这正是机会主义者所需要的。

这里所说的不是反对党,也不是一般的政治斗争,而正是{\kaishu 革命}。革命就是无产阶级\CJKunderdot{\kaishu 破坏}“行政机关”和\CJKunderdot{\kaishu 整个}国家机关,用武装工人组成的新机关来代替它。考茨基暴露了自己对“内阁”的“崇拜”,试问,为什么不可以由——譬如说——拥有全权的工兵代表苏维埃设立的各种专家委员会去代替“内阁”呢?

问题的本质完全不在于是否保留“内阁”,是否设立“专家委员会”或其他什么机关,这根本不重要。问题的本质在于:是保存旧的国家机器(它与资产阶级有千丝万陆的联系,并且浸透了因循守旧的恶习)呢,还是把它{\kaishu 破坏}并用{\kaishu 新的}来代替它。革命不应当是新的阶级利用{\kaishu 旧的}国家机器来指挥、管理,而应当是新的阶级{\kaishu 打碎}这个机器,利用{\kaishu 新的}机器来指挥、管理,——这就是考茨基所抹杀或者是他所完全不了解的马克思主义的{\kaishu 基本}思想。

他提出的关于官吏的问题,清楚地表明他不了解公社的教训和马克思的学说。他说:“我们无论在党组织或在工会组织内都非有官吏不可”······

我们{\kaishu 在资本主义下,在资产阶级统治}下是非有官吏不可的。无产阶级受资本主义的压迫,劳动群众受资本主义的奴役。在资本主义下,由于雇佣奴隶制和群众贫困的整个环境,民主制被缩小、压抑并阉割得残缺不全了。因为这个缘故,而且仅仅因为这个缘故,我们政治组织和工会组织内的负责人被资本主义环境腐化了(确切些说,有被腐化的趋势),有变为官僚的趋势,也就是说,有变为脱离群众、站在群众{\kaishu 头上}的特权者的趋势。

这就是官僚主义的{\kaishu 实质},在基本加被剥夺以前,在资产阶级被推翻以前,{\kaishu 甚至}无产阶级的负责人也不免会在一定程度上“官僚化”。

在考茨基看来,既然被选举出来的公职人员还会存在,那也就是说,在社会主义下官吏也还会存在,官僚制也还会存在!这一点恰恰是不对的。马克思正是以公社为例指出,在社会主义下,公职人员将不再是“官僚”或“官吏”,其所以能如此,那是{\kaishu 因为}除了实行选举制度以外,{\kaishu 还}可以随时撤换,{\kaishu 还}把薪金减到工人平均工资的水平,{\kaishu 并且还}“同时兼管立法和行政的工作机关”去代替议会制机关。

\blfootnote{[4]西德尼·韦伯(Sidney Webb,1859—1947)与比阿特丽丝·波特·韦伯(Beatrice Potter Webb,1858—1943年)夫妇,著名的英国工联主义和费边社会主义理论家,改良主义政治活动家,知识渊博的学者。}
实质上,考茨基用来反驳潘涅库克的全部论据,特别是考茨基说我们无论在工会组织或在党组织内部都有非有官吏不可的那个绝妙的理由,就是过去的伯恩施坦反对马克思主义的那一套“理由”。伯恩施坦在他的那本背叛变节的作品“社会主义的先决条件”中,激烈反对“原始的”民主制思想,反对他所称为“教条主义的民主制”的东西,即绝对服从选民的委托,公职人员不领报酬,中央代表机关软弱无力等等。为了证明这种“原始的”民主制不中用,伯恩施坦就援引了维伯夫妇$^{[4]}$所解释的英国工联的经验。据说,工联根据自己七十年来在“完全自由”(德文版第137页)的条件下发展的情形,确信原始的民主制已不中用,因而用普通的民主制,即与官僚制相结合的议会制来代替它。

其实,工联并不是在“完全自由”的条件下发展起来的,而是在{\kaishu 彻底的资本主义奴隶制下}发展起来的,在这种制度下,自然“不得不”对普遍存在的邪恶现象、暴虐、不公平以及把穷人排斥在“最高”行政机关之外的现象作种种让步。在社会主义下,“原始的”民主制的许多东西必然会复活起来,因为人民{\kaishu 群众}在文明社会史上破天荒第一次站了起来,不仅{\kaishu 自己}来参加投票和选举,而且{\kaishu 自己}来参加{\kaishu 日常管理}。在社会主义下,\CJKunderdot{\kaishu 所有的人}将轮流来管理,因此很快就会习惯于不要任何人来管理。

马克思用自己天才的批判分析才能,在公社所采取的实际办法中看到了一个转变,机会主义者因为胆怯、因为不愿意与资产阶级决裂而害怕这个转变,不愿意承认这个转变,无政府主义者则由于急躁或一般不懂得大规模社会变动的条件而不愿意看到这个转变。“连想也不要想破坏旧的国家机器,我们没有内阁和官吏可不行啊!”——机会主义者就是这样议论的,他们满身市侩气,实际上不但不相信革命和革命的创造力,而且还对革命害怕得要死(像我国孟什维克和社会革命党人害怕革命一样)。

“{\kaishu 只}需考虑破坏旧的国家机器,用不着探究以往无产阶级革命的{\kaishu 具体}教训,也用不着分析应当{\kaishu 用什么}和{\kaishu 怎样}去代替破坏了的东西。”——无政府主义者(当然是指无政府主义者当中的优秀分子,而不是指那些跟着克鲁泡特金先生之流去做资产阶级尾巴的无政府主义者)就是这样议论的;所以他们就采取{\kaishu 绝望的}策略,而不采取那种以大无畏的精神进行革命工作、根据群众运动的实际条件完成具体任务的策略。

马克思教导我们要避免这两种错误,教导我们要勇气百倍地去破坏全部旧的国家机器,同时又教导我们要具体地提问题,要看到公社在数星期内就能够{\kaishu 开始}建立起{\kaishu 新的}无产阶级的国家机器,并实行上述种种方法来扩大民主制和根绝官僚主义。我们要学习公社社员的革命勇气,要把他们的实际办法看成是实际迫切需要并能立刻实现的{\kaishu 初步经验},如果{\kaishu 沿着这样的道路前进},我们就能彻底消灭官僚主义。

消灭官僚主义的可能性是有保证的,因为社会主义将缩短劳动日,发动{\kaishu 群众}去建设新生活,使{\kaishu 大多数}居民\CJKunderdot{\kaishu 都}能够毫无例外地执行“国家职能”,这也就会使一切国家{\kaishu 完全消亡}。

\pskip
\leftskip=10mm
\small

考茨基继续写道,······“群众罢工的任务在任何时候都不能是{\normalsize {\kaishu 破坏}}国家政权,而只能是促使政府在某个问题上让步,或用一个顺从(entgegenkommende)无产阶级的政府去代替敌视无产阶级的政府······\quad 可是,在任何时候,在任何条件下,这”(即无产阶级对敌对政府的胜利)“不能导致国家政权的{\normalsize{\kaishu 破坏}},而只能引起{\normalsize {\kaishu 国家政权内部}}力量对比的某种{\normalsize {\kaishu 变动}}(Verschiebung)······\quad 因此,我们政治斗争的目的,和从前一样,仍然是以取得议会中多数的办法来夺取国家政权,并且使议会变成驾于政府之上的主宰。”(第726、727、732页)

\pskip 
\leftskip=0mm
\normalsize

这就是最纯粹最卑鄙的机会主义,口头上承认革命,实际上却背弃了革命。考茨基的思想仅限于要有一个“顺从无产阶级的政府”,这与1847年“共产党宣言”所宣布的“把无产阶级组织成为统治阶级”的思想比较起来倒是倒退了,落到了庸俗的地步。

考茨基一定会同谢德曼、普列汉诺夫和王德威尔得之流实行他所爱好的“统一”,因为他们都赞成争取一个“顺从无产阶级的”政府。

我们却要同这些社会主义的叛徒决裂,要为破坏全部旧的国家机器而奋斗,使武装的无产阶级自己{\kaishu 成为执政者}。这是“二者之间莫大的区别”。

考茨基一定会成为列金、大卫、普列汉诺夫、波特列索夫、策烈铁里和切尔诺夫之流的亲密伙伴,因为他们完全同意为争取“国家政权内部力量对比的变动”而斗争,为争取“议会中的多数和驾于政府之上的全权议会”而斗争,——这是一个多么崇高的目的,它是机会主义者完全可以接受的,一点没有超出资产阶级议会制共和国的范围。

我们却要同机会主义者决裂;整个觉悟的无产阶级会同我们一起进行斗争,不是为了争取“力量对比的变动”,而是为了{\kaishu 推翻资产阶级,破坏}资产阶级的议会制,建立公社式的民主共和国或工兵代表苏维埃共和国,建立无产阶级的革命专政。

\mbox{}

\mbox{}

在国际社会主义运动中比考茨基更右的派别,在德国有“社会主义月刊”$^{26}$派(列金、大卫、科尔布等人,其中还包括斯堪的那维亚人陶宁格和布兰亭),在法国和比利时有饶勒斯派和王德威尔得,在意大利党内有图拉梯、特雷维斯以及其他右翼代表,在英国有费别社分子和“独立党人”$^{27}$(即“独立工党”,实际上它是始终依附自由派的),如此等等。所有这些先生无论在议会工作中或在党的政论方面都起着很大的而且往往是主要的作用,他们公开否认无产阶级专政,鼓吹露骨的机会主义。在这些先生看来,无产阶级“专政”是与民主“矛盾”的!!他们在实质上跟小资产阶级民主派没有多大区别。

根据这种情况,我们可以得出结论说:第二国际绝大多数正式代表已经完全滚到机会主义那边去了。他们不仅忘记了公社的经验,还要把它加以歪曲。他们没有教导工人群众说,工人们应当起来的时候快到了,应当打碎旧的国家机器而用新的国家机器来代替它,从而把自己的政治统治变为对社会进行社会主义改造的基础。他们不但没有这样教导工人群众,反而教导工人群众相反的东西,他们对“夺取政权”的了解,也给机会主义者留下了无数空隙。

\blfootnote{* 手稿上还有下面这一段:\par

\leftskip=50mm

{\xbsong 第七章}\par

\leftskip=35mm

{\xbsong 1905年和1917年俄国革命的经验}\par

\leftskip=10mm

这一章的题目非常大,可以而且应当写几卷书来论述它。自然,在这本小册子里只能谈一谈与无产阶级在革命中对国家政权的任务直接有关到最主要的经验教训。(手稿到此中断——编者注) }

\leftskip=0mm

现在,为了解决究竟由英国或德国、由这个财政资本或那个财政资本来统治世界的争执,国家及其帝国主义竞赛而加强的军事机关已经变成了屠杀千百万人民的军事怪物,在这个时候曲解和抹杀无产阶级革命对国家的态度问题,就不能不产生极大的影响$^{*}$。

\centering {\rule[-1pt]{2.5cm}{0.1em} }

\leftskip=0mm













