\chapter{国家是阶级矛盾不可调和的产物} % 小章节名称
% \blfootnote{} % 底线批注,在所需段落前加入
% \pskip % 跳大行,段落间用
% \noindent % 书信体的开头
% \mbox{} % 插入个盒子,设定好的大小,用于隔开文本
% \leftskip= 文本左边腾出空格,用于文本排版
% \CJKunderdot{} % 文字下加点
% {\kaishu } % 更改字体为楷书


马克思的学说在今天的遭遇,正如历史上各被压迫阶级解放斗争中的革命思想家和领袖的学说的遭遇一样。当伟大的革命家在世时,压迫阶级总是不断迫害他们,以最恶毒的敌意、最疯狂的仇恨、最放肆的诽谤对待他们的学说。在他们逝世以后,便企图把他们变为无害的神像,即所谓把他们偶像化,赋予他们的{\kaishu 名字}某种荣誉,以便“安慰”和愚弄被压迫阶级,同时却阉割革命学说的内容,磨灭它的革命锋芒,把它庸俗化,这同现在资产阶级和工人运动中的机会主义对马克思主义作的这种“修改”是一致的。他们遗忘、抹杀和歪曲这个学说的革命方面和革命精神,把资产阶级可以接受或者似乎可以接受的东西放在第一位来加以颂扬。现在,一切社会沙文主义者都成了“马克思主义者”(请不要笑!)。那些德国的资产阶级学者,昨天还是摧残马克思主义的专家,现在却愈来愈频繁地谈论起“德意志民族的”马克思来了,仿佛马克思培育极有组织的工人协会是为了进行掠夺战争!

在这种情况下,在歪曲马克思主义的风气空前流行的时候,我们的任务首先就是要{\kaishu 恢复}马克思关于国家的真正学说。为此,必须引证马克思和恩格斯著作中的许多话。当然,很多的引证会使文章冗长,不通俗,但是没有这样的引证是绝对不行的。马克思和恩格斯著作中所有论到国家问题的地方,至少一切有决定意义的地方,我们要尽可能完整地加以引证,一方面是读者对科学社会主义创始人的整个观点以及这些观点的发展有一个独立的概念,同时也可以确切地证实并指明现在占统治地位地“考茨基主义”对这些观点的歪曲。

我们现在先从传播最广的恩格斯的著作“家庭、私有制和国家的起源”一书讲起,这本书于1894年在斯图加特印行了第六版。我们必须根据德文原著译出一段引文,该书俄文译本虽然很多,但多半译得不完全,或者译得很糟。

\pskip
\leftskip=10mm
\small
恩格斯在总结他所做得历史分析时说:“国家绝不是从外面强加于社会得一种力量。国家也不想黑格尔所断言的是‘道德观念的现实’或‘理性的形象和现实’。国家是社会发展到一定阶段的产物;国家是社会陷入自身不可解决的矛盾的表现,是社会分裂为不可调和的对立面而又无力摆脱这种对立情况的表现。为了使这些对立面——这些经济利益彼此冲突的阶级不致在无谓的斗争中互相消灭,使社会同归于尽,于是,一种似乎驾于社会之上的力量,似乎可以缓和冲突、使它不致破坏‘秩序’的力量,就成为必要了。这个从社会中产生、驾于社会之上并日益同社会脱离的力量,就是国家。”(德文第六版第177-178页)$^{3}$

\leftskip=0mm
\normalsize
\pskip
这一段话已经十分清楚地表明了马克思主义关于国家的历史作用及其意义的基本思想。国家使阶级矛盾{\kaishu 不可调和}的产物和表现。在阶级矛盾客观上达到{\kaishu 不能调和}的地方、时候和程度,便产生国家。反过来说,国家的存在表明阶级矛盾的不可调和。

正是在这个最重要的根本问题上,人们从两个主要方面来歪曲马克思主义。

一方面,资产阶级的思想家,贴别是小资产阶级的思想家,迫于无可辩驳的历史事实而不得不承认,只有在有阶级矛盾和阶级斗争的地方才有国家,但他们又来“改正”马克思,说国家是阶级{\kaishu 调和}的机关。在马克思看来,如果阶级调和是可能的话,国家就不会产生,也不会保持下去。在市侩的庸俗的教授和政论家们(他们往往善意地引用马克思的言论!)看来,国家正是用来调和阶级的。在马克思看来,国家是阶级{\kaishu 统治}的机关,是一个阶级{\kaishu 压迫}另一个阶级的机关,是建立一种“秩序”,来使这种压迫合法化、固定化,使阶级冲突得到缓和。在小资产阶级政治家看来,秩序正是阶级调和,而不是一个阶级压迫另一个阶级;缓和冲突就是调和,而不是剥削被压迫阶级用来推翻压迫者的一定的斗争工具和手段。

\blfootnote{[1] 孟什维克(меньшевик),是地处欧洲东部和亚洲北部的俄国早期工人运动中的资产阶级改良主义派别}
例如,在1917年革命的时候,对国家的意义和作用的看法是一个非常重要的问题,是需要在实践中立刻行动,而且是大规模行动的问题,全体社会革命党人和孟什维克$^{[1]}$在这个问题上一下子就完全滚到“国家”“调和”阶级的小资产阶级理论方面去了。这两个政党的无数决议和他们的政治家的许多论文,都浸透了这种市侩的庸俗和“调和”论。国家是一定阶级的统之机关,这个阶级{\kaishu 绝不能}与同它对立的一方(同它对抗的阶级)调和,这一点是小资产阶级民主派始终不能了解的。在对待国家的态度问题上,再明显不过地表明我国社会革命党人和孟什维克根本不是社会主义者(我们布尔什维克向来就这样说),而是唱准社会主义高调地小资产阶级民主派。

另一方面,“考茨基主义”歪曲马克思主义地方法就巧妙得多了。“在理论上”,它不否认国家是阶级统治的机关,也不否认阶级矛盾是不可调和的。但是,它忽视或抹杀了一下一点:既然国家是阶级矛盾不可调和的产物,既然它是驾于社会之上并“\CJKunderdot{\kaishu 日益}同社会\CJKunderdot{\kaishu 脱离}”的力量,那么很明显,被压迫阶级的解放,不仅非暴力革命不可,\CJKunderdot{\kaishu 而且非消灭}统治阶级建立的、体现这种“脱离”的国家政权机关不可。这个结论在理论上是不言而喻的,西面我们会看到,这是马克思对革命的任务做了具体的历史分析后得出的绝对肯定的结论。正是这个结论(我们在下面还要详细说明)竟被考茨基······“遗忘”和歪曲了。




