\chapter{共产主义社会的高级阶段} % 小章节名称
%\blfootnote{} % 底线批注,在所需段落前加入
%\pskip % 跳大行,段落间用
%\noindent % 书信体的开头
%\mbox{} % 插入个盒子,设定好的大小,用于隔开文本
%\leftskip= 文本左边腾出空格,用于文本排版
% \CJKunderdot{\textbf{}} % 文字下加点
% {\kaishu } % 更改为楷书字体

马克思接着又说:

\pskip
\leftskip=10mm
\small

······“在共产主义社会的高级阶段,迫使人们奴隶般地服从社会分工的现象已经消失,脑力劳动和体力劳动的对立也随之消失,劳动已不仅仅是谋生的手段,而且成了生活的第一需要,生产力已随着每个人的全面发展而增长,一切社会财富的资源都会充分地涌现出来,——只有在那时候,才能彻底打破资产阶级法权的狭隘观点,社会才能把‘各尽所能,按需分配’写在自己的旗帜上。”

\pskip
\leftskip=0mm
\normalsize

现在我们才认识到,恩格斯无情地讥笑那种把“自由”和“国家”两个名词连在一起的荒谬见解,是多么正确。还有国家的时候就不会有自由,有了自由就不会有国家。

国家完全消亡的经济基础就是共产主义的高度发展,那时脑力劳动和体力劳动的对立已经消失,因而现代{\kaishu 社会}不平等的最重要根源之一也就消失,而这个根源光靠生产资料转为公有财产,光靠剥夺资本家,是决不能立刻消除的。

这种剥夺会使生产力有蓬勃发展的{\kaishu 可能},既然我们看到资本主义现在已经怎样难以想象地{\kaishu 阻碍着}这种发展,而在现有的技术基础上又可以大大推进生产力,我们就可以有十二分把握地说,剥夺资本家一定会使人类社会地生产力蓬勃发展。但是,生产力将怎样迅速地发展,将怎样迅速地打破分工、消灭脑力劳动和体力劳动的对立,把劳动变为“生活的第一需要”,这都是我们所不知道而且也{\kaishu 不可能}知道的。

因此,我们只能谈国家消亡的必然性,同时着重指出这个过程是长期的,它的长短将取决于共产主义{\kaishu 高级阶段}的发展速度。至于消亡的日期或消亡的具体形式问题,只能作为悬案,因为现在还{\kaishu 没有}可供解决这些问题的材料。

\blfootnote{[2]夏洛克是英国作家莎士比亚的剧本“威尼斯商人”中的一个典型的高利贷者。}
当社会实现“各尽所能,按需分配”的原则时,也就是说,在人们已经十分习惯于遵守公共生活的基本规则,他们的劳动生产率已经大大提高,因此他们能够自愿地{\kaishu 尽其所能}来工作的时候,国家才会完全消亡。那时,“资产阶级法权的狭隘观点”,这种使人像夏洛克$^{[2]}$那样冷酷地斤斤计较,不愿比别人多做半小时工作,不愿比别人少得一点报酬的狭隘观点就会打破。那时,社会就不必在分配产品的时候规定每人应当领取的产品数量;每人将“根据需要”自由地领取。

从资产阶级的观点来看,很容易把这样的社会制度说成是“纯粹的乌托邦”,并冷嘲热讽地说社会主义者许过诺言,要使每个人都有权利向社会领取任何数量地香菌、汽车、钢琴等等,而对每个公民的劳动则可以不加任何监督。就是在今天,大多数资产阶级“学者”也还是用这种讽刺的话来搪塞,但他们这样做只是暴露了他们的愚昧无知和为资本主义辩护的自私目的。

其所以说愚昧无知,是因为没有一个社会主义者想到过要许下共产主义高级发展阶段一定到来的“诺言”,而伟大的社会主义者在{\kaishu 预见}这个阶段将会到来时所设想的前提,既不是现在的劳动生产率,也{\kaishu 不是现在的}庸人,因为这种庸人正如波米亚洛夫斯基小说中的教会学校学生一样,惯于“无故地”破坏社会财富的储藏和提出不能实现的要求。

在共产主义的“高级”阶段到来以前,社会主义者要求社会{\kaishu 和国家}对劳动量和消费量实行{\kaishu 极严格}的监督,不过这种监督应当从剥夺资本家和由工人监督资本家{\kaishu 开始},并且不是由官僚的国家而是由{\kaishu 武装工人}的国家来实行。

资产阶级思想家(和他们的走卒,如策烈铁里先生、切尔诺夫先生之流)为要达到替资本主义辩护的自私目的,就一昧争论和空谈遥远的未来,而{\kaishu 不谈目前}政治上的迫切问题:剥夺资本家,把全体公民变为一个大“辛迪加”即整个国家的工作人员和职员,并使整个辛迪加的全部工作完全服从真正民主的国家,即{\kaishu 工兵代表苏维埃的国家}。

其实,当剥削的教授以及附和他的庸人们和策烈铁里先生、切尔诺夫先生之流谈到荒诞的乌托邦,谈到布尔什维克的蛊惑人心的诺言,谈到不可能“实施”社会主义的时候,他们指的正是共产主义的高级阶段,但是,不仅谁也没有许过“实施”共产主义高级阶段的诺言,而且连想也没有想到“实施”,因为这根本不是可以“实施”的。

这里我们也就要谈到社会主义和共产主义在科学上的差别问题,这个问题在上面引用的恩格斯说“社会民主主义者”这个名称不恰当的一段话里已经谈到。共产主义第一阶段或低级阶段同共产主义高级阶段之间的差别,在政治上说,将来也许很大,但现在在资本主义下着重来谈它就很可笑了,至于把这个差别提到首要地位的也许只有少数无政府主义者(这是说,如果在克鲁泡特金之流,格拉弗、柯尔涅利逊以及其他无政府主义“大师”们“像普列汉诺夫那样”变成了社会沙文主义者,或像一个没有丧失廉耻和良心的无政府主义者格耶所说的那样变成了无政府主义“卫国”战士以后,无政府主义者当中还有人丝毫没有学到什么东西)。

但是社会主义同共产主义在在科学上的差别是很明显的。马克思把通常所说的社会主义称作共产主义社会的“第一”阶段或低级阶段。既然生产资料已成为{\kaishu 公有}财产,那么“共产主义”这个名词在这里也是可以用的,只要不忘记这还{\kaishu 不是}完全的共产主义。马克思的这些解释的伟大意义,就在于他在这里也始终应用了唯物主义辩证法,即发展学说,把共产主义看成是{\kaishu 从}资本主义{\kaishu 中}发展出来的。马克思没有经院式地臆造和“虚构”种种定义,也没有从事毫无意义地字面上地争论(什么是社会主义,什么是共产主义),而是分析了可以表现共产主义在经济上成熟程度地两个阶段。

在第一阶段,共产主义在经济上还{\kaishu 不}可能是完全成熟的,还不能完全摆脱资本主义的传统或痕迹。由此就产生一个有趣的现象,这就是在共产主义第一阶段还保留着“ {\kaishu 资产阶级}法权的狭隘观点”。既然在 {\kaishu 消费品}的分配方面存在着资产阶级的法权,那当然一定要有{\kaishu 资产阶级}的国家,因为如果没有一个能够{\kaishu 迫使}人们遵守法规的机关,权利也就等于零。

可见,在共产主义下,在一定的时期内,不仅会保留资产阶级的法权,甚至还会保留没有资产阶级的资产阶级国家!

这好像是奇谈怪论,或只是一种聪明的辩证把戏,那些没有花过一点功夫去研究马克思主义的极其深刻的内容的人,就常常这样来谴责马克思主义。

其次,无论在自然界或在社会实际生活中,我们随时随地都可以看到新事物中有旧的残余的情形。马克思并不是随便把“资产阶级的”法权塞到共产主义中,而是抓住了刚{\kaishu 从}资本主义{\kaishu 腹内}脱胎出来的社会里那种在经济上和政治上不可避免的东西。

在工人阶级反对资本家,争取解放的斗争中,民主制具有巨大的意义。但是民主制决不是一种不可逾越的极限,它只是从封建主义到资本主义和从资本主义到共产主义的道路上的一个阶段。

民主制意味着平等。很明显,如果把平等正确地了解为消灭{\kaishu 阶级},那么无产阶级争取平等的斗争以及平等的口号就具有伟大的意义。但是,民主制仅仅是{\kaishu 形式上的}平等。一旦社会全体成员{\kaishu 在}占有生产资料{\kaishu 方面}的平等实现以后,也就是说,一旦劳动平等和工资平等实现以后,在人类面前就必然会产生一个问题:要更进一步,从形式上的平等转到事实上的平等,即实现“各尽所能,按需分配”的原则。至于人类会经过哪些阶段,通过哪些实际办法达到这个最高目的,那我们不知道,也不可能知道。可是,必须认识到,资产阶级总是非常虚伪地把社会主义看成是一种僵死的、凝固的、一成不变的东西,实际上,{\kaishu 只有}从社会主义实现时起,社会生活和私人生活的各个方面才会开始真正地迅速地向前推进,形成一个有{\kaishu 大多数}居民甚至全体居民参加的真正群众性的运动。

民主制是一种国家形势,一种国家形态。因此,它同任何国家一样,也对人们使用有组织有系统的强力,这是一方面。但另一方面,民主制在形式上承认公民一律平等,承认大家都有决定国家制度和管理国家的平等权利。而这一点又和下面一点联系着:民主制在其发展的某个阶段首先把反资本主义的革命阶级——无产阶级团结起来,使他们有可能去摧毁、粉碎、连根铲除资产阶级的(哪怕是共和派资产阶级的)国家机器即常备军、警察和官僚,而代之以{\kaishu 更}民主的、但仍然是国家的机器,即由武装工人群众(以后要过渡到全民民警制)构成的国家机器。

这就是“从量变到质变”:达到{\kaishu 这样}高度的民主,就会越出资产阶级社会的范围,开始对社会进行社会主义改造。如果{\kaishu 所有的人}都真正参加国家的管理,那么资本主义就不能支持下去。而资本主义的发展本身却又为“所有的人”真正{\kaishu 能够}参加国家管理创造了{\kaishu 前提}。这种前提就是:在许多最先进的资本主义国家中已经达到了人人都识字,而且千百万工人已经在邮局、铁路、大工厂、大商业企业、银行等等社会化的巨大复杂的机关里“受了训练并养成了遵守纪律的习惯”。

在这种{\kaishu 经济}前提下,完全有可能在一天之内立刻推翻资本家和官僚,由武装工人、普通武装的人民代替他们去{\kaishu 监督}生产和分配,{\kaishu 统计}劳动和产品。(不要把监督和统计的问题同具有科学知识的工程师和农艺师等等的问题混为一谈,这些先生今天在资本家的支配下工作,明天他们就会在武装工人的支配下更好地工作)


统计和监督是把共产主义社会{\kaishu 第一阶段}“安排好”并使它能正确地进行工作所必需的{\kaishu 主要条件}。在这里,{\kaishu 全体}公民都成了国家(武装工人)的雇员。{\kaishu 全体}公民都成了一个全民的、国家的“辛迪加”的职员和工人。全部问题在于要他们做同等的工作,正确完成工作量,领取同等的报酬。资本主义使这种统计和监督变得非常\CJKunderdot{\kaishu 简单},成为一种非常容易、任何一个识字的人都能胜任的监察和登记的手续,只是算算加减乘除和发发有关字据的手续$^{*}$。

\blfootnote{* 当国家的最主要职能简化为由工人自己担当的这样一种统计和监督的时候,国家就不再成为“政治的国家”,那时“社会职能就由政治职能变为简单的行政管理职能”(参看第四章第二节恩格斯同无政府主义者的论战)。}
当{\kaishu 大多数}人民到处开始自己来进行这种统计,对资本家(这时已成为职员)和保留着资本主义恶习的知识分子先生们实行这种监督的时候,这种监督就成为真正包罗万象、普遍的和全民的监督,那时他们就绝对无法逃避这种监督,就会“无处躲藏”了。

整个社会将成为一个管理处,成为一个劳动平等、报酬平等的工厂。

但是,无产阶级在战胜资本家和推翻剥削者以后在全社会推行的这种“工厂”纪律,决不是我们的理想,也决不是我们的最终目的,而只是为了彻底肃清社会上资本主义剥削造成的丑恶现象,并且为了{\kaishu 继续}前进所必需的一个{\kaishu 阶段}。

当社会全体成员或至少绝大多数成员{\kaishu 自己}学会了管理国家,自己账掌握了这个事业,并对一小撮资本家、保留着资本主义恶习的先生们和深深受到资本主义腐蚀的工人们“安排好”监督的时候,任何管理都开始不需要了。民主制愈完备,他成为多余的东西的时候就愈接近。“国家”,即由武装工人组成的、“已经不是原来意义上的国家”愈民主,则{\kaishu 一切的}国家开始消亡也愈快。

当\CJKunderdot{\kaishu 大家}都学会了管理,实际上都自己来管理社会生产,自己来进行统计并对寄生虫、老爷、骗子手等等“资本主义传统保护者”实行监督的时候,企图逃避这种全民的统计和监督就必然很难达到目的,必然只会是极少数的例外,并且还可能受到极迅速极严厉的惩罚(因为武装工人是实事求是的,不像知识分子那样抹不开情面;他们未必会让人随便跟自己开玩笑),这样,人们对于人类一切公共生活的简单的基本规则就会很快从\CJKunderdot{\kaishu 必须}遵守变成\CJKunderdot{\kaishu 习惯}于遵守了。

到那时候,从共产主义社会的第一阶段过渡到它的高级阶段的大门就会敞开,国家也就完全消亡了。






