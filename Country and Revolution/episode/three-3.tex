\chapter{议会制的消失} % 小章节名称
%\blfootnote{} % 底线批注,在所需段落前加入
%\pskip % 跳大行,段落间用
%\noindent % 书信体的开头
%\mbox{} % 插入个盒子,设定好的大小,用于隔开文本
%\leftskip= 文本左边腾出空格,用于文本排版
% \CJKunderdot{} % 文字下加点
% {\kaishu } % 更改字体为楷书


\pskip
\leftskip=10mm
\small

马克思写道:“公社不应当是议会式的,而应当是同时兼管立法和行政的工作机关”······

······“普选制不是为了每三年或六年决定一次,究竟由统治阶级中的什么人在议会里代表和压迫(ver-und zertreten)人民,普选制应当为组织在公社里的人民服务,使他们能为自己的企业找到工人、监工和会计,正如个人选择的权利为了同一目的服务于任何一个工厂主一样。”

\normalsize
\leftskip=0mm
\pskip

由于社会沙文主义和机会主义占了统治地位,这个在1871年对议会制提出的卓越批评,现在也成为马克思主义中“被人遗忘的言论”了。以部长和议员为职业的人们,现今的无产阶级叛徒和“实际的”社会主义者,把批评议会制的事情完全让给无政府主义者去做,而根据这一理由又振振有词地宣布,对议会制的{\kaishu 任何}批评都是“无政府主义”!!难怪“先进的”议会制国家的无产阶级一看到谢德曼、大卫、列金、桑巴、列诺得尔、韩德逊、王德威尔德、斯陶宁格、布兰亭、比索拉蒂之流的“社会主义者”,就产生一种恶感,而日益同情无政府工团主义,尽管无政府工团主义是机会主义的亲兄弟。

但是,马克思从来没有像普列汉诺夫和考茨基等人那样,把革命的辩证法看做是一种时髦的空谈或动听的词藻。马克思善于无情地摒弃无政府主义,因为它不会利用资产阶级议会制的“畜圈”,特别是在显然不具备革命形势的时候,但同时马克思又善于给议会制一种真正的革命无产阶级的批评。

每隔几年决定一次究竟是由统治阶级中的什么人在议会里代表和压迫人民,——这就是资产阶级议会制的真正本质,不仅在议会制的君主立宪国是这样,而且在最民主的共和国内也是这样。

但是,如果提出国家问题,如果把议会看成一个国家机关,那么就无产阶级在{\kaishu 这}方面的任务来说,摆脱议会制的出路何在呢?怎样才能避免议会制呢?

我们不得不一再指出,马克思从研究公社中得出的教训竟被现代的“社会民主党人”(请读作现代的社会主义叛徒)忘掉了,他们只知道对议会制的无政府主义批评或反动批评,简直不懂得任何其他的批评。

摆脱议会制的出路,当然不在于取消代表机关和废除选举制,而在于把代表机关由清谈馆变为“工作”机关。“公社不应当是议会式的,而应当是同时兼管立法和行政的工作机关”。

“不应当是议会式的,而应当是工作”机关,这正好打中了现代社会民主党议会议员和议会“哈巴狗”们的要害!请看一看任何一个议会制的国家,从美国到瑞士,从法国到英国和挪威等等,那里真正的“国家”工作是在后台决定而由各部、官厅和司命部来执行的。议会为了愚弄“老百姓”,专门从事空谈。这是千真万确的事实,甚至在俄国这样的共和国,在这个资产阶级民主共和国还没有来得及建立真正的议会以前,议会制的所有这些弊病就已经显露出来了。腐朽的市侩英雄们,如斯柯别列夫和策列铁里之流,切尔诺夫和阿夫克森齐也夫之流,竟按最卑鄙的资产阶级议会主义方式来玷污苏维埃,把它变成了清谈馆。在苏维埃里,“社会主义的”部长先生们用空谈和决议来愚弄轻信的农民。在政府里,则不断更换角色,一方面为的是依次让更多的社会革命党人和孟什维克尝尝高官厚禄的“甜头”,另一方面为的是“转移”人民的“视线”。而“国家”工作却在官厅和司令部里“作”!

执政的“社会革命党”的机关报“人民事业报”不久以前在一篇社论中,用“大家”都以为政治卖淫为业的“好社会”中的人物的坦率口吻自供说,甚至在“社会主义者”(请原谅我用这个名词!)管的各部中,全体官吏实际上原封未动,他们像以前一样行使着职权,十分“自由地”对革命地创举实行怠工!即使没有这个自供,社会革命党人和孟什维克参加政府地事实不也证明了这一点吗?这里值得注意的是,同立宪民主党人一起呆在政府官场里的切尔诺夫、鲁萨诺夫、晋集诺夫之流以及“人民事业报”的其他编辑先生,竟毫不害羞地、满不在乎地当众宣布说,在“他们的”各部中一切照旧!!革命民主的词句是用来愚弄乡下人的,官僚主义官厅的拖拉作风则是为了博得资本家的“欢心”,这就是“真诚的”联合政府的{\kaishu 实质}。

在公社用来代替资产阶级社会贪污腐败的议会制的机关中,发表意见和讨论的自由不会流为骗局,因为议员必须亲自工作,亲自执行自己通过的法律,亲自检查在实际生活中执行的结果,亲自对选民负责。代表机关仍然存在,然而作为一种特殊的机构,作为立法和行政的分工以及议员们享有特权的议会制,在这里是{\kaishu 不存在}的。如果没有代表机关,那我们就很难想象有什么民主制,也很难想象有无产阶级的民主制;但是,如果我们对资产阶级社会的批评不是空谈,如果推翻资产阶级统治的愿望是我们真正的和真诚的愿望,而不是像孟什维克和社会革命党人、谢德曼、列金、桑巴、王德威尔德之流的那种骗取工人选票的“竞选”词句,那我们可以而且{\kaishu 应当}不要议会制。

非常值得注意的是:马克思在谈到既为公社需要、又为无产阶级民主制需要的{\kaishu 那种}官吏机构的职能时,他拿“任何一个工厂主”雇用的人员来加以比较,即拿雇用“工人、监工和会计”的普通资本主义企业来加以比较。

马克思丝毫没有陷入空想主义,他没有虚构和幻想“新”社会。相反,他把{\kaishu 从}旧社会{\kaishu 诞生}新社会、从前者进到后者的过渡形式,作为一个自然历史过程来研究。他吸取无产阶级群众运动的实际经验,竭力从这个经验中取得实际教训。他向公社“学习”,正像一切伟大的革命思想家不怕向被压迫阶级的伟大运动的经验学习一样,他对这个运动从来没有讲过一句学究式的“训言”(像普列汉诺夫所说的“本来是不需要拿起武器的”,或策列铁里所说的“阶级应当自己约束自己”)。

立刻彻底消灭各地的官吏机构是谈不到的。这是空想。但是立刻{\kaishu 打碎}旧的官吏机器,开始建立一个新的机器,逐步消灭一切官吏机构,这并\CJKunderdot{\kaishu 不是} 空想,这是公社的经验,这是革命无产阶级当前的直接任务。

资本主义使“国家”管理机关的职能简化了,使我们有可能抛弃“长官制”,把全部事情交给以全社会名义雇用“工人、监工和会计”的无产者(统治阶级)组织。

我们不是空想主义者,我们决不“梦想”{\kaishu 立刻}取消任何管理制度和任何从属关系;这种由于不了解无产阶级专政的任务而产生的无政府主义梦想,是与马克思主义根本不相容的。实际上这种梦想只会把社会主义革命拖延下去,直到人们变成另一种人的时候。我们不是这样,我们希望由现在的人们来实行社会主义革命;现在的人们没有从属关系、没有监督、没有“监工和会计”是不行的。

但是应当服从的是一切被剥削劳动者的武装先锋队——无产阶级。国家官吏的特殊“长官制”可以并且应该在一天之内用“监工和会计”的简单职能来代替,这些职能现在只要有一般市民水平的人就能胜任,只要发给“工人的工资”就完全能够执行了。

我们工人将以资本主义创造的成果为基础,依靠自己的经验来{\kaishu 亲自}组织大生产,建立依靠武装工人的国家政权来维护的最严格的铁的纪律,使国家官吏成为不过是执行我们的委托的工作人员,使他们成为对选民负责的、随时可以撤换的而且是领取普通薪金的“监工和会计”(当然还要用各式各样的和具有各种水平的技术人员),这就是{\kaishu 我们}无产阶级的任务,无产阶级革命实现以后,就可以而且应该从这里{\kaishu 开始}做起。在大生产的基础上,这个开始自然会使一切官吏机构逐渐“消亡”,使这个不带引号的、与雇用奴隶制不同的秩序逐渐建立起来,在这个秩序下,日益简化的监督和报告的职能将由大家轮流行驶,等到大家逐渐习惯了遵守秩序,这些职能也就不成其为特殊阶层的{\kaishu 特殊}职能了。

19世纪70年代,有一位聪明的德国社会民主党人认为{\kaishu 邮政}是社会主义经济的范例。这是非常正确的。目前邮政是按国家{\kaishu 资本主义}垄断组织形式组织的一种经济。帝国主义逐渐把所有托拉斯都变成这种类型的组织。现在压在那些工作繁重、忍饥挨饿的“粗笨的”劳动者头上的也正是这个资产阶级的官僚机构。但是社会管理机构在这里已经准备好了。只要推翻资本家,用武装工人的铁拳粉碎这些剥削者的反抗,摧毁现代国家的官僚机器,我们就会有一个派排除了“寄生虫”而拥有高度技术设备的机构,这个机构完全可以由已经团结起来的工人亲自使用,雇用一些技术人员、监工和会计,付给{\kaishu 所有}这些人的工资,也像付给{\kaishu 所有}“国家”官员的工资一样,将相当于工人的工资。这就是对一切托拉斯的具体、实际而且立即可行的任务,这样做会使劳动者免除剥削,并估计到了公社在实践中创造的经验(特别是在国家建设方面的经验)。

把{\kaishu 整个}国民经济组织得像邮政一样,使技术人员、监工、会计以及{\kaishu 所有}公职人员所领的薪金不超过“工人的工资”,使他们受武装无产阶级的监督和领导,这就是我们最近的目标。我们所需要的正是建立在这样的经济基础上的国家。这样才能消灭议会制而保留代表机关,这样劳动阶级才能使这些机关不受资产阶级的糟蹋。










