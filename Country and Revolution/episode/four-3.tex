\chapter{给倍倍尔的信} % 小章节名称
%\blfootnote{} % 底线批注,在所需段落前加入
%\pskip % 跳大行,段落间用
%\noindent % 书信体的开头
%\mbox{} % 插入个盒子,设定好的大小,用于隔开文本
%\leftskip= 文本左边腾出空格,用于文本排版
% \CJKunderdot{\textbf{}} % 文字下加点
% {\kaishu } % 更改字体为楷书

\blfootnote{[5]奥古斯特·倍倍尔(August Bebel,1840年2月22日——1913年8月13日),德国社会主义者,德国社会民主党创始人之一,西欧社会党历史上最受欢迎的杰出人物之一。}
恩格斯在1875年3月18-28日给倍倍尔$^{[5]}$的信中有下面这样一段话,这段话在马克思和恩格斯关于国家问题的著作中,如果不算是最精彩的论述,也得算是最精彩得论述之一。附带说一下,据我们所知,倍倍尔第一次发表这封信是他1911年出版的回忆录(“我的一生”)第2卷里,也就是说在恩格斯发出这封信的三十六年后。

\blfootnote{[6]威廉·白拉克,职业为书商和出版者,德国社会民主工党(爱森纳赫派)创建人和领导人之一。}
恩格斯在给倍倍尔的信里批判了马克思在给白拉克$^{[6]}$的有名的信里批判过的哥达纲领草案,并且特别谈到了国家问题,他写道:

\pskip
\leftskip=10mm
\small

······“自由的人民国家变成了自由国家。从字面上看,自由国家就是可以自由对待本国公民的国家,即具有专制政府的国家。应当抛弃这一切关于国家的废话,特别是在巴黎公社以后,公社已经不是原来意义上的国家了。无政府主义者用‘人民国家’这一名词把我们挖苦得很久了。虽然马克思驳斥蒲鲁东的著作以及后来的‘共产党宣言’都已经直接指出:社会主义社会制度建立以后,国家就会自行消散(sich aufl\"{o}st)和消失。既然国家只是在斗争中、在革命中用暴力镇压敌人的一种暂时机关,那么所谓自由的人民国家就纯粹是无稽之谈。无产阶级还\CJKunderdot{\textbf{需要}}国家的时候,不是为了自由,而是为了镇压自己的敌人,到了有可能来谈自由的时候,国家就不成其国家了。因此,我们建议把‘国家’一词全部改成‘公团’(Gemeinwcsen),这是一个极好的德文古词,相当于法文中的‘公社’。”(德文原本第321-322页)$^{18}$

\normalsize
\leftskip=0mm
\pskip

应当指出,在这封信里谈到了党纲,几星期以后,马克思在一封信(马克思的信写于1875年5月5日)里批判的党纲就是这个党纲;当时恩格斯和马克思一起住在伦敦,因此,恩格斯在最后一句话里用“我们”二字,无疑是以他自己和马克思的名义向德国工人党的领袖建议,把“国家”一词{\kaishu 从党纲中去掉},用“公团”来代替。

如果向现在这些为迁就机会主义者而伪造的“马克思主义”的首领们建议这样来修改党纲,那他们该会怎样大叫大骂“无政府主义”呵!

让他们叫骂吧。资产阶级会因此称赞他们的。

我们还是要做我们自己的事情。在审查我们的党纲时,绝对必须考虑恩格斯和马克思自己的意见,以便更接近真理,以便恢复马克思主义,消除歪曲马克思主义的一切言论,更正确地指导工人阶级争取自身解放的斗争。在布尔什维克当中大概不会有人反对恩格斯和马克思的意见。困难也许只是在名词上。德文中有两个词都作“公团”解释,恩格斯用的那个词{\kaishu 不是}指单独的公团,而是指公团的总和即公团体系。俄文中国没有这样一个词,也许只好采用法文中的“公社”一词,虽然这个词也有它的不便之处。

“公社已经不是原来意义上的国家了”——这是恩格斯在理论上最重要的论断。看了上文以后,这个论断是完全可以理解的。公社{\kaishu 已经不成其为}国家了,因为公社所要镇压的不是大多数居民,而是少数居民(剥削者);它已经打碎了资产阶级的国家机器;居民已经自己上台来代替实行镇压的{\kaishu 特别}力量。所有这一切都已经不是原来意义上的国家了。如果公社得到巩固,那么公社的国家痕迹就会自行“消亡”,它就用不着“废除”国家机关,因为国家机关将无事可做而逐渐失去其作用。

“无政府主义者用‘人民国家’这一个名词挖苦我们”,——恩格斯的这句话首先是指巴枯宁和他对德国社会民主党的攻击说的。恩格斯认为他攻击得对,因为“人民国家”和“自由的人民国家”这两种说法都是荒谬的,都是离开社会主义的。恩格斯力图纠正德国社会民主党人在反对无政府主义者的斗争中的偏差,使这个斗争在原则上正确,清楚它在“国家”问题上的种种机会主义偏见。真可惜!恩格斯的这封信竟被搁置了三十六年。我们在下面可以看到,就在这封信发表以后,考茨基实际上还是顽固地重犯恩格斯警告过的那些错误。

\blfootnote{[7]卡尔·李卜克内西(Karl Liebknecht,1871-1919),德国社会民主党和第二国际左派领袖,德国共产党创始人之一,德国青年运动的领袖,著名的无产阶级革命家,国际共产主义运动中著名的宣传鼓动家和组织家。}
倍倍尔在1875年9月21日写回信给恩格斯,信中顺便谈到他“完全同意”恩格斯对纲领草案的意见,并说他则被了李卜克内西$^{[7]}$的让步态度(见倍倍尔的回忆录德文版第2卷第334页)。但是把倍倍尔的“我们的目的”(《Unsere Ziele》)这本小册子拿来,我们就可以看到一些关于国家的完全不正确的议论:

\pskip
\small

“必须把基于{\kaishu 阶级统治}的国家变成{\kaishu 人民国家}。”(“我们的目的”1886年德文版第14页)

\normalsize
\pskip

这就是倍倍尔那本小册子{\kaishu 第九}版(第九版!)中的话。难怪德国社会民主党还是如此顽固地重复着关于国家问题的机会主义议论,特别是在恩格斯所作的革命解释被人埋藏起来而整个生活环境又长期使人“忘记”革命的时候。




