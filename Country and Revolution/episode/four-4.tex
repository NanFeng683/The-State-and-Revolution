\chapter{爱尔福特纲领草案批判} % 小章节名称
%\blfootnote{} % 底线批注,在所需段落前加入
%\pskip % 跳大行,段落间用
%\noindent % 书信体的开头
%\mbox{} % 插入个盒子,设定好的大小,用于隔开文本
%\leftskip= 文本左边腾出空格,用于文本排版
% \CJKunderdot{\textbf{}} % 文字下加点
% {\kaishu } % 更改字体为楷书

在分析马克思主义的国家学说时,不能不提到恩格斯在1891年6月29日寄给考茨基而过了十年以后才在“新时代”杂志上发表的爱尔福特纲领草案$^{19}$批判,因为这篇文章主要是批判社会民主党在{\kaishu 国家}结构问题上的{\kaishu 机会主义}观点的。

顺便指出,恩格斯还对经济问题作了一个非常宝贵的指示,这说明恩格斯是如何细心、如何深刻地考察了现代资本主义的种种变化,从而说明了他是如何善于在一定程度上预先想到当前帝国主义时代的任务。这个指示是恩格斯在谈到该纲领草案用“无计划(Planlosigkeit)”这几个字来说明资本主义的特征时作的,他写道:

\pskip
\small
\leftskip=10mm

······“当股份公司演进到统治并垄断许多工业部门的托拉斯的时候,不仅私人生产停止了,而且无计划的现象也没有了。”(见“新时代”杂志第20卷1901-1902年第1册第8页)

\normalsize
\leftskip=0mm
\pskip

这里指出了对现代资本主义即帝国主义的理论评价中最主要的东西,即资本主义变成了垄断\CJKunderdot{\kaishu 资本主义}。后面这四个字必须加上着重号,因为目前最普遍的一种错误就是根据资产阶级改良主义的观点来断定,垄断资本主义或国家垄断资本主义{\kaishu 已经不是}资本主义,已经可以称为“国家社会主义”等等。完备的计划性当然是托拉斯所从来没有而且也不可能有的。尽管托拉斯有一点计划性,机关资本大王们预先考虑到了一国范围内甚至国际范围内的生产规模,尽管他们有计划地调节生产,但是现在还是处在{\kaishu 资本主义}下,虽然是在它的新阶段,但无疑还是资本主义。在无产阶级的真正代表来看,这种资本主义之“接近”社会主义,只是证明社会主义革命已经接近,已经不难实现,已经可以实现,已经不容延缓,而绝不是证明可以容忍一切改良主义者否认社会主义革命和粉饰资本主义的言论。

现在我们回过来将国家问题。恩格斯在这里作了三类特别宝贵的指示:第一是关于共和国问题;第二类是关于民族问题和国家结构的联系;第三类是关于地方自治。

关于共和国,恩格斯把它作为批判爱尔福特纲领草案的重点。如果我们还记得当时爱尔福特纲领在整个国际社会民主运动中其有怎样的意义,它怎样成了整个第二国际的模范纲领,那么可以毫不夸大地说,恩格斯在这里是批判了整个第二国际的机会主义。

\pskip
\small
\leftskip=10mm

恩格斯写道:“草案提出的政治要求有一个很大的缺点。\CJKunderdot{\kaishu 草案中没有说}(着重号是恩格斯加的)本来应当说的东西。”

\normalsize
\leftskip=0mm
\pskip

其次,恩格斯解释道,德国的宪法实质上是1850年最反动的宪法的抄本;正如威廉·李卜克内西所说的,德国议会只是“专制制度的遮羞布”;想在规定各小国存在、日耳曼各小国结成联盟的宪法的基础上把“一切劳动工具变为公共财产”,那“显然是荒谬的”。

\pskip
\small
\leftskip=10mm

“谈论这个问题是危险的”,——恩格斯补充说,因为他深知在德国不能在纲领中公开提出建立共和国的要求。但是,恩格斯并不因为这个理由很明显,“大家”都满意,就这样算了。他接着说:“但是,无论如何,事情总要被人推进的。现在在大部分社会民主党的刊物都在散布(einreissende)机会主义,这就说明提出这个要求是多么必要。现在人们因害怕反社会党人法令重新宣布有效,或回想起这个法令的统治下发表的几篇过早的声明,就想要承认德国的现行法制能够和平地实现党的一切要求”······

\normalsize
\leftskip=0mm
\pskip

恩格斯把德国社会民主党人的行动是出于害怕非常法令重新宣布有效这个主要事实提到首位,毫不犹豫地称之为机会主义,而且说明,正是因为在德国没有共和制和自由,所以走“和平”道路的想法是十分荒谬的。但恩格斯非常谨慎,他没有把问题说死了。他承认,在有共和制或有充分自由的国家里,和平地向社会主义发展是“可以想象”的(仅仅是“想象”!),但是在德国,他重复说:

\pskip
\small
\leftskip=10mm

······“在德国,政府几乎有无上的权力,议会和其他一切代表机关都没有实权,因此,在德国宣布某种类似的做法,在没有任何必要时宣布这种做法,就是揭去专制制度的遮羞布,自己去替它遮羞”······

\normalsize
\leftskip=0mm
\pskip

德国社会民主党把这些指示“束之高阁”,党的绝大多数正式领袖果然就成了专制制度的掩护者。

\pskip
\small
\leftskip=10mm

······“这种政策归根到底只能把党引入迷途。他们把一般的抽象的政治问题提到首要地位,从而把那些在巨大事变和政治危机一旦发生就会自然而然地提到日程上来的迫切的具体问题掩盖起来。其结果只会使党在紧要关头突然陷于束手无策的地步,使党对具有决定意义的问题看法模糊、缺乏一致的见解,因为这些问题从来没有经过讨论······

为了眼前暂时的利益而忘记根本大计,只图一时的成就而不顾后果,为了目前而牺牲未来的运动,这种做法也许是出于‘真诚的’动机。但这是机会主义,始终是机会主义,而且‘真诚的’机会主义也许比其他一切机会主义更危险······

毫无疑义的是,我们党和工人阶级只有在民主共和国这种政治形式下才能实行统治。这种民主共和国甚至适用于无产阶级专政的一种特殊形式,这一点已经被法国大革命证明了”······

\normalsize
\leftskip=0mm
\pskip

恩格斯在这里特别明确地重述了像红线一样贯穿在马克思地一切著作中的基本思想,这就是:民主共和国是走向无产阶级专政的捷径。这样的共和国虽然丝毫没有消除资本的统治,因而也丝毫没有消除对群众的压迫和阶级斗争,但是,它必然会使这个斗争扩大、展开和尖锐化,以致一有可能满足被压迫群众的根本利益时,这种可能就必然而且只有在无产阶级专政即在无产阶级领导这些群众的条件下得到实现。对于整个第二国际来说,这也是马克思主义中“被人遗忘的言论”,孟什维克党在俄国1917年革命的开头半年的历史,非常清楚地表明他们也把这些话忘记了。

恩格斯在谈到同居民的民族成分有关的联邦共和国问题时写道:

\pskip
\small
\leftskip=10mm

“应当用什么东西来代替现在的德国呢?”(它拥有反动的君主立宪和同样反动的小国分立制,这种分立制把“普鲁士主义”的种种特点固定下来,而不是把它们溶化在德国的整体中)“在我看来,无产阶级只能采用统一而不可分的共和国的形式。联邦共和国一般来说现在还是美国广大地区所必须的,虽然在它的东部这已经成了障碍。如果在英国建立联邦共和国,那就是前进一步,因为在英国两个海岛上居住着四个民族,虽然议会是统一的,但是有三种立法体系同时并存。联邦共和国在小小的瑞士已成了障碍,那里所以还能容忍联邦共和制,那只是因为瑞士甘愿充当欧洲国家体系中纯粹消极的一员。如果德国实行瑞士式的联邦制,那就是倒退一大步。联邦制国家和完全统一的国家有两点区别,首先一点是每个加盟国都有它特别的民事法规和刑事法规,都有它特别的法院组织;其次,每个加盟国都有与国民议院同时并存的由各加盟国代表组成的议院,在这个议院中,每一个邦无论大小都以一个邦的资格参加表决。”在德国,联盟制国家是转到完全统一的国家的过渡,所以不是要使1866年和1870年的“自上而下的革命”倒退,而是要用“自下而上的运动”来加以补充。

\normalsize
\leftskip=0mm
\pskip

恩格斯对国家形势问题不但不抱冷淡态度,相反,却非常细致地去分析那些过渡形式,以便根据各个不同场合地具体历史特点来估计某一个过渡形式是{\kaishu 从什么到什么}的过渡。

恩格斯同马克思一样,从无产阶级和无产阶级革命的观点出发坚持民主集中制,坚持统一而不可分割的共和国。他认为联邦共和国是一种例外,是发展的障碍,是由君主国向集中制共和国的过渡,也是在一定的特殊条件下的“进步”。在这些特殊条件下,民族问题就提出来了。

恩格斯同马克思一样,虽然无情地批判了小国的反动性和在一定的具体情况下用民族问题来掩盖这种反动性的现象,但是他在任何地方都丝毫没有忽视民族问题的倾向,而荷兰和波兰两国的马克思主义者在反对“自己的”小国的狭隘市侩民族主义的极正当的斗争中,却常常表现出由这种倾向。

无论从地理条件、从共同的语言或从数百年的历史来看,英国似乎已经把各个小地区的民族问题都“解决了”。可是,甚至在这个国家里,恩格斯也注意到一个明显的事实,即民族问题还存在,因此他承认建立联邦共和国是一个“进步”。自然,这里他丝毫没有放弃批评联邦制共和国的缺点,丝毫没有放弃为实现统一的民主集中制的共和国而最坚决地进行宣传和斗争。

恩格斯绝对不像资产阶级思想家和包括无政府主义者在内地小资产阶级思想家那样,从官僚主义的意义上去了解民主集中制。在恩格斯看来,集中制丝毫不排斥广泛的地方自治,只要“公社”和省自愿坚持国家的统一,这种地方自治就一定可以消除任何官僚主义和任何“命令主义”。

\pskip
\small
\leftskip=10mm

恩格斯在发挥马克思主义对于国家问题的纲领性观点时写道,······“总之,需要统一的共和国,但并不是像现在法兰西共和国那样的共和国,因为现在的法兰西共和国同1798年建立的没有皇帝的帝国并没有什么不同。从1792年到1798年,法国的每个省,每个公社(Gemeinde)都有美国式的充分的自治权,而这正是我们所应该有的。至于应当怎样组织地方自治和怎样才可以不要官僚制,这已经由美国和第一个法兰西共和国向我们表明,而现在又有加拿大、澳大利亚以及其他英属殖民地向我们证明了。这种省的和公社的自治制比瑞士的联邦要自由得多,在瑞士,每个邦对整个联邦国家固然具有很大的独立性,但它对县和公社也具有独立性。由邦政府任命县长(Staathalter)和其他地方长官,这在讲英语的国家里是绝对没有的,我们将来在自己国内也必须坚决消除这种现象,取消普鲁士式的Landrat和Regierungsrat”(专员、县长、省长以及所有由上面任命的官吏)。根据这一点,恩格斯建议把党纲关于自治问题的条文表述如下:“各省”(省或区域)“各县和各公社通过普选选出的官吏实行完全的自治;取消由国家任命的一切地方的和省的政权机关。”

\normalsize
\leftskip=0mm
\pskip

\blfootnote{[8]见“列宁全集”1957年人民出版社第24卷第498-501页。}
在被克伦斯基和其他“社会主义”部长的政府封闭的“真理报”(1917年5月28日第68号)$^{[8]}$上,我已经指出过,在这一点上(自然远不止这一点),我国假革命、假民主、假社会主义的代表们是如何惊人地\CJKunderdot{\textbf{离开了民主主义}}。自然,这些同帝国主义资产阶级组成“联合政府”而把自己束缚起来的人对这些指示是充耳不闻的。

必须特别指出的是,恩格斯根据确凿的事实和最确切的例子驳斥了一种非常流行的,特别是在小资产阶级民主派中间非常流行的偏见,即认为联邦制共和国一定要比集中制共和国自由。这种看法是不正确的。恩格斯所举的1792-1798年法兰西集中制共和国和瑞士联邦制共和国的事实推翻了这种偏见。民主集中制共和国赋予的自由实际上比联邦制共和国要{\kaishu 多}。换句话说,在历史上,地方、省等等能够享有{\kaishu 最多}自由的是{\kaishu 集中制}共和国,而不是联邦制共和国。

对于这个事实,以及所有一般关于联邦制共和国与集中制共和国和地方自治的问题,无论过去和现在,在我们党的宣传和鼓动工作中都没有充分注意。