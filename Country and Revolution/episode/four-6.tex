\chapter{恩格斯论民主制的消除} % 小章节名称
%\blfootnote{} % 底线批注,在所需段落前加入
%\pskip % 跳大行,段落间用
%\noindent % 书信体的开头
%\mbox{} % 插入个盒子,设定好的大小,用于隔开文本
%\leftskip= 文本左边腾出空格,用于文本排版
% \CJKunderdot{\textbf{}} % 文字下加点
% {\kaishu } % 更改字体为楷书

恩格斯在谈到“社会民主主义者”这个名称在{\kaishu 科学上}不正确的时候,会连带说到这一点。

\blfootnote{[9]“人民杂志”上关于国际问题的论文。}
\blfootnote{[10]即“全德工人联合会”,德国工人政治组织。其信奉拉萨尔主义,反对马克思主义,把斗争局限于争取普选权和议会活动,严重危害了德国工人运动。}
恩格斯在70年代主要是论述“国际”问题的文集(《Internationales aus dem Volksstaat》$^{[9]}$)的自序(写于1894年1月3日,即恩格斯逝世前一年半的时候)中写道,在所有的论文里,他都用“共产主义者”这个名词,而不用“社会民主主义者”,因为当时法国的蒲鲁东派和德国的拉萨尔派$^{[10]}$都自称为社会民主主义者。

\pskip
\small
\leftskip=10mm

恩格斯接着写道,······“因此在当时,马克思和我两人都觉得,为了特别表明我们的观点,绝对不能用这样一个有伸缩性的名词。现在情况不同了,这个名词(“社会民主主义者”)也许还可以过得去(mag passieren),虽然它对我们这样的党来说仍然是不确切的(unpassend,不恰当的),因为我们党的经济纲领不单单是社会主义的,而且还是共产主义的,党的最终政治目的是完全消除国家,因而也消除民主制。然而\CJKunderdot{\kaishu 真正的(着重号是恩格斯加的)}政党的名称永远不会完全名副其实;党在发展,而名称却没有变。”$^{21}$

\leftskip=0mm
\normalsize
\pskip

辩证法家恩格斯到临终时仍然忠于辩证法。他说:马克思和我有过一个很好的科学上很确切的党的名称,可是当时没有一个真正的政党即群众性的无产阶级政党。现在(19世纪末)真正的政党是有了,可是它的名称在科学上是不正确的。但这不要紧,“可以过得去”,只要党{\kaishu 在发展},只要它意识到它的名称在科学上是不确切的,不让这个名称妨碍它朝着正确的方向发展!

也许某一位爱开玩笑的人会用恩格斯的话来安慰我们布尔什维克说,我们有真正的政党,它很好地在发展;就连“布尔什维克”这样一个毫无意义的奇怪的名称,除了表示我们在1903年布鲁塞尔-伦敦代表大会上占多数这个完全偶然的事实外并没有什么其他意思的名称,也还“可以过得去”······\quad 现在,由于共和党人和“革命”市侩民主派在7-8月间对我党实行迫害,“布尔什维克”这个名词获得了全民的荣誉,这种迫害还表明我党在{\kaishu 真正的}发展过程中迈进了多么伟大的具有历史意义的一步,在这个时候,也许连我自己也对我在4月间提出的改变我党名称的建议表示怀疑了。也许我要向同志们提出一个“妥协办法”:把我们党称为共产党,而把布尔什维克这个名词放在括弧内······

但是党的名称问题远不及革命无产阶级对国家的态度问题重要。

人民通常在谈论国家问题的时候,老师犯恩格斯在这里所警告的而我们在上面也顺便提到的那个错误。这就是:{\heiti 老是忘记国家的消灭也就是民主制的消灭,国家的消亡也就是民主制的消亡}。

乍看起来,这样的论断似乎是极端古怪而难于理解的;甚至也许会有人担心,是不是我们在期待一个不遵守少数服从多数的原则的社会制度,因为民主制也就是承认这个原则。

\textbf{不是的}。民主制和少数服从多数的原则\CJKunderdot{\kaishu 不是}一个东西。民主制就是承认少数服从多数的{\kaishu 国家},即一个阶级对另一个阶级、一部分居民对另一部分居民有系统地使用{\kaishu 强力}的组织。

我们的最终目的是消灭国家,也就是消灭任何有组织有系统的强力,消灭任何加在人们头上的强力。我们并不期待一个不遵守少数派服从多数的原则的社会制度。但是,我们向往社会主义,我们也深信社会主义将发展为共产主义,到那时候就没有任何必要对人们使用强力,没有任何必要使一个人{\kaishu 服从}另一个人,使一部分居民{\kaishu 服从}另一部人居民,因为人们将{\kaishu 习惯于}遵守公共生活的起码条件,而{\kaishu 不需要强力}和{\kaishu 服从}。

为了强调这个习惯的因素,恩格斯也说到了“在新的自由的社会条件下成长起来、能够把国家制度这一堆垃圾完全抛掉”的新的一代,这里所谓国家制度是指任何一种国家制度,其中也包括民主共和的国家制度。

为了说明这一点,就必须分析国家消亡的经济基础问题。
