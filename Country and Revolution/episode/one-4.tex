\chapter{国家“消亡”和暴力革命} % 小章节名称
%\blfootnote{} % 底线批注,在所需段落前加入
%\pskip % 跳大行,段落间用
%\noindent % 书信体的开头
%\mbox{} % 插入个盒子,设定好的大小,用于隔开文本
%\leftskip= 文本左边腾出空格,用于文本排版
% \CJKunderdot{} % 文字下加点
% {\kaishu } % 更改字体为楷书

恩格斯关于国家“消亡”的话是非常著名的,经常有人引证,它清楚地表明了目前流行的把马克思主义偷偷地改为机会主义的把戏的本质,因此我们必须详细地加以说明。现在我们把这句话的出处的那一段论述转录如下:

\leftskip=10mm
\small

“无产阶级取得了国家政权,首先把生产资料变为国家财产。按这样一来,无产阶级就消灭了自己之为无产阶级,就消灭了一切阶级差别和阶级对立,同时也就消灭了国家之为国家。过去和现在在阶级对立中向前发展的社会,需要国家,即需要一个剥削阶级的组织,以便维持其生产的全部外部条件,特别是用强力把被剥削阶级控制在当时生产方式所决定的那些压迫条件(奴隶制、农奴制、雇佣劳动制)以内。国家会是整个社会的正式代表,是社会的集中组织形式,但是国家所以成为这样,只是因为它是当时唯一代表整个社会的阶级的国家。在古代,它是奴隶主即国家公民的国家;在中世纪,它是封建贵族的国家;在我们的时代,它就自然而然地成为多余的东西了。那时候,必须加以镇压的社会阶级已不存在,一个阶级统治另一个阶级的现象以及目前生产无政府状态引起的生存斗争已不然存在,这个斗争中的冲突和过火行动(极端化)也随着消失,再也没有什么东西须要镇压了,于是,实行镇压的特别力量——国家也就不需要了。国家作为全社会的真正代表而采取的第一个行动,即以社会名义占有生产资料,也就是它以国家资格所采取的最后一个独立行动。那时,国家政权对社会关系的干涉,便会逐步成为多余的东西而自行停止。对人的管理将由对物的管理和对生产过程的指导所代替。国家不是‘被废除’的,它是自行消亡的。应该根据这一点来评价‘自由的人民国家’这句话,这句话用来鼓动暂时还可以,但毕竟是没有科学依据的。同时也应该根据这一点来评价所谓无政府主义者要在一天之内废除国家的要求。”(“反杜林论”德文第三版第301-303页)$^{4}$

\pskip
\leftskip=0mm
\normalsize

我们可以大胆地说:在恩格斯这一段思想极其丰富地论述中,只有与无政府主义“废除”国家地学说不同地国家“消亡”(马克思语)这一点,被现代社会主义政党当作真正的社会主义思想接受下来了。这样来割裂马克思主义,无异是把马克思主义编程机会主义,这样来“解释”,就只会留下一个模糊的观念,似乎只有缓慢的、平静的、逐渐的变化,而没有突变、风暴和革命。对国家“消亡”最普遍、最流行、最大众化的(如果能这样说的话)理解,无疑是抹杀革命,甚至是否认革命。

但是,这样的“理解”是对马克思主义的最粗暴的歪曲,仅仅有利于资产阶级。从理论上讲,产生这种歪曲的根源就是忘记了我们上面全部摘引的恩格斯的“总结性”论述中所指出的极重要的情况和观点。

第一、恩格斯在这段论述中一开始就说,无产阶级取得了国家政权,“也就消灭了国家之为国家”。这句话究竟是什么意思,人们“照例”是不假思索的,他们不是完全忽略这一点,就是认为这是恩格斯的“黑格尔主义的毛病”。其实这几句话扼要地表明了最伟大的一次无产阶级革命的经验,即1871年巴黎公社的经验,关于这一点,我们在下面还要详细地加以论述。实际上恩格斯在这里所讲的是以无产阶级革命来“消灭”{\kaishu 资产阶级}的国家,他讲的消亡是指社会主义革命{\kaishu 以后的无产阶级}国家制度的残余。恩格斯认为资产阶级国家是不会“自行消亡”的,而是要用无产阶级革命来“\CJKunderdot{\kaishu 消灭}”它。在这个革命以后,自行消亡的是无产阶级的国家或半国家。

第二、国家是“实行镇压的特别力量”,恩格斯下的这个绝妙而极其深刻的定义是十分明确的。从这个定义可以得出这样的结论:资产阶级对无产阶级,即一小撮富豪对千百万劳动者“实行镇压的特别力量”,应该由无产阶级对资产阶级“实行镇压的特别力量”(无产阶级专政)来代替。这就是“消灭国家之国家”。这就是以社会名义占有生产资料的“行动”。显然,以无产阶级的“特别力量”来代替资产阶级的“特别力量”,这样一种更换是绝不能靠“自行消亡”来实现的。

第三、恩格斯在谈到“自行消亡”和更鲜明更光辉的“自行停止”一语时,他十分明确而肯定地说,这是指在“国家以社会名义占有生产资料”\CJKunderdot{\kaishu 以后},即在社会主义革命\CJKunderdot{\kaishu 以后} 的事情。我们大家都知道,这时“国家”的政治形式是最彻底的民主制。但是那些无耻地歪曲马克思主义的机会主义者,却没有一个人想到恩格斯在这里所说的就是\CJKunderdot{\kaishu 民主制} 的“自行停止”和“自行消亡”。乍看起来,这似乎是很奇怪的。但是,只有那些没有想到民主制\CJKunderdot{\kaishu 也} 是国家、在国家消逝时民主制也会消逝的人,才会觉得这是“不可理解”的。资产阶级的国家只有革命才能“消灭”。最彻底的民主国家只能“自行消亡”。

第四、恩格斯在提出“国家自行消亡”这个著名的原理以后,立刻就具体地说明这一点时既反对机会主义者又反对无政府主义者的。但恩格斯放在首位的,是从这个原理中得出的反对机会主义者的结论。

可以担保,在1万个读过或听过“国家自行消亡”的论述的人中,由9990人完全不知道或不记得恩格斯从这个原理中得出的结论{\kaishu 不仅}是反对无政府主义者的。其余的10个人中可能有9个人不懂得什么是“自由的人民国家”,不懂得为什么反对这个口号就是反对机会主义者。历史就是这样记载的!伟大的革命学说就是这样被人偷偷地改成了流行的庸人主义!人们千百次地重复反对无政府主义者地结论,把它庸俗化并简单地装到头脑中去,形成了固执的偏见。至于反对机会主义者的结论,却被抹杀和“忘记”了!

“自由的人民国家”是70年代德国社会民主党人提出的纲领性要求和流行口号。这个口号只是市侩式地夸大了民主制的概念,没有丝毫政治内容。既然人们当时可以合法地利用这个口号来暗示民主共和国,恩格斯也就从鼓动的观点同意“暂时”替这个口号“辩护”。但这是一个机会主义的口号,它不仅起了粉饰资产阶级民主共和制的作用,而且表现出不懂得社会主义对一般国家的批评。我们拥护民主共和国,因为在资本主义制度下这是对无产阶级最有利的国家形势,但是,我们绝不能忘记,即使在最民主的资产阶级共和国里,人民仍然摆脱不了当雇佣奴隶的命运。在任何国家都是对被压迫阶级“实行镇压的特别力量”。因此{\kaishu 任何}国家都{\kaishu 不是}人民的。在70年代,马克思和恩格斯一再向他们党内的同志解释这一点。

第五、在恩格斯这本著作中,除了大家记得的关于国家消亡的论述以外,还谈到了暴力革命的意义。恩格斯对于革命的作用的历史评价成了对暴力革命的真正颂扬。但是,“谁都不记得”这一点,这个思想的意义在现代社会政党内是照例不谈、照例不想的,在群众进行的日常鼓动宣传中也不占任何地位。其实,这个思想同国家“自行消亡”的理论是密切联系的,是一个严密的整体。

请看恩格斯的论述:
\blfootnote{[8]“反杜林论”1956年人民出版社第190页}
\pskip
\leftskip=10mm
\small

······“暴力在历史上还起着另一种作用”(除作恶以外)“即革命的作用,暴力是替代任何一个孕育着新社会的旧社会接生的产婆(马克思语),暴力是社会运动借以开辟道路并破坏僵死硬化的政治形势的工具和手段,——关于这一切,杜林先生一个字也没有提到。他仅仅带着叹息和呻吟的口吻承认,为了推翻剥削者的经营制度,暴力也许是必要的(你们看,真可惜!),而任何暴力的采用,据说都会使采用暴力的人道德坠落。尽管在每次革命胜利以后,道德和思想都有显著的提高,他还是说出来这种话!而且这是在德国说的,当时德国人民可能被迫进行暴力冲突,发生这种事件的好处至少是可以排除三十年战争的耻辱在民族意识中造成的奴才气。难道可以把这种灰色的、萎靡不振、软弱无力的僧侣思想,推荐给历史上最革命的政党吗?”(“反杜林论”德文第三版第193页;第二编第四章末)$^{[8]}$

\pskip
\leftskip=0mm
\normalsize

怎样才能把恩格斯从1878年起至1894年他逝世时为止,一再向德国社会民主党人提出的颂扬暴力革命的论点,同国家“消亡”的理论结合起来呢?

人们往往用折衷主义,用无原则地或诡辩式地任意(或者为了讨好当权者)抽出前者或后者地方法把它们结合起来,而且在100次中有99次(如果不是更多的话)正是把国家“消亡”论摆在首位,用折衷主义代替辩证法,这就是目前在正式的社会民主党书刊中对马克思主义采取的最常见的最普遍的手法。这种做法,自然并不新奇,甚至在希腊古典哲学史上也是常见的。把马克思主义偷偷地改为机会主语的时候,用折衷主义冒充辩证法是最容易欺骗群众的。这样能使人感到一种似是似非的满足,似乎考虑到了过程的一切方面,发展的一切趋势,各方面的矛盾的影响等等,但实际上并没有对社会发展过程做出任何完整的革命分析。

我们在前面已经说过,马克思和恩格斯关于暴力革命不可避免的学说使针对资产阶级国家说的,在下面我们还要更详细地说明这一点。资产阶级国家由无产阶级国家(无产阶级专政)代替是{\kaishu 不能}经过“自行消亡”来实现的。根据一般规律,只能靠暴力革命来实现。恩格斯对暴力革命的颂扬同马克思的屡次声明完全符合(我们可以回忆一下,马克思在“哲学的贫困”$^{5}$和“共产党宣言”$^{6}$这两部著作的结尾部分,曾自豪地公开生命暴力革命的必然性;我们还可以回忆一下,约在30年以后,马克思在1875年写的“哥达纲领批判”$^{7}$中,曾无情地抨击了这个纲领表现的机会主义),这种颂扬决不是“迷恋”,决不是夸张,也决不是论战伎俩。必须不断教育群众,使他们{\kaishu 这样}来认识暴力革命,而且只能这样来认识暴力革命;这正是马克思和恩格斯{\kaishu 全部}学说的基础。现在占统治地位的社会沙文主义和考茨基主义流派对马克思和恩格斯学说的背叛,最突出地表现在这两个流派都把{\kaishu 这方面}的宣传和鼓动忘记了。

无产阶级国家代替资产阶级国家,必须通过暴力革命。无产阶级国家的消灭,即任何国家的消灭,只能通过“自行消亡”。

马克思和恩格斯在研究每一个革命形势,分析每一个革命的经验教训时,总是详细而具体地发挥了他们的这些见解。我们现在就来谈谈他们学说中这个最重要的部分。











