\chapter{民族统一的建立} % 小章节名称
%\blfootnote{} % 底线批注,在所需段落前加入
%\pskip % 跳大行,段落间用
%\noindent % 书信体的开头
%\mbox{} % 插入个盒子,设定好的大小,用于隔开文本
%\leftskip= 文本左边腾出空格,用于文本排版
% \CJKunderdot{\textbf{}} % 文字下加点
% {\kaishu } % 更改字体为楷书

\pskip
\leftskip=10mm
\small

······"公社在它没有来得及进一步加以发挥的全国组织纲要上十分肯定地说,公社甚至应该······成为一个最小村落的政治形式"······巴黎“国民代表机关”也应当由各公社选举出来。

······“那时仍由中央政府行使的一些为数不多而又非常重要的职能不应该废除(断言应该废除是有意的捏造),而应该交给公社的官吏,即交给那些真正负责的官吏”······

······“民族的统一不应该消灭,相反地应该借助于公社的机构建立起来。要实现民族的统一,必须消灭以民族统一的体现者自居同时却脱离民族、驾于民族之上的国家政权。实际上这个国家政权只是民族躯体上的寄生赘疣”······“任务就在于铲除旧政权的纯粹压迫机关,把这个妄图驾于社会之上的政权的合理职能夺过来,交给社会上负责的公仆”。

\normalsize
\leftskip=0mm
\pskip

\blfootnote{[6]皮埃尔-约瑟夫·蒲鲁东(Pierre-Joseph Proudhon,1809年1月15日——1865年1月19日),法国政论家,经济学家,小资产阶级社会主义者,无政府主义奠基人之一}
叛徒伯恩施坦所著的遗臭万年的“社会主义的先决条件和社会民主党的任务”一书,再清楚不过地表明现代社会民主党内地机会主义者是多么不了解,或者更正确些说,是多么不愿意了解马克思地这些论述。伯恩施坦在谈到马克思上述的一段话时写道,这个纲领“就其政治内容来说,在一切要点上都十分类似蒲鲁东$^{[6]}$主张的联邦制······\quad 不管马克思和‘小资产者’蒲鲁东(伯恩施坦把“小资产者”这几个字放在引号内大概是表示讽刺)之间有种种分歧,可是在这些要点上,他们的思维过程是再接近不过的”。伯恩施坦接着又说:自然,市政局的作用更加重要了,但是,“我怀疑民主制的第一个任务就是废除(Aufl\"{o}sung——直译是解散、融解) 现代国家和完全改变(Umwandlung——变革)其组成,像马克思和蒲鲁东所想象的那样:由各省或各州的会议选出代表组织全国会议,而各省或各州的会议则由各公社选出代表组成,这样,全国代表机关的旧形式就会完全消失”(伯恩施坦:“先决条件”1899年德文版第134页和136页)。

把马克思关于“消灭寄生虫式的国家政权”的观点同蒲鲁东的联邦制混为一谈,这简直就是骇人听闻的事!但这不是偶然的,因为机会主义者从来没有想到,马克思在这里谈的根本不是要建立同集中制对立的联邦制,而是要打碎一切资产阶级国家里旧的资产阶级国家机器。

机会主义者所想的,只是再自己周围、在庸俗的市侩人物中和停滞的“改良主义”环境中看到的东西,只是看到了“市政局”!至于无产阶级革命,机会主义者连想都没有想过。

这是很可笑的。但值得注意的是,在这一点上就没有人同伯恩施坦进行过争论。许多人都会排斥过伯恩斯坦,特别是俄国著作界的普列汉诺夫和西欧著作界的考茨基,但是,无论前者或后者都{\kaishu 没有}谈到伯恩施坦对马克思的{\kaishu 这一}歪曲。

机会主义者不会用革命的头脑来思考革命,他们竟把“联邦制”强加在马克思头上,把他同无政府主义的始祖蒲鲁东混为一谈。而想成为正统派马克思主义者,想捍卫革命的马克思主义学说的考茨基和普列汉诺夫却对此默不作声!这就是考茨基主义者和机会主义者如此庸俗地认为马克思主义同无政府主义的区别的根源之一,关于这种庸俗的观点,我们以后还要讲到。

在上述的马克思关于公社经验的论述中根本没有谈到联邦制。马克思和蒲鲁东相同的地方,恰巧是机会主义者伯恩施坦看不到的;马克思和蒲鲁东不同的地方,恰巧是伯恩施坦认为相同的。

马克思和蒲鲁东相同的地方,就在于他们两人都主张“打碎”现代国家机器。马克思主义同无政府主义(不管是蒲鲁东或巴枯宁)这一相同的地方,无论机会主义者或考茨基派都不愿意看见,因为他们在这一点上离开了马克思主义。

马克思同蒲鲁东和巴枯宁不同的地方,恰巧就在联邦制问题上(更不用说无产阶级专政的问题上了)。联邦制在原则上是从无政府主义的小资产阶级观点产生的。马克思是主张集中制的,在他上述的那段论书中,丝毫没有离开集中制。只有对国家充满市侩“迷信”的人们,才会把消灭资产阶级国家机器看成是消灭集中制!

无产阶级和贫苦农民把国家政权掌握在自己手中,十分自由地组织在公社内,采取{\kaishu 一致}行动打击资本,粉碎资本家的反抗,把铁路、工厂、土地以及其他私有财产交给{\kaishu 整个}民族、整个社会,难道这不是集中制吗?难道这不是最彻底的民主集中制、而是无产阶级的集中制吗?

伯恩施坦根本没有想到可能有自愿的集中制,可能使各公社自愿结合为统一的民族,可能使无产阶级的公社在破坏资产阶级统治和资产阶级国家机器的事业中自愿溶合在一起。伯恩施坦同其他所有庸人一样,以为集中制是职能从上面、只能由官吏和军阀强迫实行和维持的东西。

马克思似乎预料到会有人歪曲他的这些观点,所以故意着重指出,如果非难公社取消民族的统一、废除中央政权,那就是有意的捏造。他故意用“建立民族统一”这句话,以便把自觉的、民主的、无产阶级的集中制同资产阶级的、军阀的、官吏的集中制对立起来。

但是······不愿意听的人比聋子还聋。现代社会民主党内的机会主义者正是不愿意听消灭国家政权、铲除寄生虫这样的话的人。







