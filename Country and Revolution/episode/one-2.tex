\chapter{特别的武装队伍,监狱等等} % 小章节名称
%\blfootnote{} % 底线批注,在所需段落前加入
%\pskip % 跳大行,段落间用
%\noindent % 书信体的开头
%\mbox{} % 插入个盒子,设定好的大小,用于隔开文本
%\leftskip= 文本左边腾出空格,用于文本排版
% \CJKunderdot{} % 文字下加点
% {\kaishu } % 更改字体为楷书

恩格斯又说:
\pskip
\leftskip=10mm
\small

······“国家同旧的氏族(或宗族)组织不同的第一个特征,就是它按地域来划分它统治下的国民”······

\leftskip=0mm
\normalsize
\pskip

我们现在看来,这种划分是“很自然的”,但是这是同宗族或氏族的旧组织进行长期斗争获得的。

\pskip
\leftskip=10mm
\small

······“第二个特征,就是社会权力的建立,这个权力已经不是自己组织成武装力量的居民了。这个特别的社会权力之所以需要,是因为自从社会分裂成阶级以后,已经不可能有居民自动组成的武装了······\quad 这个社会权力在每一个国家里都有存在。构成这个权力的不仅有武装队伍,还有监狱、各种强制机关等物质附属机构,这些东西都是以前氏族社会制度没有的” ······

\leftskip=0mm
\normalsize
\pskip

恩格斯在这里阐明了由社会中产生而驾于社会之上并日益同社会脱离的国家这个力量的概念。这个力量主要是指什么呢?主要是指有用监狱等等的特别武装队伍。

应该说这是特别的武装队伍,因为任何国家所具有的社会权力已经不是武装的居民,不是居民“自动组成的武装”了。

同一切伟大的革命思想家一样,恩格斯竭力促使党悟工人注意的,正是盛行的庸俗观念认为最不值得注意、最习以为常,而被根深蒂固的,可以说是国家的权力的主要的工具,但是——难道可能不是这样吗?

\blfootnote{[2] 赫伯特·斯宾塞(Herbert Spencer,1820-1903)——“社会达尔文主义之父”、进化论先驱}
\blfootnote{[3]米海洛夫斯基(Н.К)——俄国社会学家、政治家,自由主义民粹派的著名代表,俄国主观社会学的创始人之一}
19世纪末叶,绝大多数欧洲人认为,这是不能不这样的。恩格斯的话正是对这些人说的。他们没有经历过,也没有亲眼看到过一次伟大的革命,他们完全不了解,什么是“居民自动组成的武装”。对于为什么要有驾于社会之上并使自己同社会脱离的特别武装队伍(警察、常备军),西欧和俄国的庸人总是喜欢借用斯宾塞$^{[2]}$和米海洛夫斯基$^{[3]}$的几句话来答复,说这是因为社会生活复杂化、职能分化等等。

这种说法似乎是“科学的”,而且最能迷惑庸人,掩盖社会分裂为不可调和敌对阶级这个主要的基本的事实。

如果没有这种分裂,“居民自动组成的武装”同使用棍棒的猿猴群、原始人类,或宗族社会的原始组织比较起来,只是程度上复杂些,技术上高明些,但这样的武装组织是可能的。

这样的组织之所以不可能有,就因为文明社会已分裂为敌对的而且是不可调和地敌对的阶级,如果这些阶级都有“自动组成的”武装,那在它们之间就一定会展开武装斗争。于是国家形成了,特别的力量、特别的武装队伍建立起来了。每当革命破坏国家机关的时候,我们都能清楚地看到,统治阶级是如何力图恢复替\CJKunderdot{\kaishu 它} 服务的特别武装队伍,被压迫阶级又是如何力图建立一种不替剥削者服务,而替被剥削者服务的新型组织。

上面恩格斯从理论上提出的问题,即每次大革命在实践中明显地而且是以大规模的行动提到我们面前的问题。正是“特别”武装队伍同“居民自动组成的武装”之间的相互关系问题。我们会看到,欧洲和俄国历次革命的经验是怎样具体地说明这个问题地。

现在我们再来看看恩格斯的论述。

他指出,有的时候,如在北美某些地方,这种社会权力是薄弱的(这里指的只是资本主义社会中少数的例外,以及在帝国主义时期以前北美那些自由移民占多数的地方),但一般说来,它是在加强:

\pskip
\leftskip=10mm
\small

······“社会权力是随着国家内部阶级矛盾尖锐化及邻国的扩大和人口增多而加强起来的。拿现在的欧洲来说,阶级斗争和侵略竞争已把社会权力提高到可以吞食整个社会,甚至吞食整个国家的地步” ······

\leftskip=0mm
\normalsize
\pskip

\blfootnote{[4]托拉斯(英语:Trust)是商业信托的音译,垄断组织的高级形式之一。}
这段话至迟是在19世纪90年代初期写的。恩格斯最后的序言写于1891年6月16日。当时向帝国主义的转变,无论就托拉斯$^{[4]}$的完全统治、大银行的无限权力或大规模的殖民政策等等来说,在法国还是刚刚开始,在北美和德国要差一点。此后,“侵略竞争”前进了一大步,尤其因为到了20世纪20年代初,世界已被这些“互相竞争的侵略者”,即巨大的强盗国家瓜分完了。从此海陆军备无限增长,1914年至1917年英德两国为了争夺世界霸权、为了瓜分赃物而进行的强盗战争,使社会上一切力量几乎都被强盗国家政权“吞没”,使情况发展到不可收拾的地步。

恩格斯在1891年就已指出,“侵略竞争”是各大强国对外政策最重要的特征之一,但是社会沙文主义的恶棍们在1914年至1917年,正当这个竞争加剧了许多倍并引起了帝国主义战争的时候,却用“保卫祖国”、“保卫共和国的革命”等等词句来掩盖他们维护“自己”资产阶级强盗利益的行为!





