\chapter{革命的前夜} % 小章节名称
%\blfootnote{} % 底线批注,在所需段落前加入
%\pskip % 跳大行,段落间用
%\noindent % 书信体的开头
%\mbox{} % 插入个盒子,设定好的大小,用于隔开文本
%\leftskip= 文本左边腾出空格,用于文本排版
% \CJKunderdot{} % 文字下加点
% {\kaishu } % 更改字体为楷书

成熟的马克思主义的最初著作“哲学的贫困”和“共产党宣言”,恰巧是在1848年革命前夜写成的。由于这种情况,这两本书除了叙述马克思主义的一般原理外,还在一定程度上反映了当时具体的革命形势。因此,我们来研究这两部著作的作者在做出1848-1851年革命经验的总结以前关于国家问题的言论,也许更为恰当。

\pskip
\leftskip=10mm
\small

马克思在“哲学的贫困”中写道,······“工人阶级在发展进程中将创造一个没有阶级和阶级对立的联合体来代替旧的资产阶级社会;从此再不会有任何原来意义的政权了,因为政权正是资产阶级社会内阶级对抗的正式表现。”(1885年德文版第182页)$^{8}$

\leftskip=0mm
\normalsize
\pskip

拿马克思和恩格斯在几个月后(1847年11月)写的“共产党宣言”中的论述来同这一段关于国家会随着阶级消灭而消逝的思想的论述比较一下,是颇有教益的。

\pskip
\leftskip=10mm
\small

······“当我们叙述无产阶级发展中最一般的阶段时,我们探讨了现存社会里多少带有隐蔽性的国内战争,一直到这个国内战争变成公开的革命,那是无产阶级就用暴力推翻资产阶级而建立自己的统治”······

······“我们在前面已经看见,工人革命的第一步是无产阶级变为(直译是提升为)统治阶级,争得民主。”

“无产阶级利用自己的政治统治,一步步地夺取资产阶级地全部资本,把一切生产工具集中到国家即组成统治阶级的无产阶级手中,并尽量迅速地增殖生产力总量。”(1906年德文第七版第31页和37页)$^{9}$

\leftskip=0mm
\normalsize
\pskip

在这里我们看到马克思主义关于国家问题的一个最卓越最重要的思想,即“无产阶级专政”(如马克思和恩格斯在巴黎公社以后所说的那样)的思想,其次我们还看到国家下的一个十分重要的定义,这个定义也属于马克思主义中“被人遗忘的言论”:“{\kaishu 国家即组成为统治阶级的无产阶级}”。

国家的这个定义,在各正式社会民主党最流行的宣传鼓动书刊中从来没有解释过。它恰巧被遗忘,因为它同改良主义是根本不相容的,它直接打击了“民主制和平发展”的一般机会主义偏见和市侩幻想。

无产阶级需要国家——一切社会主义者,社会沙文主义者和考茨基主义者都这样重复,硬说马克思的学说就是如此,但是“{\kaishu 忘了}”补充:马克思认为,第一、无产阶级所需要的只是逐渐消亡的国家,即需要建立一个立刻开始消亡而且不能不消亡的国家;第二、劳动者所需要的“国家”,就是“组织成为统治阶级的无产阶级”。

国家是特殊的组织,是用来镇压某一个阶级的强力组织。无产阶级要镇压的究竟是哪一个阶级呢?当然只是剥削阶级,即资产阶级。劳动者需要国家只是为了镇压剥削者的反抗,而能够领导和实行这种镇压的只有无产阶级,因为无产阶级是唯一彻底革命的阶级,是唯一能够团结一切被剥削劳动群众去反对资产阶级并把它完全铲除的阶级。

剥削阶级需要政治统治是为了维持剥削,也就是为了极少数人的私利去反对绝大多数人民。被剥削阶级需要政治统治,是为了彻底消灭一切剥削,也就是为了绝大多数人民的利益去反对极少数的现代奴隶主——地主和资本家。

小资产阶级民主派,这些以幻想阶级妥协来代替阶级斗争的假社会主义者,对于社会主义改造也抱着幻想,他们不是把改造设想为推翻剥削阶级的统治,而是设想为少数驯顺地服从那意识到了本身任务的多数。这种小资产阶级空想同认为国家是超阶级的观点有密切的联系,这种空想在实际上必然导致出卖劳动阶级的利益,法国1848年革命和1871年革命的历史已经证明了这一点,19世纪末和20世纪初英、法、意等国的“社会主义者”参加资产阶级内阁的经验也表明了这一点。

马克思一生都在反对这种小资产阶级社会主义,即目前在俄国由社会革命党和孟什维克复活起来的这种小资产阶级社会主义。马克思一贯坚持阶级斗争的理论,并把它贯彻到政权学说、国家学说之中。

只有无产阶级这一特殊阶级才能推翻资产阶级的统治,因为无产阶级生存的经济条件使它做好了推翻资产阶级统治的准备,使它有可能、有力量达到这个目的。资产阶级一方面分离和拆散农民及一切小资产者阶层,另一方面也使无产阶级联合、团结和组织起来。只有无产阶级,由于它在大生产中的经济作用,才能成为{\kaishu 一切}被剥削劳动群众的领袖,这些被剥削劳动群众受资产阶级的剥削、压迫和蹂躏往往比无产阶级更厉害,可是他们不能为自己的解放进行{\kaishu 独立}的斗争。

马克思在国家和社会主义革命问题上运用的阶级斗争学说,必然会承认无产阶级的{\kaishu 政治统治} ,承认无产阶级专政,即承认不与任何人分掌而直接依靠群众武装力量的政权。只有使无产阶级变为{\kaishu 统治阶级},变为能够镇压资产阶级必然要进行拼命的反抗、能够组织{\kaishu 一切}被剥削劳动群众来建立新的经济制度的统治阶级,才能推翻资产阶级。

无产阶级需要国家政权,集中的权力组织,强力组织,为的是镇压剥削者的反抗和{\kaishu 领导}广大民众即农民、小资产阶级和半无产阶级来“组织”社会主义经济。

马克思主义教育工人的党,教育无产阶级的先锋队,使它能够夺取政权并{\kaishu 引导全体人民}走向社会主义,领导建立新制度,成为所有被剥削劳动者在没有资产阶级参加并反对资产阶级而建立自己社会生活的事业中的导师、领导者和领袖。反之,现在占统治地位的机会主义却把工人的党教育成为一群脱离群众而代表工资优厚的工人的人物,只图在资本主义制度下“苟且偷安”,为了一碗稀饭而出卖长子的权利,即放弃权利不当领导人民反对资产阶级的革命领袖。

“国家即组织成为统治阶级的无产阶级”,——马克思的这个理论同他关于无产阶级在历史上的革命作用的全部学说,有不可分割的联系。这种作用的最高表现是无产阶级专政,无产阶级的政治统治。

既然无产阶级需要国家这样一个{\kaishu 反对} 资产阶级的{\kaishu 特殊}强力组织,那末自然就会得出一个结论:不预先消灭和破坏资产阶级为自己建立的国家机器,就不可能建立这样一个组织。在“共产党宣言”中已接近于得出这个结论,马克思在总结1848—1851年革命的经验时也就谈到了这个结论。
























