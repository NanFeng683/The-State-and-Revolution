\chapter{“住宅问题”} % 小章节名称
%\blfootnote{} % 底线批注,在所需段落前加入
%\pskip % 跳大行,段落间用
%\noindent % 书信体的开头
%\mbox{} % 插入个盒子,设定好的大小,用于隔开文本
%\leftskip= 文本左边腾出空格,用于文本排版
% \CJKunderdot{\textbf{}} % 文字下加点
% {\kaishu } % 更改字体为楷书

恩格斯在他论住宅问题的著作(1872年出版)中,已经估计到了公社的经验,并且屡次谈到革命对于国家的任务。值得注意的是,他在谈到这个具体问题时,一方面明显地说明了无产阶级国家同现代国家相似的特征,根据这些特征我们可以把前者和后者都称为国家;另一方面他又说明了二者之间不同的特征,即无产阶级国家是要逐渐消亡的。

\pskip
\leftskip=10mm
\small

“怎样解决住宅问题呢?在现在的社会里,解决这个问题同解决其他一切社会问题完全一样,即靠供求关系在经济上的逐渐均衡来解决,但是这样解决之后这个问题还会不断产生,也是就说,等于没有解决。社会革命将怎样解决这个问题呢?这不仅要以时间地点为转移,而且也同一些带根本性的问题有关,其中最重要的就是消灭城乡对立的问题。既然我们不预备凭空设想未来社会结构的制度,现在各大城市中有足够的住宅,只要合理使用,就能够使真正{\kaishu 需要}住宅的人立刻得到帮助。当然,要实现这一点,就必须剥夺现在的房东,让没有房子住或住得很挤的工人搬到这些住宅里去。只要无产阶级取得了政权,这种有关社会福利的措施就会像现代国家剥夺其他东西的措施和占据住宅那样容易实现。”(1887年德文版第22页)$^{16}$

\normalsize
\leftskip=0mm
\pskip

这里没有谈国家政权形式的改变,只是谈到了国家政权活动的内容。根据现代国家的命令也就是要剥夺住宅和占据住宅的,无产阶级的国家,从形式上来讲,也会“下令”占据和剥夺住宅。但是很明显,旧的执行机关,即与资产阶级勾结的官吏机构,是完全不能执行无产阶级国家的命令的。

\pskip
\leftskip=10mm
\small

······“必须指出,由劳动人民实际占有一切劳动工具和全部工业,是同蒲鲁东主义的‘赎买’政策完全相反的。如果采用后一种措施,则每个工人将成为某一所住宅、某一块土地、某些劳动工具的所有者;如果采用前一种措施,则‘劳动人民’将成为全部住宅、工厂和劳动工具的集体所有者。这些住宅、工厂等等,至少在过渡时期还未必会毫无代价地分配给个人或共耕社使用。同样,消灭土地私有制不是消灭地租,而是用另一种形式把地租转交给社会。所以由劳动人民实际占有一切劳动工具,无论如何都不排斥承租和出租。”(第68页)$^{[1]}$

\normalsize
\leftskip=0mm
\pskip

\blfootnote{[1]“马克思恩格斯文选(两卷集)1954年莫斯科中文版第1卷第605页。”}
我们在下一章将要考察在这段论述中提到的问题,即关于国家消亡的经济基础的问题。恩格斯非常谨慎,他说无产阶级国家“至少在过渡时期”“未必”会毫无代价地分配住宅。把属于全民的住宅租给个别家庭就要征收租金,要实行一定的监督并规定分配住宅的某种标准。这一切都是要求有一定的国家形式,但决不需要特别的军事官僚机关及其享有特权的长官。至于过度到免费分配住宅,那是与国家的完全“消亡”联系着。

\blfootnote{[2]布朗基主义(法语:Blanquisme)是19世纪中期工人运动中的革命冒险主义的思潮,其强调社会主义革命应由相对少数的高度组织化和秘密的密谋者施行,主张依靠少数革命家的密谋活动来推翻资产阶级的统治,建立少数人的专政,一步就跳到共产主义。}
恩格斯谈到布朗基主义者$^{[2]}$在公社以后因受到公社经验的影响而转到马克思主义的原则立场上的时候,会顺便把这个立场表述如下:

\pskip
\leftskip=10mm
\small

······“无产阶级必须采取政治行动、实行专政,是为了过渡到废除阶级并废除国家”······(第55页)$^{[3]}$

\normalsize
\leftskip=0mm
\pskip
\blfootnote{[3]同上第589页}

那些喜欢咬文嚼字的批评家或者“糟蹋马克思主义”的资产阶级分子大概认为,在这里{\kaishu 承认}“废除国家”,在上述“反杜林论”的一段论述中又把这个公式当作无政府主义的公式加以否定,这是矛盾的。如果机会主义者把恩格斯算作“无政府主义者”,那并没有什么奇怪,因为社会沙文主义者斥责国际主义者是无政府主义者的做法现在已经成为一种风气了。

国家会随着阶级的废除而废除,这是马克思经常教导我们的。“反杜林论”的那段人所共知的关于“国家消亡”的论述,并不是单纯斥责无政府主义主张废除国家,而是斥责他们宣传可以“在一天之内”废除国家。

现在占统治地位的“社会民主主义”学说把马克思主义在消灭国家问题上对无政府主义的态度完全歪曲了,因此我们来回忆一下马克思和恩格斯同无政府主义者的一次论战,是特别有益的。

