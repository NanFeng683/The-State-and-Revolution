\chapter{马克思对这个问题的提法} % 小章节名称
%\blfootnote{} % 底线批注,在所需段落前加入
%\pskip % 跳大行,段落间用
%\noindent % 书信体的开头
%\mbox{} % 插入个盒子,设定好的大小,用于隔开文本
%\leftskip= 文本左边腾出空格,用于文本排版
% \CJKunderdot{\textbf{}} % 文字下加点
% {\kaishu } % 更改为楷书字体

如果把马克思在1875年5月5日给白拉克的信同上述恩格斯在1875年3月28日给倍倍尔的信比较一下,从表面上看也许会觉得马克思比恩格斯带有更浓厚的“国家派”色彩,也许会觉得这两位著作家对国家的看法有很大差别。

恩格斯劝倍倍尔根本抛弃关于国家的废话,把“国家”一词从党纲中完全去掉而用“公团”来代替;恩格斯甚至宣布公社已经不是原来意义上的国家。而马克思却还谈到“未来共产主义社会的国家制度”,这就是说,似乎他认为就是在共产主义下也还要有国家。

但这种看法是根本不对的。如果仔细研究一下,就可以知道马克思和恩格斯对国家和国家消亡问题的看法是完全一致的,上面所引的马克思的话也就是指{\kaishu 正在消亡的}国家制度。

至于确定{\kaishu 将来}“消亡”的日期,这当然无从谈起,但很明显,它是一个长期的过程。马克思和恩格斯之间所以有表面上的差别,是因为他们研究的问题和研究的目的不同。恩格斯的目的是要清楚地、尖锐地、扼要地向倍倍尔指明,当时流行的(也是拉萨尔颇为赞同的)关于国家问题的偏见是完全荒谬的。而马克思只是在论述另一个问题即共产主义社会{\kaishu 发展}的时候,顺便提到了\CJKunderdot{\kaishu 这个}问题。

马克思的全部理论,就是运用最彻底、最完整、最周密、内容最丰富的发展论去考察现代资本主义。自然,他也运用这个理论去考察资本主义{\kaishu 即将}崩溃的问题,去考察{\kaishu 未来的}共产主义的{\kaishu 未来}发展问题。

究竟有什么{\kaishu 根据}可以提出未来共产主义的未来发展问题呢?

这里的根据就是共产主义是从资本主义中{\kaishu 产生}的,它在历史上是从资本主义中发展起来的,它是资本主义{\kaishu 产生}的那种社会力量发生作用的结果。马克思丝毫不想制造乌托邦,不想凭空猜测无法知道的事情。马克思提出共产主义的问题,正像自然科学家提出某一新的生物变种的发展问题一样,因为我们已经知道,这一变种是怎样产生以及朝着哪个方向演变的。

马克思首先扫除了哥达纲领对国家同社会的相互关系问题的糊涂观念。

%\pskip
\leftskip=10mm
\small

他写道,······“现代社会,就是一切文明国家里的资本主义社会,都或多或少地摆脱了中世纪的杂志,仅仅因为每个国家历史发展的特点而在形态上多少有些不同,在发展程度上也多少有些不同。‘现代国家’确实各不相同的。普鲁士德意志帝国同瑞士完全不同,英国同美国也完全不同。所以,‘现代国家’只是一种虚构的概念。

但是,不管国家的形式如何纷繁,各个不同的文明国家却有一个共同点:它们都建筑在资本主义多少已有发展的现代资产阶级社会的基础上。所以它们具有某些极重要的共同特征。在这个意义上,同现在国家的根基资产阶级社会已经消亡的未来相对来说,也可以说‘现代国家’。

其次,还有这样一个问题:在共产主义社会里国家制度会发生怎样的变化呢?换句话说,那时会有哪些同现代国家职能相类似的社会职能保留下来呢?这个问题只有用科学的方法才能解答;否则,即使你千百次把‘人民’和‘国家’这两个名词连在一起,也丝毫不会对这个问题的解决有所帮助”······$^{22}$

%\pskip
\normalsize
\leftskip=0mm

马克思这样讥笑了关于“人民国家”的一切空话,提出了问题,并且好像是警告说;要对这个问题作出科学的解答,只有依靠确切证明了的科学材料。

十分确切地由整个发展论和全部科学证明了的首要的一点,也是从前被空想主义着所遗忘、现在又被害怕社会主义革命的机会主义者所遗忘的那一点,就是在历史上必然会有一个从资本主义向共产主义{\kaishu 过渡}的特别时期或特别阶段。





