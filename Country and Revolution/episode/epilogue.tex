\chapter{初版跋} % 小章节名称
%\blfootnote{} % 底线批注,在所需段落前加入
%\pskip % 跳大行,段落间用
%\noindent % 书信体的开头
%\mbox{} % 插入个盒子,设定好的大小,用于隔开文本
%\leftskip= 文本左边腾出空格,用于文本排版
% \CJKunderdot{\textbf{}} % 文字下加点

\id 

这本小册子是在1917年8、9两月写成的。我当时已经拟定了下一侧即第七章“1905年和1917年俄国革命的经验”的大纲。但是,除了题目以外,我连一行字也没有来得及写,因为1917年十月革命前夜的政治危机“妨碍”了我。对于这种“妨碍”,只有高兴。但是本书第二部分(关于“1905年和1917年俄国革命的经验”)也许要拖延很久才能写出,因为做出“革命的经验”总比论述“革命的经验”更愉快,更有益。

\pskip
\leftskip=65mm

作者

\pskip
\leftskip=45mm

1917年11月30日与彼得格勒

\pskip
\leftskip=50mm
\small

写于1917年8月-9月

\pskip

1918年由“生活和知识”出版社

出版单行本

\pskip

按1919年“共产党人”出版社

出版的小册子原文刊印,并根据

手稿和1918年的版本做过校订

\leftskip=0mm
\normalsize