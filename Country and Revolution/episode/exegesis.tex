\chapter{注\qquad 释} % 小章节名称
%\blfootnote{} % 底线批注,在所需段落前加入
%\pskip % 跳大行,段落间用
%\noindent % 书信体的开头
%\mbox{} % 插入个盒子,设定好的大小,用于隔开文本
%\leftskip= 文本左边腾出空格,用于文本排版
% \CJKunderdot{\textbf{}} % 文字下加点
% {\kaishu } % 更改为楷书字体
% itemize 为实心圆符号

\small

\begin{enumerate}
	\item {\kaishu “国家与革命”}\quad 一书是列宁1917年8-9月在秘密状态中写成的。列宁在1916年下半年就认为有必要从理论上探讨国家问题。当时他写了一篇短评“青年国际”(见“列宁全集”1958年人民出版社版第23卷第163-167页),在这篇短评里他批评了布哈林在国家问题上的反马克思主义立场,并答应写一篇文章详细地论述马克思主义对国家问题的观点。列宁在1917年2月17日(新历)给亚·米·柯伦泰的信中谈到,他差不多已经把马克思主义关于国家问题的材料准备好了。这些材料是用很小的字体写在以“马克思主义论国家”为标题的蓝皮笔记本里的。笔记中的材料有的摘自马克思和恩格斯的著作,有的摘自考茨基、潘涅库克和伯恩施坦的著作并有列宁的评语、结论和概括。
	
	“国家与革命”一书按原定计划有7章,最后一章,即第七章“1905年和1917年俄国革命的经验”列宁没有写,保存下来的只有这一章的详细提纲(见“列宁文集”1933年俄文版第21卷第25页-26页)。关于出版这本书的问题,列宁在给出版者的信中写道,如果他“实在来不及写完第七章,或者这本书太厚,那么可以把前6章单独出版,作为第一分册······”
	
	在手稿的第1页上,作者用的笔名是“弗·弗·伊万诺夫斯基”。列宁预计用这个笔名可以出版这本书,不然就会被临时政府没收。但是这本书到1918年才出版,因此也就不需要再用笔名了。第2版是在1919年出的,列宁在第二章中加了“1852年马克思对问题的提法”一节。——(扉页)
	
	\item {\kaishu 费边社分子}\quad 是一部分英国资产阶级知识分子于1884年成立的改良主义的、极端机会主义的“费边社”的成员。该社以古罗马大将费边·孔克达特(“缓进者”)命名。费边素以缓进待机、回避决战著称。用列宁的话说,费边社“最完备地表达了机会主义和自由主义所采取的工人政策”。费边社分子诱惑无产阶级脱离阶级斗争,鼓吹用改良方法由资本主义和平地、逐步地过渡到社会主义。在帝国主义世界大战期间(1914-1918年),费边社分子采取了社会沙文主义的立场。关于费边社分子的评定,可以看列宁的如下著作:“‘约·菲·贝克尔、约·狄慈根、弗·恩格斯、卡·马克思等致弗·阿·左尔格等书信集’俄译本序言”(“列宁全集”1959年人民出版社版第12卷第354-355页)、“社会民主党在俄国革命中的土地纲领”(“列宁全集”1959年人民出版社版第15卷第151页)、“英国的和平主义和英国的不爱理论”(“列宁全集”1959年人民出版社版第21卷第237-238页)等。——(正文第1页)
	
	\item 见恩格斯的“家庭、私有制和国家的起源”(“马克思恩格斯文选”(两卷集)1955年莫斯科中文版第2卷第316页)
	
		在本书第6、8-12页上,列宁引用的也是恩格斯的这一著作(同上第317-320页)。——(正文第4页)
	\item 见恩格斯的“反杜林论”1948年俄文版第264-265页(1956年人民出版社版第294-295页)。
	
		在本书第16页上,列宁引用的也是恩格斯的这一著作(同上第190页)。——(正文第13页)
	\item 见马克思的“哲学的贫困”1941年俄文版。——(正文第17页)
	\item 见马克思和恩格斯的“共产党宣言”(“马克思恩格斯文选”(两卷集)1954年莫斯科中文版第1卷第8-41页)。——(正文第17页)
	\item 见马克思的“哥达纲领批判”(“马克思恩格斯文选”(两卷集)1955年莫斯科中文版第2卷第11-47页)。
	
		{\kaishu 哥达纲领}是1875年在哥达举行的爱森纳赫派和拉萨尔派(当时这两派都是德国独立存在的社会主义政党)联合代表大会上通过的德国社会主义工党纲领。这是一个彻头彻尾的机会主义纲领,因为爱森纳赫派在一切重大问题上向拉萨尔派让步,接受拉萨尔派的意见。马克思和恩格斯给予哥达纲领以歼灭性的批判。——(正文第17页)
	\item 见马克思的“哲学的贫困”1941年俄文版第148-149页——(正文第19页)
	\item 见马克思和恩格斯的“共产党宣言”1948年俄文版第63、78页(“马克思恩格斯文选”(两卷集)1954年莫斯科中文版第1卷第20、28页)——(正文第19页)
	\item 见马克思的“拿破仑第三政变记”(“马克思恩格斯文选”(两卷集)1954年莫斯科中文版第1卷第308-309页)。
	
		在本书第25-27页上,列宁引用了恩格斯为该书第3版写的序言(同上第221-222页)。——(正文第23页)
	\item {\kaishu “新时代”}(《Die Neue Zeit》) 杂志是德国社会民主党的杂志,1883年至1823年在斯图加特出版。1885-1895年“新时代”杂志曾发表过恩格斯的几篇文章。恩格斯经常向该杂志编辑部提出意见,并因它背弃马克思主义而给予尖锐的批评。从90年代后半期起,即在恩格斯逝世以后,该杂志系统地刊载了修正主义者的文章。在帝国主义世界大战期间(1914-1918年),该杂志采取了考茨基中派立场,支持社会沙文主义者。——(正文第28页)
	\item 见“马克思恩格斯书信选集”1953年俄文版第63页。——(正文第28页)
	\item 见马克思和恩格斯的“共产党宣言”1948年俄文版第9页(“马克思恩格斯文选”(两卷集)1954年莫斯科中文版第1卷第2页)。——(正文第31页)
	\item 见“马克思恩格斯书信选集”1953年俄文版第263页。——(正文第32页)
	\item 见马克思的“法兰西内战”(“马克思恩格斯文选”(两卷集)1954年莫斯科中文版第1卷第498-499页)。
		
		在本书第38、38-39、44-48页上,列宁引用的也是马克思的这一著作(同上第498-500、501页)。——(正文第36页)
	\item 见恩格斯的“论住宅问题”(“马克思恩格斯文选”(两卷集)1954年莫斯科中文版第1卷第549-550页)。
		在本书第50-51、51页上,列宁引用的也是恩格斯的这一著作(同上第605、589页)。——(正文第50页)
	\item 列宁指的是马克思的“政治上的漠不关心”一文和恩格斯的“论权威”一文(见“马克思恩格斯全集”1935年俄文版第15卷第88-95页和第134-137页)。
		
		在本书第52、53、54-55页上,列宁引用的也是这两篇文章(同上第88-91、136、136-137页)。——(正文第52页)
	\item 见“马克思恩格斯书信选集”1947年俄文版第296页。——(正文第57页)
	\item {\kaishu 爱尔福特纲领}是德国社会民主党于1891年10月在爱尔福特召开的代表大会上通过的,用来代替1875年的哥达纲领。恩格斯在“对1891年社会民主党纲领草案的批判”(见“马克思恩格斯全集”1936年俄文版第16卷第2部第101-116页)一文中批判了爱尔福特纲领的错误。
		
		在本书第59-65页上,列宁引用的也是恩格斯的这篇文章(同上第105-111页)。——(正文第59页)
	\item 指恩格斯为马克思的“法兰西内战”一书所写的序言(见“马克思恩格斯文选”(两卷集)1954年莫斯科中文版第1卷第452-464页)。
	
		在本书第66、67、67-70页上,列宁引用的也是恩格斯的这篇序言(同上第454、458、462-463页)。——(正文第66页)
	\item 见“马克思恩格斯全集”1936年俄文版第16卷第2部第386-387页。——(正文第71页)
	\item 见马克思的“哥达纲领批判”(“马克思恩格斯文选”(两卷集)1955年莫斯科中文版第2卷第30-31页)。
	
		在本书第76、81、82-84页上,列宁引用的也是马克思的这篇文章(同上第31、21-23页)。——(正文第75页)
	\item {\kaishu 第一国际海牙会议} 于1872年9月2-7日(新历)举行。马克思和恩格斯出席了这次代表大会。参加大会的有65名代表。列入议程的问题有:(1)关于总委员会的权利;(2)关于无产阶级的政治活动等。代表大会的全部工作是在同巴枯宁派作尖锐的斗争中进行的。大会通过了关于扩大总委员会权利的决议。代表大会就“无产阶级的政治活动”问题所通过的决议中说到:无产阶级应当组织自己的政党以保证社会革命的胜利,无产阶级的伟大任务就是夺取政权。在这次代表大会上,巴枯宁和吉约姆被开除出国际,因为他们是破坏者和新的反无产阶级政党的组织者。——(正文第92页)
	\item {\kaishu “曙光”}杂志是马克思主义的科学政治刊物;由“火星报”编辑部于1901-1902年在斯图加特出版,共出版了4期(3册)。在“曙光”杂志上刊载了列宁的下列文章:“时评”、“地方自治机关的迫害者和自由主义的汉尼拔”、“土地问题和‘马克思的批评家’”一书的前4章(标题为“土地问题的‘批评家’先生们”)、“内政评论”和“俄国社会民主党的土地纲领”。——(正文第93页)
	
	\item 指1900年9月23-27日(新历)在巴黎举行的第二国际第五次国际社会党人代表大会。参加大会的有791名代表。俄国代表团由23人组成。在无产阶级夺取政权这一根本问题上,代表大会以多数票通过了考茨基提议的列宁称之为“对机会主义者采取调和态度的”决议。除了其他决议以外,代表大会还决定设立国际社会党执行局,由各国社会主义政党的代表组成,并设书记处于布鲁塞尔。——(正文第94页)
	\item {\kaishu “社会主义月刊”}(《Sozialistsche Monatshefte》 )是德国社会民主党机会主义者的主要机关刊物,国际机会主义的机关刊物之一;在帝国主义世界大战期间(1914-1918年),它采取了社会沙文主义立场。该月刊从1897年到1933年在柏林出版。——(正文第107页)
	\item {\kaishu 英国独立工党}(Independent Labour Party)于1893年成立。该党领袖是詹姆斯·凯尔-哈第、拉·麦克唐纳等。独立工党自命在政治上不依赖资产阶级政党,实际上是“不依赖社会主义,而以来自由主义”({\kaishu 列宁})。在帝国主义世界大战期间(1914-1918年)独立工党最初发表宣言反对战争(1914年8月13日【新历】)。之后,在1914年2月协约国社会主义者伦敦代表会议上,独立党人同意代表会议通过的社会沙文主义的决议。从那时起,独立党首领以和平主义的词句打掩护,采取了社会沙文主义立场。1919年共产国际成立后,在左派党员群众的压力下,独立工党的首领通过了退出第二国际的决议。1921年独立党人参加了所谓第二半国际,在第二半国际瓦解后,他们重新加入了第二国际。1921年,英国独立工党的左翼脱离该党,加入了英国共产党。——(正文第107页)
\end{enumerate}

\begin{center}
	\rule[-1pt]{2.5cm}{0.1em}
\end{center}