\chapter{国家是剥削和被压迫阶级的工具} % 小章节名称
%\blfootnote{} % 底线批注,在所需段落前加入
%\pskip % 跳大行,段落间用
%\noindent % 书信体的开头
%\mbox{} % 插入个盒子,设定好的大小,用于隔开文本
%\leftskip= 文本左边腾出空格,用于文本排版
% \CJKunderdot{} % 文字下加点
% {\kaishu } % 更改字体为楷书

为了维持驾于社会之上的特别社会权力,就需要捐税和国债。

\pskip
\leftskip=10mm
\small

恩格斯说:“官吏既然掌握着社会权力和征税权,他们就会成为社会机关而驾于社会之上。从前人们对氏族(或宗族)社会机关的那种自愿的敬意,即使他们能够获得,也不能使他们满足了。”······于是制定了官吏是神圣不可侵犯的特别法律。“一个微不足道的警察”却有大于氏族代表的“权威”,然而,即使是文明国家掌握军权的首脑,也会对“不是用强迫手段获得社会尊敬”的氏族首领表示羡慕。

\leftskip=0mm
\normalsize
\pskip

这里提出了作为国家政权机关的官吏的特权地位问题。指出了这样一个基本问题:究竟什么东西使他们驾于社会之上?我们在下面就会看到,1871年巴黎公社如何实际地解决了这个理论问题,而在1912年又如何被考茨基反动地抹杀了。

\pskip
\leftskip=10mm
\small

······“因为国家是为了控制阶级对抗而产生的,同时又是在这种阶级冲突中产生的,所以,它照例是最强大的、在经济上占统治地位的阶级的国家,这个阶级借助与国家而在政治上也成为占统治地位的阶级,因而获得了压制和剥削被压迫阶级的新手段”······不仅古代的国家和封建国家是剥削奴隶和农奴的机关,“现代的代议制国家也是资本剥削雇佣劳动的工具。但是也常有一些例外,如相互斗争的阶级达到势均力敌的地步,国家政权就暂时获得某种独立性,似乎成了这两个阶级之间的中介人”······17世纪和18世纪的君主专制,法国第一帝国和第二帝国的拿破仑主义,德国的俾斯麦时代,都是如此。

\leftskip=0mm
\normalsize
\pskip

\blfootnote{[5]亚历山大·弗多洛维奇·克伦斯基(А.В Керенский,1881-1970)——俄社会革命党人,发动二月革命。}
我们还可补充说,在共和制俄国的克伦斯基$^{[5]}$政府还是压迫革命无产阶级以后,由于小资产阶级民主派的领导,苏维埃{\kaishu 已经}软弱无力,而资产阶级{\kaishu 还}没有足够的力量来直接解散苏维埃的时候,也如此。

\pskip
\leftskip=10mm
\small

恩格斯又说,在民主共和国内,“财富是间接地发挥它的权力地,因此是更可靠的”,它所采用地第一个方法是“直接收买官吏”(美国),第二个方法是“政府同交易所结合”(法国和美国)。

\leftskip=0mm
\normalsize
\pskip

\blfootnote{[6]法文译词(syndicat),原意是“组合”,是垄断组织的重要形式之一}
\blfootnote{[7]上述人物为社会革命党或孟什维克派人士}
目前,任何最民主地共和国中地帝国主义和银行统治,都把这两种维护和实现财富地无限权力地方法“发展”到了非常巧妙地地步。例如,在俄国实行民主共和的头几个月里,在社会革命党人和孟什维克这两种“社会主义者”同资产阶级联姻的“蜜月”期间,帕尔钦斯基先生在联合政府中实行怠工,不愿实施制裁资本家、制止他们进行掠夺和借军事订货盗窃国库的种种措施,在帕尔钦斯基先生退出内阁以后(接替他的自然是同他一模一样的人),资本家“奖赏”给他年薪12万卢布的肥缺,试问这究竟是怎么一回事呢?是直接的收买,还是间接的收买?是政府同辛迪加$^{[6]}$勾结,还是“仅仅”是一种友谊关系?切尔诺夫、策烈铁里、阿夫克森齐耶夫、斯柯别列夫$^{[7]}$之流究竟起着什么作用?他们是盗窃国库的百万富翁的“直接”同盟者,还是仅仅是间接的同盟者?

“财富”的无限权力在民主共和制度下之所以更{\kaishu 可靠},是因为它不依赖资本主义的不好的政治外壳。民主共和制度是资本主义所能采用的最好的政治外壳,所以资本一掌握(通过帕尔钦斯基、切尔诺夫、策烈铁里之流)这个最好的外壳,就能十分可靠十分巩固地确立自己地权力,在资产阶级民主共和国中,无论人员、机关或政党地{\kaishu 任何}更换,都不会使这个权力动摇。

还应该指出,恩格斯十分肯定地认为,普选制是资产阶级统治的工具。他显然是估计了德国社会民主党的长期经验,他说普选制是
\pskip
\leftskip=10mm
\small

“工人阶级成熟的指标。在现代国家,普选制不能而且永远不会提供更多的东西”。

\leftskip=0mm
\normalsize
\pskip

小资产阶级民主派,如我国社会革命党人和孟什维克,及其同胞兄弟西欧一切社会沙文主义者和机会主义者,都希望从普选制中得到“更多的东西”。他们自己相信而且要人民也相信这种荒谬的想法,似乎普选制“在{\kaishu 现代}国家中”真正能够体现大多数劳动者的意志,并保证实现这种意志。

我们在这里只是指出这个错误的想法,只是指出,恩格斯这个十分明确而具体的说明,经常在“正式的”(即机会主义的)社会主义政党的宣传活动中遭到歪曲。至于恩格斯如何揭露这种想法的全部虚伪性,我们以后在谈到马克思和恩格斯对“{\kaishu 现代}”国家的看法时,还会详细地加以阐明。

恩格斯在他那部最通俗的著作中,把自己的看法总结如下:
\pskip
\leftskip=10mm
\small

“由此可见,国家不是自古就有的。曾经有过不需要国家、而且根本不知国家和国家政权为何物的社会。在经济发展到一定阶段而必然使社会分裂为几个阶级时,国家就成为必要了。现在我们正以迅速的步伐走上这样的生产发展阶段,在这个阶段上,这些阶级的存在不仅已经没有必要,而且成了生产的直接障碍。阶级必然会消失,正如他们从前必然会产生一样。随着阶级的消失,国家也必然会消灭。以生产者自由平等的联合体为基础的,按新方式来组织生产的社会,将把全部国家机器放到它应该去的地方,即放到古物陈列馆去,同纺车和青铜斧陈列在一起。”

\leftskip=0mm
\normalsize
\pskip

这一段引文在现代社会民主派的宣传鼓动书刊中很少看到,即使引用也多半是为了崇拜偶像,也就是说,为了正式表示对恩格斯的尊敬。他们根本不去考虑,先要经过怎样广泛而深刻的革命,才能“把全部国家机器放到古物陈列馆去”。甚至他们往往不懂恩格斯说的国家机器是什么。





